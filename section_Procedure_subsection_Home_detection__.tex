% Procedimiento:
	% Obtener la lista de antenas de GC
	% Determinar la casa de los usuarios
	% Determinar, para cada usuario, si se comunicó con un habitante de GC
	% Determinar las agregaciones por antena
	% Visualizar (agrego alfa, min_volume y zoom en regiones)

\section{Procedure}
\subsection{Home detection}
	\begin{block}{Home Prediction}
		\begin{itemize}
			\item As a first step, we determined each user's residence antenna.
			%Como primer paso, determinamos para cada usuario, su lugar de residencia.
			%No es la ubicación de su casa exactamente, pero es una buena aproximación. % Aclarar más esto? Aclararlo menos? %creo que esta bien
			
			\item This was chosen to be the most used frequently used antenna, considering only calls made in week evenings.
			%La antena elegida como hogar es la más frecuentemente usada, 
			%considerando llamados nocturnos en días de semana.
			
			\item The hypothesis: most of people are at home on any given weeknight. 
			%La hipótesis: la mayoría de los días, las personas se encuentran en sus casas durante la noche.
			
			%\bigskip
			%Así, cada antena queda asociada a un conjunto de usuarios: sus \textit{habitantes}.
			\item Note: users for which the inferred home antenna is located in \textit{Gran Chaco} (the risk area) will
			be considered the set of \textit{residents of Gran Chaco}. 
			%Nota: Los usuarios cuya antena inferida pertenece al Gran Chaco (la zona de riesgo)
			%se consideran el conjunto de \textit{habitantes de Gran Chaco}.
			
		\end{itemize}
	\end{block}

% Tambien aclarar que obtenemos el conjunto de usuarios que viven en la zona endemica
\subsection{Detection of vulnerable users}
% Estos son los que sacamos con la difusion
\subsection{Heatmap generation}