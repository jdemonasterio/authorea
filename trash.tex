%
% AGRANDA 2016 Camera ready version!
%

\documentclass{article}%{llncs}

\usepackage{graphicx}
\usepackage[space]{grffile}
\usepackage{latexsym}
\usepackage{textcomp}
\usepackage{longtable}
\usepackage{booktabs}
% \usepackage{row}
\usepackage{amsfonts,amsmath,amssymb} %paquetes de matematica
\usepackage{amsthm}
\usepackage{url}
% \usepackage{hyperref}d
% \hypersetup{colorlinks=false,pdfborder={0 0 0}}
% You can conditionalize code for latexml or normal latex using this.
% \newif\iflatexml\latexmlfalse
\usepackage[utf8]{inputenc}
\usepackage[english]{babel}
\usepackage{algorithm2e}
\usepackage{qtree}
\usepackage{amssymb}
\usepackage{hyperref}

% \setcounter{tocdepth}{3}
%\usepackage{graphicx}
% \usepackage{subfigure}
\usepackage{nicefrac}

% for confusion matrix building
\usepackage{array}
\usepackage{multirow}

\newcommand\MyBox[2]{
\fbox{\lower0.75cm
	\vbox to 1.7cm{\vfil
		\hbox to 1.7cm{\hfil\parbox{1.4cm}{#1\\#2}\hfil}
		\vfil}%
}%
}


% mathcal certain letters
\newcommand{\calG}{\mathcal{G}}
\newcommand{\calN}{\mathcal{N}}
\newcommand{\calE}{\mathcal{E}}
\newcommand{\calL}{\mathcal{L}}

% \renewcommand{\labelitemi}{$\bullet$}
\setlength{\tabcolsep}{6pt}

%comandos de operadores de esperanza, varianza etc.
%\newcommand{\Expect}{{\rm I\kern-.3em E}}
\newcommand{\Expect}{{\mathbb{E}}}

\newcommand{\Var}{\mathrm{Var}}
\newcommand{\Cov}{\mathrm{Cov}} 

%comandos de teoremas, corolarios, lemmas, pruebas y definiciones
\newtheorem{theorem}{Theorem}[section]
\newtheorem{corollary}{Corollary}[theorem]
\newtheorem{lemma}[theorem]{Lemma}
\newtheorem{definition}{Definition}[subsection]
\theoremstyle{definition}


\begin{document}

% \mainmatter  % start of an individual contribution
% first the title is needed
\title{Trash Tex file to test different formulas and stuff}




\author{
	Juan de Monasterio
	\and Carlos Sarraute
}

%
% NB: a more complex sample for affiliations and the mapping to the
% corresponding authors can be found in the file "llncs.dem"
% (search for the string "\mainmatter" where a contribution starts).
% "llncs.dem" accompanies the document class "llncs.cls".
%

%\toctitle{Lecture Notes in Computer Science}
%\tocauthor{Uncovering the Diffusion of an Infectious Disease with Mobile Phone Data
%%Unveiling Chagas with Big Data}

\maketitle
\begin{abstract}
	Your text should have one idea per paragraph and use transitions between sentendes. The following linnk provides a whole set of transition words to use.
	\url{http://grammar.ccc.commnet.edu/Grammar/transitions.htm}
	
	
%%%% HOW TO USE SPLITS IN EQUATIONS (same level for equal signs)
%\begin{split}
%	var(mr) & =  \Expect_{\Theta, \Theta'} 
%	\left[ 
%	cov_{\textbf{x},\textbf{y}}
%	(rmg(\Theta,\textbf{x},\textbf{y} )rmg(\Theta',\textbf{x},\textbf{y} )) 
%	\right]
%\end{split}

%%%% Equations with no \$ chars
%\[
%l(\theta) = \sum_{i=1}^N \big(y_i log(P(y_i|x_i,\theta)) + (1-y_i)log(1 - P(y_i|x_i,\theta)) \big)
%\]

%%%% Equations with no \$ chars
%\begin{equation}
%l(\theta) = \sum_{i=1}^N \big(y_i log(P(y_i|x_i,\theta)) + (1-y_i)log(1 - P(y_i|x_i,\theta)) \big)
%\end{equation}

%\begin{definition}{Vapnik–Chervonenkis (VC) Dimension}
%\end{definition}
\end{abstract} 




\section{Classifier : Naive Bayes}

The Naive Bayes model encompasses a group of simple and computationally efficient algorithms which are based on a strong statistical independence among the features. Even though this assumption is in practice wrong, the model still achieves acceptable classification rates for some problems. In addition, it does not suffer in problems of high-dimensionality, where $p >> n$.  

It is presented here mostly for computational benchmark purposes, where in practice the classification rate achieved by this model serves as a baseline for other, more complex, learners. Furthemore, the algorithm has linear  complexity in the number of features and samples $O(d+n)$, so it can be easlity extended to \textit{bigger} problem implementations. Furthermore, its maximum-likelihood estimation of the parameters has a closed form solution which is faster to compute over other iterative methods such as gradient descent techniques.

Let $x = (x_1,...,x_p)$ be any given data sample and $C_k$ be one of $K$ possible output classes of a classification problem. We take $p(C_k \mid x)$  to be the  class posterior probability of this class given the sample. 

In section \ref{section-example}
we have used 
\[
p(C_k| x) = \frac{P(x|C_k)P(C_k)}{P(x)}
\]\label{equation-posteriorProbabilties}

and argued that if our data is given, then our model can only improve the posterior probability by optimizing $P(x|C_k)P(C_k)$ which is just the joint probability of the sample and the class. 

Here we can approximate the posterior as

\[
P(C_k \mid x) \approx p(C_k) * \prod_{j=1}^{p}    P(x_j \mid \bigcap_{k=j+1}^{p} x_k \cap C_k)
\]\label{equation-posteriorProbabilityDecomposition1}

We now impose a strong independence assumption among features, given the target class, to let the conditional probabilities factors become the probability of each feature. %This assumption is what gives the model its 

This yields a posterior probability which depends only on the prior probability and on the individual likelihood of each feature.

\[
P(C_k \mid x) \approx p(C_k) * \prod_{j=1}^{p}    P(x_j | C_k)
\]\label{equation-posteriorProbabilityDecomposition2}

As we have said before, the parameters of the model can only reweigh the likelihood factors, so if we look to maximize the posterior probability, our final estimate of the posterior will take the following form.

\[
P(C_k \mid x) = \frac{1}{Z} p(C_k) * \prod_{j=1}^{p}    P(x_j | C_k)
\]\label{equation-posteriorProbabilityDecomposition3}

where in the equation $Z = p(x)$ is a scaling factor and is fixed to the dataset.

In practice, the model will stem into different algorithms where each variant will have a different probabilistic assumption on the likelihoods $p(x_j \mid C_k)$ of the model and on the priors $p(C_k)$. It is common to choose among using a nonparametric density estimations from the data or assuming that the data comes from an exponential family distribution such as a Gaussian, Bernoulli or Multinomial distributions. Different choices will certainly lead to different cross validation scores among problems. Altogether, these choices can be treated as part our model's hyperparamters and the best one can be selected with our CV procedure. 
%Indeed, 


Finally, the output class for a given sample will be given by taking the class $k'$ which maximizes the probability  $P(C_k' \mid x)$. 
\end{document}

