
%
% AGRANDA 2016 Camera ready version!
%

\documentclass{article}%{llncs}

\usepackage{graphicx}
\usepackage[space]{grffile}
\usepackage{latexsym}
\usepackage{textcomp}
\usepackage{longtable}
\usepackage{booktabs}
% \usepackage{row}
\usepackage{amsfonts,amsmath,amssymb} %paquetes de matematica
\usepackage{amsthm}
\usepackage{url}
% \usepackage{hyperref}d
% \hypersetup{colorlinks=false,pdfborder={0 0 0}}
% You can conditionalize code for latexml or normal latex using this.
% \newif\iflatexml\latexmlfalse
\usepackage[utf8]{inputenc}
\usepackage[english]{babel}
\usepackage{algorithm2e}
\usepackage{qtree}
\usepackage{amssymb}
\usepackage{hyperref}

% \setcounter{tocdepth}{3}
%\usepackage{graphicx}
% \usepackage{subfigure}
\usepackage{nicefrac}

% for confusion matrix building
\usepackage{array}
\usepackage{multirow}

\newcommand\MyBox[2]{
	\fbox{\lower0.75cm
		\vbox to 1.7cm{\vfil
			\hbox to 1.7cm{\hfil\parbox{1.4cm}{#1\\#2}\hfil}
			\vfil}%
	}%
}


% mathcal certain letters
\newcommand{\calG}{\mathcal{G}}
\newcommand{\calN}{\mathcal{N}}
\newcommand{\calE}{\mathcal{E}}
\newcommand{\calL}{\mathcal{L}}

% \renewcommand{\labelitemi}{$\bullet$}
\setlength{\tabcolsep}{6pt}

%comandos de operadores de esperanza, varianza etc.
%\newcommand{\Expect}{{\rm I\kern-.3em E}}
\newcommand{\Expect}{{\mathbb{E}}}

\newcommand{\Var}{\mathrm{Var}}
\newcommand{\Cov}{\mathrm{Cov}} 

%comandos de teoremas, corolarios, lemmas, pruebas y definiciones
\newtheorem{theorem}{Theorem}[section]
\newtheorem{corollary}{Corollary}[theorem]
\newtheorem{lemma}[theorem]{Lemma}
\newtheorem{definition}{Definition}[subsection]
\theoremstyle{definition}

\begin{document}
	
	% \mainmatter  % start of an individual contribution
	
	% first the title is needed
	\title{Trash Tex file to test different formulas and stuff}
	
	
	
	
	\author{
		Juan de Monasterio
		\and Carlos Sarraute
	}
	
	%
	% NB: a more complex sample for affiliations and the mapping to the
	% corresponding authors can be found in the file "llncs.dem"
	% (search for the string "\mainmatter" where a contribution starts).
	% "llncs.dem" accompanies the document class "llncs.cls".
	%
	
	%\toctitle{Lecture Notes in Computer Science}
	%\tocauthor{Uncovering the Diffusion of an Infectious Disease with Mobile Phone Data
	%%Unveiling Chagas with Big Data}
	
	\maketitle
	\begin{abstract}
		
		Texstudio is better at helping to write fast tex documents using spell-checkers, pre-loaded commands and \textit{stuff}. All of the following text is just copy/paste trash being tested before being added to a real doc.
		
		%%%% HOW TO USE SPLITS IN EQUATIONS (same level for equal signs)
		%\begin{split}
		%	var(mr) & =  \Expect_{\Theta, \Theta'} 
		%	\left[ 
		%	cov_{\textbf{x},\textbf{y}}
		%	(rmg(\Theta,\textbf{x},\textbf{y} )rmg(\Theta',\textbf{x},\textbf{y} )) 
		%	\right]
		%\end{split}
		
		%%%% Equations with no \$ chars
		%\[
		%l(\theta) = \sum_{i=1}^N \big(y_i log(P(y_i|x_i,\theta)) + (1-y_i)log(1 - P(y_i|x_i,\theta)) \big)
		%\]
		
		%%%% Equations with no \$ chars
		%\begin{equation}
		%l(\theta) = \sum_{i=1}^N \big(y_i log(P(y_i|x_i,\theta)) + (1-y_i)log(1 - P(y_i|x_i,\theta)) \big)
		%\end{equation}
		
		%\begin{definition}{Vapnik–Chervonenkis (VC) Dimension}
		%\end{definition}
		
		
	\end{abstract} 
	
	\subsection{Extension to Random Forests}
	\textbf{L Breiman Paper}\cite{breiman-randomforests}
	
	
	Random Forests where a natural ensemble extension of Decision Trees as base learners. First described by \cite{HoFirstRandomForest} they combine a number of trees with a technique called bootstrap aggregation or \textit{bagging}. 
	
	This algorithm is run by \textit{randomizing } the way we build each base learner. For example, we could select i.i.d samples from the training set to build each specific tree, or we could sample a subset of features at each decision node. Also, when splitting the data at a tree node, we could choose a random split among the $K$ best splits at that stage.
	
	When using a base learner which has low-bias and high-variance, bagging the estimators will produce a new learner which has a reduced overall variance and slightly worse bias than each single tree. Note that if we have a selection of $K$ i.i.d. random variables with common variance $\sigma^2$, then the average of this selection will have variance $\frac{\sigma^2}{K}$. This is the idea behind ensembling all of the trees.
	
	
	The generalization error for the forests will converge almost surely as we increase the number of base estimators used to train the model. This error will depend uniquely on the predictive error of each base learner and collectively from the correlation among them.
	
	In classification problems the forest will output a solution by making the base learners vote on the  target class. For regression problems, the average of the outputs will be taken as the output of the algorithm. The procedure for sampling new predictions is fast once the forest has been constructed, but they are computationally heavy structures.
	
	Given $K$ number of trees, let $\Theta_k$ encode the parameters for the $k$-th tree and $h(\textbf{x},\Theta_k)$ the corresponding classifier. Let $N$ be the number of samples in the training set. The creation of a random forests involves an iterative procedure where at each $k$ step $\Theta_k$ is drawn from the same distribution but independently of the previous parameters $\Theta_1, \ ..., \ \Theta_{k-1}$ created at previous steps. $\Theta$ will be encoded by a vector of randomly drawn integers from 1 to $M$ which is part of the model's hyperparameters.
	
	Let $\{h_k(\textbf{x})\}_{i=1}^K$  \footnote{There is an abuse of notation by noting trees as $h_k(\textbf{x}$ and not $h(\textbf{x}, \Theta_k)$ } be a set of classifying trees and let $I$ denote the indicator function.  Define the margin function as
	
	$$mg(\textbf{x},\textbf{y}) =  \frac{1}{K}   \sum_{k=1}^K I(h_k(\textbf{x}) = \textbf{y})  
	- max_{j\neq \textbf{y}}\left(\frac{1}{K} \sum_{k=1}^K I(h_k(\textbf{x}) = j) \right) $$ \label{eq:rf-marginFun}
	
	%\end{equation}
	$ P_{\textbf{x}, \textbf{y} }(mg(\textbf{x}, \textbf{y}) < 0) $
	
	
	The function measures, in average, how much do the trees vote for the correct class in comparison to all other classes and it can be shown that when $K$ is large then the prediction error converges almost surely to 
	
	$$ P_{\textbf{x}, \textbf{y} } ( P_{\Theta} (h(\textbf{x}, \Theta) = \textbf{x}) - max_{j \neq \textbf{y}} (\textbf{x}, \textbf{y}) < 0) $$
	
	\subsubsection{Proof}
	The proof follows from seeing that given a training set, a parameter $\Theta$ and a class $j$ then 
	$$\forall \textbf{x} \lim_{K\to\infty} \frac{1}{K} \sum_{k=1}^K I(h_k(\textbf{x}) = j) \ =   \ P_\Theta(h(\theta,\textbf{x}) = j) $$
	almost surely.
	
	By looking at the nature of each tree, $\{\textbf{x} / h_k(\textbf{x}, \Theta) = j \}$ denotes a union of hyper-rectangles partitioning feature space. And given the finite size of the training set, there can only be a finite set of these unions. Let $S_1, ..., S_K$ be an indexation of these unions and define $\phi(\Theta) = k $ if $\{\textbf{x} / h(\textbf{x}, \Theta) = j \} = S_k$. 
	
	We denote by $N_k$ the number of times that $\phi(\Theta_n) =k $, where $n \in {1...N}$ and $N$ is the total of trials.
	
	It is immediate that 
	
	$$ \frac{1}{N} \sum_{n=1}^N I(h_n(\textbf{x}) = j) \ = \  \frac{1}{N} \sum_{k=1}^K N_k I(\textbf{x} \in S_k)  $$
	
	and that there is a convergence almost everywhere of $$ \frac{N_k}{N} = \sum_{n=1}^N  I(\phi(\Theta_n) = k)  \xrightarrow[N \to \infty]{}   P_{\Theta}(\phi(\Theta)= k)$$. 
	
	If we let $C = $ $\bigcup\limits_{k=1}^{K} C_{k}$ where each $C_k$ are zero-measured sets representing the points where the sequence is not converging. We will finally have that  outside of $C$, 
	
	$$ \frac{1}{N} \sum_{n=1}^N I(h_n(\textbf{x}) = j) \xrightarrow[N \to \infty]{} \sum_k^K    P_{\Theta}(\phi(\Theta)= k) I(\textbf{x} =j ) \ = \ P_{\Theta}(h(\textbf{x}, \Theta) = j)  $$ 
	
	
	
	
	
	\subsection{Predictive error bounds}
	
	Random Forests are built upon a bag of weaker classifier, of which each individual estimator has a different prediction error. To build an estimate on the generalization error of the ensemble classifier, these individual scores and the inter-relationship between them must be taken into account. For this purpose, the \textit{strength} and \textit{correlation} of a Random Forest must be analyzed to arrive on an estimate of the generalization error.
	
	%\begin{lemma}
	%Given two line segments whose lengths are $a$ and $b$ respectively there is a 
	%real number $r$ such that $b=ra$.
	%\end{lemma}
	
	\begin{theorem}
		There is an upper bound for the generalization error.
	\end{theorem}
	
	Define $\hat{j}(\textbf{x},\textbf{y})$ as $arg max_{j\neq \textbf{y}} P_{\Theta}(h(\textbf{x}) = j)$ and let the margin function for a random forest be defined as
	
	\begin{equation}\label{eq:rf-marginFunRf}
	mr(\textbf{x},\textbf{y}) =  P_{\Theta}(h(\textbf{x}) = \textbf{y}) - P_{\Theta}(h(\textbf{x}) = \hat{j}) 
	\\ 
	= \Expect_{\Theta} \left[  I(h(\textbf{x},\Theta ) = y ) - I( h( \textbf{x},\Theta ) = \hat{j} )  \right]
	\end{equation} 
	
	
	%\begin{equation}%\end{equation}
	
	%\end{equation}
	
	Here the margin function is described as the expectation taken over another function which is called the \textbf{raw margin function}\label{eq:rf-rawMarginFun}. Intuitively, the raw margin function takes each sample to be $1$ or $-1$ according to whether the classifier can correctly classify or not the sample's label, given $\Theta$.
	
	With these definitions, it is straight to see that 
	
	%\begin{equation}%\end{equation}
	$$mr( \textbf{x},\textbf{y} )^2 = \Expect_{\Theta, \Theta'} \left[ rmg( \Theta,\textbf{x},\textbf{y} ) \ rmg(\Theta',\textbf{x},\textbf{y} )  \right] $$.
	
	This in turn implies that
	
	\begin{equation}\label{eq:rf-marginFunVar}
	\begin{split}
	var(mr) & =  \Expect_{\Theta, \Theta'} 
	\left[ 
	cov_{\textbf{x},\textbf{y}}
	(rmg(\Theta,\textbf{x},\textbf{y} )rmg(\Theta',\textbf{x},\textbf{y} )) 
	\right] \\
	& =  \Expect_{\Theta, \Theta'}
	\left[ 
	\rho(\Theta, \Theta')\sigma(\Theta)\sigma(\Theta')
	\right] 
	\end{split}
	\end{equation}
	
	where $ \rho(\Theta, \Theta')$ is the correlation between $rmg(\Theta,\textbf{x},\textbf{y})$ and $rmg(\Theta',\textbf{x},\textbf{y})$, and $\sigma(\Theta)$ is the standard deviation of $rmg(\Theta,\textbf{x},\textbf{y})$. In both cases, $\Theta$ and $\Theta'$ are given to be fixed. 
	
	Equation \ref{eq:rf-marginFunVar} in turn implies that 
	
	\begin{equation}\label{eq:rf-varianceBound}
	\begin{split}
	var(mr) & =  \overline{\rho} (\Expect_{\Theta}\left[ \sigma(\Theta)\right] )^2 \\
	& \leq  \overline{\rho} \Expect_{\Theta} \left[ var(\Theta) \right] 
	\end{split}
	\end{equation}
	
	where we have conveniently defined $\overline{\rho}$ as 
	
	\begin{equation}\label{eq:rf-meanCorrelation}
	\frac{\Expect_{\Theta, \Theta'} \left[ \rho(\Theta, \Theta') \sigma(\Theta) \sigma(\Theta')\right]}
	{\Expect_{\Theta, \Theta'} \left[ \sigma(\Theta) \sigma(\Theta')\right]}
	\end{equation}
	
	Note this is the mean value of the correlation.
	
	Let the strength of the set of weak classifiers in the forest be defined as 
	
	$$s =  \Expect_{\textbf{x},\textbf{y}} \left[ mr(\textbf{x},\textbf{y} ) \right] $$.\label{eq:rf-strength}
	
	Assuming that $s \geq 0$ we have that the prediction error is bounded by 
	\begin{equation}\label{eq:rf-predictiveErrorBound1}
	PE^* \leq var(mr)/s^2
	\end{equation}
	by Chebyshev's inequality. On the other hand it can also be noted that
	
	
	\begin{equation}\label{eq:rf-expectedVarBound}
	\begin{split}
	\Expect_{\Theta} \left[ var(\Theta) \right]  & \leq \Expect_{\Theta} \left[ \Expect_{\textbf{x},\textbf{y}}\left[ rmg(\Theta,\textbf{x},\textbf{y})   \right]  \right]^2 -s^2  \\
	& \leq 1-s^2
	\end{split}
	\end{equation}
	
	\begin{proof}
		We can use  \ref{eq:rf-varianceBound}, \ref{eq:rf-predictiveErrorBound1} and \ref{eq:rf-expectedVarBound} to establish the upper bound for the prediction error we are looking for
		\begin{equation}\label{eq:rf-PEBound}
		PE^* \leq \overline{\rho}\frac{(1-s^2)}{s^2}
		\end{equation}
	\end{proof}
	
	This bound on the generalization error shows the importance of the strength of each individual weak classifier in the forest and the correlation interdependence among them. The author \cite{breiman-randomforests} of the algorithm remarks here that this bound may not be strong. He also puts special importance on the ratio between the correlation and the strength $\frac{\overline{\rho}}{s^2}$ where this should be as small as possible to build a strong classifier. 
	\subsection{Binary Class}
	In the context of a binary class problem, where the target variable can only take two values, there are simplifications to the formula \ref{eq:rf-PEBound}. In this case, the margin function takes the form of $2 P_{\Theta}(h(\textbf{x}) = \textbf{y}) -1$ and similarly the raw margin function results in $2 I(h(\textbf{x}, \Theta) = \textbf{y}) -1$. 
	
	The bounds prediction error bounds derived in \ref{eq:rf-predictiveErrorBound1} assume that $s >0$ which in this case results in
	$$ \Expect_{\textbf{x},\textbf{y}} \left[ P_{\Theta}(h(\textbf{x}) = \textbf{y}) \right] > \frac{1}{2} $$
	
	Also, the correlation between $I(h(\textbf{x}, \Theta) = \textbf{y})$ and \  $I(h(\textbf{x}, \Theta') = \textbf{y})$, denoted $\overline{\rho}$ will take the form
	
	$$ \overline{\rho} =  \Expect_{\Theta,\Theta'} \left[ \rho \left( h(\cdot{},\Theta) ,h(\cdot{},\Theta') \right)   \right] $$
	
	
	One benefit of building Random Forest classifiers is that the algorithm easily increases the prediction error of a group of estimators by randomly building each of these in a way that decreases the variance of the overall model whilst trading a small loss in bias. The model is also robust to the introduction of number of noisy features. If the ratio of informative to non-informative features is not extreme, selecting $m$ features at each split at random will mean that in most cases, splits will be made on those informative features. Note that in any given tree, the probability of drawing at least one informative feature in a split is still very high. This is because it follows a hypergeometric distribution $\mathcal{H}(P,j,l)$ with $l$ draws from a total population of $P$ features and only $j$ informative ones.
	
	The depth of growth for each tree is another important tuning parameter. We must chose it correctly by assessing the model's performance across different values for $m$.  A deep tree will tend to overfit the data by partitioning input space to fit the training data. This effect will counter the overall reduction in variance of the forest and thus increase the generalization error of our algorithm.
	
	%Therefore controlling the maximum allowed growth for the base learners will be important to improve the performance of the model.
	
	A special modification in the way Random Forests are built, allows to derive a measure of variable importance. The idea for this, is that at each split we can measure the gain of using a certain variable for the split versus not using it. Given a candidate feature $X_j$ to be analyzed and fore every node in a tree where a split is to be done, we compare the the improvement in the split performance, as measured by some pre-selected criteria, with and without $X_j$. These results are recorded and averaged across all trees and all of the split scenarios to have a score for the feature. The features with the highest scores can be thought as the most informative variables of the model.  
	
	
	
\end{document}

