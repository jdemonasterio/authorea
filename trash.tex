%
% AGRANDA 2016 Camera ready version!
%

\documentclass{article}%{llncs}

\usepackage{graphicx}
\usepackage[space]{grffile}
\usepackage{latexsym}
\usepackage{textcomp}
\usepackage{longtable}
\usepackage{booktabs}
% \usepackage{row}
\usepackage{amsfonts,amsmath,amssymb} %paquetes de matematica
\usepackage{amsthm}
\usepackage{url}
% \usepackage{hyperref}d
% \hypersetup{colorlinks=false,pdfborder={0 0 0}}
% You can conditionalize code for latexml or normal latex using this.
% \newif\iflatexml\latexmlfalse
\usepackage[utf8]{inputenc}
\usepackage[english]{babel}
\usepackage{algorithm2e}
\usepackage{qtree}
\usepackage{amssymb}
\usepackage{hyperref}

% \setcounter{tocdepth}{3}
%\usepackage{graphicx}
% \usepackage{subfigure}
\usepackage{nicefrac}

% for confusion matrix building
\usepackage{array}
\usepackage{multirow}

\newcommand\MyBox[2]{
\fbox{\lower0.75cm
	\vbox to 1.7cm{\vfil
		\hbox to 1.7cm{\hfil\parbox{1.4cm}{#1\\#2}\hfil}
		\vfil}%
}%
}


% mathcal certain letters
\newcommand{\calG}{\mathcal{G}}
\newcommand{\calN}{\mathcal{N}}
\newcommand{\calE}{\mathcal{E}}
\newcommand{\calL}{\mathcal{L}}

% \renewcommand{\labelitemi}{$\bullet$}
\setlength{\tabcolsep}{6pt}

%comandos de operadores de esperanza, varianza etc.
%\newcommand{\Expect}{{\rm I\kern-.3em E}}
\newcommand{\Expect}{{\mathbb{E}}}

\newcommand{\Var}{\mathrm{Var}}
\newcommand{\Cov}{\mathrm{Cov}} 

%comandos de teoremas, corolarios, lemmas, pruebas y definiciones
\newtheorem{theorem}{Theorem}[section]
\newtheorem{corollary}{Corollary}[theorem]
\newtheorem{lemma}[theorem]{Lemma}
\newtheorem{definition}{Definition}[subsection]
\theoremstyle{definition}


\begin{document}

% \mainmatter  % start of an individual contribution
% first the title is needed
\title{Trash Tex file to test different formulas and stuff}




\author{
	Juan de Monasterio
	\and Carlos Sarraute
}

%
% NB: a more complex sample for affiliations and the mapping to the
% corresponding authors can be found in the file "llncs.dem"
% (search for the string "\mainmatter" where a contribution starts).
% "llncs.dem" accompanies the document class "llncs.cls".
%

%\toctitle{Lecture Notes in Computer Science}
%\tocauthor{Uncovering the Diffusion of an Infectious Disease with Mobile Phone Data
%%Unveiling Chagas with Big Data}

\maketitle
\begin{abstract}
	Your text should have one idea per paragraph and use transitions between sentendes. The following linnk provides a whole set of transition words to use.
	\url{http://grammar.ccc.commnet.edu/Grammar/transitions.htm}
	
	
%%%% HOW TO USE SPLITS IN EQUATIONS (same level for equal signs)
%\begin{split}
%	var(mr) & =  \Expect_{\Theta, \Theta'} 
%	\left[ 
%	cov_{\textbf{x},\textbf{y}}
%	(rmg(\Theta,\textbf{x},\textbf{y} )rmg(\Theta',\textbf{x},\textbf{y} )) 
%	\right]
%\end{split}

%%%% Equations with no \$ chars
%\[
%l(\theta) = \sum_{i=1}^N \big(y_i log(P(y_i|x_i,\theta)) + (1-y_i)log(1 - P(y_i|x_i,\theta)) \big)
%\]

%%%% Equations with no \$ chars
%\begin{equation}
%l(\theta) = \sum_{i=1}^N \big(y_i log(P(y_i|x_i,\theta)) + (1-y_i)log(1 - P(y_i|x_i,\theta)) \big)
%\end{equation}

%\begin{definition}{Vapnik–Chervonenkis (VC) Dimension}
%\end{definition}
\end{abstract} 




\section{Classifier : Decision Trees}

%rview of this supervised learner used in regression and classification problems.

% The models builds a
Decision trees are such as in the graph theoretical sense where each node acts as a rule to which any input sample can comply or not. The rules are built as linear (or similar) partitions of input space and they are built on the value of a given feature 
%properties of the values for each feature. 

At any6 given node the model takes a decision to include or exclude a sample $s$ from that partition by checking if $X_i(s) \in U$ where $X_i$ might be any given feature of the data and $U$ is a subset of said feature's space.

For numerical features, the input space $X$ willi be partitioned into  $L$ and $R$ where $L$ of the form $(-\infty,c]$.  The value $c \in \mathbb{R}$ is any given number predefined by the rule itself.   In the same way, for categorical features $L$ will be a subset of the possible values of that feature. 

Altogether, a tree defines a partition of feature space in disjoint regions $A_1,...,A_K$ such that the tree's predicted output $\hat{y}$ for a sample $x$ is $c_k$ if the sample belongs to $A_k$. Here $c_k$ is one of the possible values taken by the target variable $y$ in the training set. And by the way trees were built, each $A_k$ is a hyper-rectangles in feature space.

In brief, the learner can be characterized by 
\[
h(X) = \sum_{k=1}^K c_k I(X \in A_k)
\]\label{equation-decisionTreeModel}

where $c_k$ is the value that our model estimates for samples in the $A_k$ region. Both of these will have to be learnt by the model in the optimization procedure. %by minimizing its loss funciton.

By the points given above, the algorithm needs to determine what is the best way to split a set of samples that flow through a decision node. The \textit{goodness of split} will be measured by the loss metric of the algorithm. Here, the criteria used to decide on node splits are called \textit{node impurity measures}.
%them according, in order to optimize a loss metric. 

%at each node 
Most variations for this machine learning model build rules in a greedy fashion, where node impurity measures are locally optimized at each node to decide on which is the best splitting value. The reason for doing this is because not doing so will result in an algorithm whose computational complexity is infeasible. Given that the construction of optimal binary decision trees is NP-Complete \cite{decisionTreesNP}, the optimal rules for the tree are then fit sequentially.

In conclusion, at any splitting node we have to find the \textit{best} feature $X^p$ and value split $t$ for which to partition the data in
$A_L = \{x \in \mathcal{T} \  / \ x^p \leq t \} $ and $A_R = \{x \in \mathcal{T}\  / \ x^p> t \} $. Let $N_l$ and $N_r$ be $|A_L|$ and $|A_R|$ respectively. Then, to quantify the \textit{best} feature for this split the algorithm minimizes:

%\frac{1}{N_{left}}
%\frac{1}{N_{right}}

\[
%\begin{split}
min_{p,t} \big[ min_{c_L }   \frac{1}{N_l}\sum_{x \in A_L(p,t) } L(y,c_L)        \ +   min_{c_R}   \frac{1}{N_r}\sum_{x \in A_R(p,t) }  L(y,c_R) \big]
%\end{split}
\]\label{equation-decisionTreeGreedyOptimization}

where $y$ is the target associated to our sample $x$ and $L(\cdot)$ is the loss function we have used to measure the quality of our split. Note that this can be done efficiently for a wide range of loss functions since the minimization can be done for each feature independently. 

A tree is then grown in an iterative way from the top down \footnote{In this context the \textit{top} of a tree refers to the root of the tree.}, estimating the appropriate parameters at each rule split. All of the training set's samples would start at the top (the root node) and then travel down through the trees branches, in accordance to their fulfillment or not of each node's rule. A branch of the tree would stop growing once all samples at a node belong to the same target class.

Finally, we would have that the tree's leafs are the partition subsets over the input data and once a learner is fit, predicting targets for new samples is straightforward: the prediction of their target class will be the value given after traveling the sample down to its corresponding leaf node.  

To illustrate this method, an instance is show in figure \ref{rf-treeFigure}. This classification tree example is built for the two class problem of gender prediction using data from CDRs:
%[.{\textit{Woman}}]
\smallskip
\begin{figure}[h]\label{rf-treeFigure}
	\Tree[.{ $Calling\_Volume \leq 23$ } [.{$Province \in \{ San Luis, Chubut \} $} [.{$Time\_Weekend \geq 16$} [.{\textit{M}} ] [.{\textit{F}} ]  ]
	[.{$Calls\_Weekdays \leq 48$} 
	[.{ $Time\_Weekday \geq 17$} [.{\textit{M}} ] [.{\textit{F}} ]] [.{\textit{F}} ] ]  ]
	[.{$Calls\_Mondays \geq 2$} [.{$Province \in \{ Chubut, Cordoba \} $}  [.{\textit{M}} ] [.{\textit{F}} ] ]
	[.{\textit{M}}  ]]]
	
\end{figure}

\smallskip

%[.{\textit{M}} ] [.{\textit{F}} ]

The most used metrics to build each rule are the \textit{Gini impurity measure} and the \textit{entropy} or \textit{information gain} criterion. The former optimizes for misclassification error in the two resulting sets. Where it values the accuracy of the model if all samples were to be tagged with the most common target-label in that split. The latter optimizes for information entropy, which is analogous to minimizing \textit{Kullback-Liebler divergence} of the resulting sets with respect to the original set previous to the split. 

In decision trees, the process of iteratively partitioning the samples in splits continues until a predefined tuning parameter limit will stop the optimization or when there is only a single target class for all samples at the node. 

The hyper-parameters for this model include the length of the tree,  the splitting rule threshold and the node impurity measures. From the descriptions previously given, we can list these directly:

\begin{itemize}
	\item Max depth of the tree, or the number of allowed level of splits.
	\item The criteria or measure used to select the best split feature at each node.
	\item The leaf size or the total number of minimum samples allowed per leaf. Note that this is a limit imposed on  branch depth. 
	\item Number of features selected to decide on the best split feature at each node.
\end{itemize}


Intuitively, it is natural to find that trees of longer depth will overfit the data since more complex interactions among variables will be captured by refining the partition on input space. A trivial case would be to allow a tree to grow fully in depth and later assign to each sample in the training set its own region. This would yield a model with absolutely no bias but which will have a very high prediction error since new samples won't be labeled accordingly. 

On the other hand having a tree which is too shallow will result in most cases in a biased algorithm.  This is because it results in an overly simple model incapable of correctly assigning labels.  We must then consider that the depth of a tree is a measure of the model's complexity and as such one of the most important hyperparameters of our model. 

Another drawback of the model is the high variance instability. Authors point out that two very similar datasets can grow two very different resulting trees. This is due to the hierarchical nature of the splits, where errors randomly made in the first splits will be carried onward towards the leafs sinces samples continue down on a single branch.

The most common methods to control the tree's depth grow a very large tree $T_0$ that will continue until it reaches a depth limit threshold that is very unrestrictive. Then the tree will be pruned by removing branches and nodes to remove model complexity with only  a slight loss of accuracy. 

To expand on this idea, let $T \subset T_0$ be a subtree of the first tree, where $T$ is obtained by pruning $T_0$.  Here the partition regions $R_j$ will be associated to $T$'s terminal nodes or leafs, indexed by $j$, which $j \in \{1,...,|T|  \}$. 

Given a loss function, and a problem with $K$ possible target classes, we can define the following values:
\begin{equation}
\begin{split}
N_j & =  \mid\{x \in R_j \}\mid\\
\hat{p}_{jk} & = \frac{1}{N_j} \sum_{x \in R_j}  I(y=k)\\
c_j & =  argmax_{k} \  \hat{p}_{jk} \\
\end{split}
\end{equation}\label{decisionTreePruneParameters}

As we have mentioned before, at each split we use the node impurity measure to quantify this action. Here, we will denote $Q_j(T)$ to be the impurity measure for region $j$. In addition, we list the three most used impurity measures of classification trees:

\begin{itemize}
	\item Misclassification error: $ \displaystyle \frac{1}{N_j} \sum_{x \in R_j}  I(y\neq c_j)  = 1 - c_j $
	\item Gini index: $ \displaystyle \sum_{k\neq k'} \hat{p}_{jk} \hat{p}_{jk'}   = \sum_{k=1}^{K} \hat{p}_{jk} (1 - \hat{p}_{jk})  $
	\item Cross-entropy: $ \displaystyle \sum_{k=1}^{K} -log(\hat{p}_{jk})\hat{p}_{jk} $
\end{itemize}


As an extension to the Gini index, it is also common to use a reweighted version of the sum. A loss matrix is multiplied to the factors in each summand. The matrix is reassigning weights to different cases of misclassification. Indeed, these weights become practical when we have that a sample from class $k$ incorrectly assigned to class $k'$ might be less important than other misclassifications. 

Let $L \in \mathbb R_{\ge 0}^{K \times K}$ where $(L)_{(k,k')}$ is the cost of misclassifying a class $k$ sample into $k'$. Naturally we have that $L$ is a null diagonal matrix. Then the Gini index's summands will take the reweighted form  $L_{kk'} \hat{p}_{jk} \hat{p}_{jk'}$.

For the binary (two class) case $Q_j(T)$ can be expressed in simpler terms. If we consider $p$ to be the probability of success, then we have 

\begin{itemize}
	\item Misclassification binary error: $1 - max(p, 1-p)$
	\item Gini binary index: $ 2p(1-p) $
	\item Binary cross-entropy: $ -log(p)p - log(1- p)(1-p) $
\end{itemize}\label{decisionTreeCostFunctions}

In summary, we will have an impurity measure for each region $R_j$ and the altorithm will then aggregate all of them into a single measure called the \textit{cost complexity criterion}. The idea is that this loss function, as a function of a tree, will control the whole optimization procedure through the tree's parameters. The criterion will be managed to control the final model's bias and variance. We define it in the following way

\begin{equation}
C_\alpha(T)  = \sum_{j=1}^{|T|} N_j Q_j(T)  + \alpha|T| 
\end{equation}\label{decisionTreeCostComplexity}


Here  $\alpha \in \mathbb{R}_{\geq 0}$ is a tuning parameter that values the trade-off between the tree complexity, as given by its depth, and the accuracy of the model as given by the measure we proposed in \ref{decisionTreeCostFunctions}. The idea is that given an $\alpha$ we find the subtree $T_{alpha} \subset T_0$ that minimizes \ref{decisionTreeCostComplexity}. 

There are various methods we could use to find the optimal subtree. As an example, here we give an example of the \textit{weakest link pruning} algorithm which goes as follows:
%must \textit{prune} the initially grown tree 

Let  $B(T)  = \sum_{j} N_j Q_j(T) $ be our pure loss function, without any complexity cost added. \cite{breiman-cart84} shows that we can find $T_{alpha}$ included in a sequence of trees built for this. This sequence is constructed by iteratively pruning the node $j$ that, when removed from the tree, creates the smallest increase in $B(T)$. 

In this way, we'll have a sequence of trees $T_0,T_1,...,T_l$ and a sequence of nodes $j_0, j_1,...,j_l$ respectively the ones minimizing the increase in $B(T_0),B(T_1),...,B(T_l)$ at each step. The algorithm will stop when have reached the root node and we will find our tree $T_{alpha}$ by comparing all of the $C_\alpha(T)$ for all of the trees built in the sequence. In practice, it is common to have this procedure done within a $K$-fold cross validation routine to reach to an estimated $\hat{\alpha}$.

For a more complete explanation of a decision tree for classification or regression problems, please refer to \cite{breiman-cart84}.

\textit{}


\end{document}

