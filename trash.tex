%
% AGRANDA 2016 Camera ready version!
%

\documentclass{article}%{llncs}

\usepackage{graphicx}
\usepackage[space]{grffile}
\usepackage{latexsym}
\usepackage{textcomp}
\usepackage{longtable}
\usepackage{booktabs}
% \usepackage{row}
\usepackage{amsfonts,amsmath,amssymb} %paquetes de matematica
\usepackage{amsthm}
\usepackage{url}
% \usepackage{hyperref}d
% \hypersetup{colorlinks=false,pdfborder={0 0 0}}
% You can conditionalize code for latexml or normal latex using this.
% \newif\iflatexml\latexmlfalse
\usepackage[utf8]{inputenc}
\usepackage[english]{babel}
\usepackage{algorithm2e}
\usepackage{qtree}
\usepackage{amssymb}
\usepackage{hyperref}

% \setcounter{tocdepth}{3}
%\usepackage{graphicx}
% \usepackage{subfigure}
\usepackage{nicefrac}

% for confusion matrix building
\usepackage{array}
\usepackage{multirow}

\newcommand\MyBox[2]{
\fbox{\lower0.75cm
	\vbox to 1.7cm{\vfil
		\hbox to 1.7cm{\hfil\parbox{1.4cm}{#1\\#2}\hfil}
		\vfil}%
}%
}


% mathcal certain letters
\newcommand{\calG}{\mathcal{G}}
\newcommand{\calN}{\mathcal{N}}
\newcommand{\calE}{\mathcal{E}}
\newcommand{\calL}{\mathcal{L}}

% \renewcommand{\labelitemi}{$\bullet$}
\setlength{\tabcolsep}{6pt}

%comandos de operadores de esperanza, varianza etc.
%\newcommand{\Expect}{{\rm I\kern-.3em E}}
\newcommand{\Expect}{{\mathbb{E}}}

\newcommand{\Var}{\mathrm{Var}}
\newcommand{\Cov}{\mathrm{Cov}} 

%comandos de teoremas, corolarios, lemmas, pruebas y definiciones
\newtheorem{theorem}{Theorem}[section]
\newtheorem{corollary}{Corollary}[theorem]
\newtheorem{lemma}[theorem]{Lemma}
\newtheorem{definition}{Definition}[subsection]
\theoremstyle{definition}

\begin{document}

% \mainmatter  % start of an individual contribution

% first the title is needed
\title{Trash Tex file to test different formulas and stuff}




\author{
	Juan de Monasterio
	\and Carlos Sarraute
}

%
% NB: a more complex sample for affiliations and the mapping to the
% corresponding authors can be found in the file "llncs.dem"
% (search for the string "\mainmatter" where a contribution starts).
% "llncs.dem" accompanies the document class "llncs.cls".
%

%\toctitle{Lecture Notes in Computer Science}
%\tocauthor{Uncovering the Diffusion of an Infectious Disease with Mobile Phone Data
%%Unveiling Chagas with Big Data}

\maketitle
\begin{abstract}
	
	Texstudio is better at helping to write fast tex documents using spell-checkers, pre-loaded commands and \textit{stuff}. All of the following text is just copy/paste trash being tested before being added to a real doc.
	
%%%% HOW TO USE SPLITS IN EQUATIONS (same level for equal signs)
%\begin{split}
%	var(mr) & =  \Expect_{\Theta, \Theta'} 
%	\left[ 
%	cov_{\textbf{x},\textbf{y}}
%	(rmg(\Theta,\textbf{x},\textbf{y} )rmg(\Theta',\textbf{x},\textbf{y} )) 
%	\right]
%\end{split}

%%%% Equations with no \$ chars
%\[
%l(\theta) = \sum_{i=1}^N \big(y_i log(P(y_i|x_i,\theta)) + (1-y_i)log(1 - P(y_i|x_i,\theta)) \big)
%\]

%%%% Equations with no \$ chars
%\begin{equation}
%l(\theta) = \sum_{i=1}^N \big(y_i log(P(y_i|x_i,\theta)) + (1-y_i)log(1 - P(y_i|x_i,\theta)) \big)
%\end{equation}

%\begin{definition}{Vapnik–Chervonenkis (VC) Dimension}
%\end{definition}
	
	
\end{abstract} 

\section{Logistic Regression for Classification}

This model is named as a \textit{regression} model but is actually used for \textit{classification} and this convention has been used in this way for historical reasons.



Let $\{C_1,..,C_k\}$ denote the set of possible target classes. We are  interested in maximizing the probability of belonging to a certain class, given the input data:
$$p(C_k| X) = \frac{p(x|C_k)p(C_k)}{p(x)} $$

\begin{definition}{Logistic Regression}
	Given a choice of parameter $\theta$, 	
\end{definition}

Logistic regression models the posterior probabilities of the classes with a transformation on a linear function of inputs. It also conditions that the posterior probabilities must sum to one. Here the probabilities describing the possible values for the target variable is modeled using a logistic function.

To satisfy this the log-odss ratios of all the classes with respect to a fixed class must all be linear in the inputs. This means that

$$ log\big( \frac{P(C_i|x)}{P(C_j|x)}\big) = \theta_{i0}  + \theta_i^\intercal x  $$ for $i,j \in \{1,2,...,K\}, i\neq j$ \label{logit-logOddss}

With these same indices we have that

$$ P(C_i|x) = \frac{ exp(\theta_{i0}  + \theta_i^\intercal x)}{1 + exp(\theta_{j0}  + \theta_j^\intercal x)}   $$ 

Then the model is specified in terms of the log-odds for each class and is parameterized by $\theta$.

This model sets the target as a Bernoulli random variable when conditioned on the input variables. Formally,

\begin{equation}
\begin{split}
Y_i \mid X_i \  \sim & \operatorname{Bernoulli}(p_i) \\
\mathbb{E}[Y_i \mid X_i ] = & p_i  
\end{split}
\end{equation}


The probability function of the target given the features $\Pr(Y_i=y\mid X_i)$ is given by 
$$\Pr(Y_i=y \mid X_i = x_i) = p_i^{y} (1-p_i)^{(1-y)}$$\label{logit-probabilityDensity}
where this depends on the class dependence of $y$.

Here the logit function is used to map log odds into conditional probabilties and the model outputs the predicted probabilities of the target variables belonging to the target class.

In this way, we are specifying a model where the target is a linear function of the inputs, corrected by an error term:
$$Y_i = I(\theta_0 + \theta \cdot X_i + \epsilon) \ \forall i$$. \label{logit-indicatorFunction}

where $\epsilon$ is the error of the approximation and is distributed with the standard logistic distribution. %and parameter $p_i$. 

If we take the conditional probability on \ref{logit-indicatorFunction}, given the features we will have that
$$logit(p_i)= ln(\frac{p_i}{1-p_i}) = logit\big( \Expect[Y_i| X_i] \big) = 	\theta_0 + \theta \cdot X_i$$

From this equation we have once again that


$$p_i = \sigma(\theta \cdot X_i) = \frac{exp(\theta \cdot X_i) }{1 + exp(\theta \cdot X_i)}$$

where we have used an abuse of notation to absorb $\theta_0$ into $\theta$. Finally, if use this and plug it into the initial conditional probability of $Y_i$ in \ref{logit-probabilityDensity} we will have

$$  \Pr(Y_i=y \mid X_i = x_i) =  p_i^{y} (1-p_i)^{(1-y)} = \frac{exp(y . \theta \cdot X_i) }{1 + exp(\theta \cdot X_i)}$$


Maximum likelihood is the most common method used to fit the model. %with multinomial distributions modeling the features. 
Given a parameter $\theta$, the probability of having a target vector $y$ is 
\[
P(Y =y \mid \theta )  = \prod_{i=1}^N P(y_1 \in C_1 \mid x_i, \theta)^y_i(1 - P(y_1 \in C_1 \mid x_i, \theta) )^{(1-y_i)}
\]

If we take into account that $P(y=1 \mid x,\theta) = 1 - P(y=0 \mid x,\theta)$ , then the estimation of $\theta$ for $N$ samples gives the following loss function

\[
l(\theta) = \sum_{i=1}^N \big(y_i log(P(y_i \mid x_i,\theta)) + (1-y_i)log(1 - P(y_i \mid x_i,\theta) ) \big)
\]

Note the advantage that the loss function is concave in the parameters. Another advantage of this model is that closed forms can be given for the gradient and the Hessian of the loss function. The negative gradient can be given in the following way: %can thus be analytically expressed

\[
- \nabla  l(\theta) = \sum_{i=1}^N (y_i - P(y_i \mid x_i,\theta))\cdot x_i = \textbf{X}^{\intercal}(\textbf{y}-\textbf{p})
\]

whilst the second order derivatives take the following form:

\[
\frac{\partial^2 l(\theta)}{\partial \theta \partial \theta^\intercal} = \sum_{i=1}^N x_i \cdot x_i^\intercal P(y_i \mid x_i,\theta)(1 -P(y_i \mid x_i,\theta))
\]

\subsection{Regularization}

Regularization of models is the process in which restrictions are imposed on the models for problems which are ill-posed either because the model is biased or, more common, overfit.  

\begin{equation} \label{logit-optimization}
\min_{\theta} \gamma\| \theta\|_{j}  + \sum_i=1^N log(e^{-y_i (x_i \cdot \theta + c )} +1) 
\end{equation}



\subsection{Hyper-parameters}

Note that in this model there are two specific tuning parameters which must be predefined in the optimization procedure. Notably $\gamma$ which measure the weight on the regularization term and $j$ which determines the type of norm to measure the argument of the minimization.




\textit{}

\textit{}

\textit{}

\textit{}

\textit{} 






\end{document}

