%
%

\documentclass{article}%{llncs}

\usepackage{graphicx}
\usepackage[space]{grffile}
\usepackage{latexsym}
\usepackage{textcomp}
\usepackage{longtable}
% \usepackage{row}
\usepackage{amsfonts,amsmath,amssymb} %paquetes de matematica
\usepackage{amsthm}
\usepackage{url}
% \usepackage{hyperref}d
% \hypersetup{colorlinks=false,pdfborder={0 0 0}}
% You can conditionalize code for latexml or normal latex using this.
% \newif\iflatexml\latexmlfalse
\usepackage{lmodern}                      % Use Latin Modern fonts
\usepackage[T1]{fontenc}        % Better output when a diacritic/accent is used
\usepackage[utf8]{inputenc}
\usepackage[
backend=biber,
sortlocale=us_EN,
natbib=true,
url=false, 
doi=true,
eprint=false
]{biblatex}

%\usepackage[english]{babel}
\usepackage{algorithm2e}
\usepackage{qtree}
\usepackage{amssymb}
\usepackage{hyperref}

% \setcounter{tocdepth}{3}
%\usepackage{graphicx}
% \usepackage{subfigure}


%----------------+
% Table packages |
%----------------+

\usepackage{array}          % Flexible column formatting
% \usepackage{spreadtab}  % Spreadsheet features
\usepackage{multirow}       % Allows table cells that span more than one row
\usepackage{booktabs}       % Enhance quality of tables
\setlength{\heavyrulewidth}{1pt}

% for confusion matrix building
\usepackage{array}
% to sidewaysfigure rotate vertically or horizontally
\usepackage{rotating}
% rotates tables
\usepackage{pdflscape}
% Table colors
\usepackage[table,x11names]{xcolor}


\newcommand\MyBox[2]{
\fbox{\lower0.75cm
	\vbox to 1.7cm{\vfil
		\hbox to 1.7cm{\hfil\parbox{1.4cm}{#1\\#2}\hfil}
		\vfil}%
}%
}

% to build master table
\newcommand{\ra}[1]{\renewcommand{\arraystretch}{#1}}

% mathcal certain letters
\newcommand{\calG}{\mathcal{G}}
\newcommand{\calN}{\mathcal{N}}
\newcommand{\calE}{\mathcal{E}}
\newcommand{\calL}{\mathcal{L}}

% \renewcommand{\labelitemi}{$\bullet$}
\setlength{\tabcolsep}{6pt}

%comandos de operadores de esperanza, varianza etc.
%\newcommand{\Expect}{{\rm I\kern-.3em E}}
\newcommand{\Expect}{{\mathbb{E}}}

\newcommand{\Var}{\mathrm{Var}}
\newcommand{\Cov}{\mathrm{Cov}} 

%comandos de teoremas, corolarios, lemmas, pruebas y definiciones
\newtheorem{theorem}{Theorem}[section]
\newtheorem{corollary}{Corollary}[theorem]
\newtheorem{lemma}[theorem]{Lemma}
\newtheorem{definition}{Definition}[subsection]
\theoremstyle{definition}


\begin{document}

% \mainmatter  % start of an individual contribution
% first the title is needed
\title{Trash Tex file to test different formulas and stuff}




\author{
	Juan de Monasterio
	\and Carlos Sarraute
}

%
% NB: a more complex sample for affiliations and the mapping to the
% corresponding authors can be found in the file "llncs.dem"
% (search for the string "\mainmatter" where a contribution starts).
% "llncs.dem" accompanies the document class "llncs.cls".
%

%\toctitle{Lecture Notes in Computer Science}
%\tocauthor{Uncovering the Diffusion of an Infectious Disease with Mobile Phone Data
%%Unveiling Chagas with Big Data}

\maketitle
\begin{abstract}
	Your text should have one idea per paragraph and use transitions between sentendes. The following linnk provides a whole set of transition words to use.
	\url{http://grammar.ccc.commnet.edu/Grammar/transitions.htm}
	
	
%%%% HOW TO USE SPLITS IN EQUATIONS (same level for equal signs)
%\begin{split}
%	var(mr) & =  \Expect_{\Theta, \Theta'} 
%	\left[ 
%	cov_{\textbf{x},\textbf{y}}
%	(rmg(\Theta,\textbf{x},\textbf{y} )rmg(\Theta',\textbf{x},\textbf{y} )) 
%	\right]
%\end{split}

%%%% Equations with no \$ chars
%\[
%l(\theta) = \sum_{i=1}^N \big(y_i log(P(y_i|x_i,\theta)) + (1-y_i)log(1 - P(y_i|x_i,\theta)) \big)
%\]

%%%% Equations with no \$ chars
%\begin{equation}
%l(\theta) = \sum_{i=1}^N \big(y_i log(P(y_i|x_i,\theta)) + (1-y_i)log(1 - P(y_i|x_i,\theta)) \big)
%\end{equation}

\end{abstract} 

\begin{tabular}{|*{16}{c|}}  % repeats {c|} 18 times
	\hline
	\multicolumn{4}{|c}{Problem1} & \multicolumn{4}{|c|}{Problem 2} \\ \hline 
%	& \multicolumn{4}{|c|}{Problem 3} & \multicolumn{4}{|c|}{Problem 4} \\ \hline
	\multicolumn{1}{|c}{$Accuracy$} & \multicolumn{1}{|c|}{$AUC$} & \multicolumn{1}{|c|}{$F1$} & \multicolumn{1}{c|}{$Runtime$} & 
	\multicolumn{1}{|c}{$Accuracy$} & \multicolumn{1}{|c|}{$AUC$} & \multicolumn{1}{|c|}{$F1$} & \multicolumn{1}{c|}{$Runtime$}
	\\ \hline 
	1 & 2 & 3 & 4 & 5 & 6 & 7 & 8 \\ \hline
	
%	& SVD & CJ & HT & SVD & CJ & HT & SVD &CJ \\ \hline
%	& & & & & & & & & & & & & & & & &  \\ \hline
\end{tabular}


\bigskip


%\begin{tabular}{|*{18}{c|}}  % repeats {c|} 18 times
%	\hline
%	\multicolumn{9}{|c}{k-means clustering} & \multicolumn{9}{|c|}{Fuzzy c-means clustering} \\ \hline
%	\multicolumn{3}{|c}{50 clusters} & \multicolumn{3}{|c}{60 clusters} & \multicolumn{3}{|c}{70 clusters} & 
%	\multicolumn{3}{|c}{50 clusters} & \multicolumn{3}{|c}{60 clusters} & \multicolumn{3}{|c|}{70 clusters} \\ \hline 
%	CJ & HT & SVD &CJ & HT & SVD &CJ & HT & SVD &CJ & HT & SVD &CJ & HT & SVD &CJ & HT & SVD \\ \hline
%	& & & & & & & & & & & & & & & & &  \\ \hline
%\end{tabular}
%


\bigskip

% rotates tables
%\usepackage{pdflscape}
%\usepackage{booktabs}
%\newcommand{\ra}[1]{\renewcommand{\arraystretch}{#1}}

\begin{landscape}% Landscape page
		\begin{table*}
			\centering
			\ra{1.3}
			\begin{tabular}{@{}rrrrcrrrrcrrcrr@{}} \toprule
				&  \multicolumn{4}{c}{Problem 1} &  \multicolumn{4}{c}{Problem 2} & \multicolumn{4}{c}{Problem 3}  & \multicolumn{4}{c}{Problem 4}\\
				 \cmidrule{2-5} \cmidrule{6-9} \cmidrule{10-13} \cmidrule{14-17}
				& $Accuracy$ & $AUC$ & $F1$ & $Runtime (m)$ && $Accuracy$ & $AUC$ & $F1$ & $Runtime (m)$ && $Accuracy$ & $AUC$ & $F1$ & $Runtime (m)$ && $Accuracy$ & $AUC$ & $F1$ & $Runtime (m)$ \\ 
				\midrule
				$Naive Bayes$ & 0.84 & 0.82 & 0.75 & 2 && 0.64 & 0.61 & 0.31 & 2 && 0.65 & 0.63 & 0.45 & 1 && 0.85 & 0.76 & 0.62 & 1 \\
				$Logistic \ Regression$ & 0.893 & 0.857 & 0.9 & 96 && 0.714& 0.726 & 0.058  & 119 && 0.705& 0.754 & 0.107 & 115 && 0.883 & 0.85 & 0.181 & 66 \\
				$Random \ Forest$ & 0.878 & 0.903 & 0.77 & 33  && 0.79 & 0.776 & 0.088 & 45 && 0.792 & 0.845 & 0.156 & 21 && 0.898 & 0.853 & 0.203 &  19\\
				$Gradient \ Tree \ Boosting$  & 0.974 & 0.978 & 0.952 & 41 && 0.838 & 0.819 &  0.101 & 54  && 0.811 & 0.855 & 0.169 & 33 && 0.885 & 0.873  & 0.194 & 47 \\
				\bottomrule
			\end{tabular}
			\caption{Caption}
		\end{table*}	
\end{landscape}
	
\bigskip



%\usepackage{booktabs}
%\newcommand{\ra}[1]{\renewcommand{\arraystretch}{#1}}
\begin{table*}\centering
	\ra{1.3}
	\begin{tabular}{@{}rrrrcrrrcrrr@{}} \toprule
		&  \multicolumn{3}{c}{$w = 8$} &  \phantom{abc} & \multicolumn{3}{c}{$w = 16$} & \phantom{abc} & \multicolumn{3}{c}{$w = 32$}\\
		\cmidrule{2-4} \cmidrule{6-8} \cmidrule{10-12} 
		& $t=0$ & $t=1$ & $t=2$ && $t=0$ & $t=1$ & $t=2$ && $t=0$ & $t=1$ & $t=2$\\ \midrule
		$dir=1$\\
		$c$ & 0.0790 & 0.1692 & 0.2945 && 0.3670 & 0.7187 & 3.1815 && -1.0032 & -1.7104 & -21.7969\\
		$c$ & -0.8651& 50.0476& 5.9384 && -9.0714& 297.0923& 46.2143&& 4.3590& 34.5809& 76.9167\\
		$c$ & 124.2756& -50.9612&-14.2721&& 128.2265& -630.5455& -381.0930&& -121.0518& -137.1210& -220.2500\\
		$dir=0$\\
		$c$ & 0.0357& 1.2473& 0.2119&& 0.3593& -0.2755& 2.1764&& -1.2998& -3.8202& -1.2784\\
		$c$ & -17.9048& -37.1111& 8.8591&& -30.7381& -9.5952& -3.0000&& -11.1631& -5.7108& -15.6728\\
		$c$ & 105.5518& 232.1160& -94.7351&& 100.2497& 141.2778& -259.7326&& 52.5745& 10.1098& -140.2130\\
		\bottomrule
	\end{tabular}
	\caption{Master cross table on all best-fit learners for all four migration problems}
\end{table*}

\end{document}
