o
As we know, the learner will approximate the target with $y \approx \hat{y} = h\left(\sum_{j}\theta_jx_j\right)$ and the first idea leads to find the hyperplane that best separates the two classes by estimating parameter $\hat{\theta}$. Given that we have now a problem of $p$ degrees of freedom, what is left is to find the parameter by optimizing a certain criteria. This choice will certainly depend on the way we decide that one parameter is better than another. For example the choice of $0.5$ as a threshold in \ref{formula:logitThreshold} is ad-hoc and certainly one which we could try to fit in the optimization process. In the context of machine learning the criteria used to choose the best parameters is called a \textit{metric}.

A naive approach to find the parameters in our problem would be to minimize the residual sum of squares. Typical of other scenarios such as linear regression, one would like to minimize the following sum (RSS)  

\begin{equation} \label{eq:rss}
RSS(\theta_0,..,\theta_p)  = \sum_{i=1}^n [y^i - \hat{y}^i]^2  \\
=  \sum_{i=1}^n [y^i - h( \theta \cdot x^i)]^2
\end{equation}

The equation reflects our goal to correctly classify aling training samples with their targets. But in truth we are interested in having a generalized model, one which can make a \textit{good} prediction on any sample, even new ones. The previous equation is a bad attempt to generalize the classification model from our data to any given sample of the \textit{true} distribution for \mathrm{T}. 

Given a choice of parameter \theta, the \textbf{prediction error} for the resulting classifier $f_\theta$ is

\[
    Err_{true}(\theta)  = \Expect_{ \mathrm{T}}[(\textbf{y} - h(\textbf{x} \cdot \theta) )^2] \\
    = \int (y - h(x \cdot \theta) )^2 p(x,y)dxdy
\]

Integration here is over the joint distribution of inputs and outputs. In practical cases, we have incomplete information on $p(x,y)$ and a finite sample.
In practical cases though, we will have only a finite sample. We must assume then that calculating this integral is not feasible for any $\theta$ and must rely on estimation procedures.

For example, we could try and sample $M$ iid points from $p(x,y)$ to approximate the integral by a Monte Carlo scheme such as 

\begin{equation} \label{eq:mcarlo-approx}
    Err_{true}(\theta)  \approx \frac{1}{M} \sum_i^M ( y - h(x \cdot \theta) )^2
\end{equation}

However this would be unfeasible as well, since the sampling process should be done for each specific $\theta$. Notice however the close resemblance of this equation \ref{eq:mcarlo-approx} to the form in \ref{eq:rss}. 

In most applications classification problems don't use the residual sum of squares to estimate the model\'s parameters. Instead, they rely on other \textit{loss} functions that we will introduce later.

Let $f: X \rightarrow Y$ a function mapping feature space to the target space and assume that $y  =  f(x)  +  \epsilon $ is a good relationship for our data, where $\epsilon \sim \calN(0,1) $ then the equation in \ref{eq:rss} can be read as the parameter error given the training set.

Our interest is know on minimizing the error
\[
Err_{train}(\theta) \approx \frac{1}{n} \Expect_{ \mathrm{T}}[(y - h(x \cdot \theta) )^2]
\]


%, and can be decomposed into two types of errors