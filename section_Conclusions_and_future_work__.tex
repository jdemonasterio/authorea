\section{Conclusions}
The results presented in this work show that it is possible to explore CDRs as a mean to tag human mobility. Combining social and geolocated information, the data at hand has been given a new innovative use different to the end for which the data was created (billing).

Analysis of this type of data is of low cost, based on open-source tools and run on medium computing power (TENDRIA QUE HACER una referencia aca a una instancia de amazon con 8 cores y 64gb de ram para correr este analisis?). The potential value the results could add to health research is exposed
 POTENCIALMENTE MUCHO VALOR

The outcome of this work also stands as a proof of concept which can be extended to other countries or to diseases with similar characteristics.


\section{Future Work}
Apply best fit model to Argentinian dataset. Provide that info to a Chagas risk model. Assuming that a high influx of individuals from epidemic regions is correlated with a higher risk in that area the algorithm could highlight these points.

TAG rural antennas.

FEATURE EXPLORATION. Hypothesis testing. 

SEASONAL MIGRATIONS AND OTHER regions

Search for EPIDEMIOLOGICAL DATA at a finer grain. (such as specific infection cases, ) Differencing the epidemic region on specific infection cases rather than on
Maybe even compare model with field results from NGOs partnered in this project.

\section{Acknowledgements}
Mundo Sano.
