
$\Expect$

$\calG \calN \calE \calL$

\section{Bias, Variance, Generalization and Model Complexity}


\subsection{Hastie Tibshiranie Friedman}

\textbf{Gralization}

Given a problem setting, it is said that an algorithm's predictive power relies on its ability to correctly label samples on independent test data. Measuring and comparing this ability to correctly label data among models is central to supervised learning.  The best learner will be said to have the lowest \textit{predictive error} where this error can be divided into an "error" due to \textit{bias} and an "error" due to \textit{variance}.

Conceputally the first type of error arises to the accuracy of the model in labeling predictions correctly. Either by assigning the sample to its correct class in classification settings or by predicting correctly the sample's target value. In this part, the loss function plays an important part in determining how \textit{good} the model's prediction and quantifying these errors. Given a training set $\mathrm{T} = (X,Y)$, let $\hat{f}(X)$ be the model's prediction on the features and denote $L( Y,\hat{f(X)} )$ the loss function.

\begin{defintion}{Generalization Error}
The generalizatione error, test error or prediction error is the model's expected error over an independent test set $\mathrm{Ts}$:

$$ Err_{\mathrm{T}} = \Expect \left[ L(Y,\hat{f}(X)) |  \mathrm{T}\right]$$

Note that in this definition the expectation is taken conditional on the training set and the samples used to measure the error are drawn from an independent test set. here the model is fit \textbf{only} from the training set.
\end{definition}
 



For the next part, \textbf{x} $\in \mathbb{R}^{p}$ will denote a random input variable and \textbf{y}  $\in \mathbb{R}$ will denote a random output variable with joint distribution $P\left(\textbf{x},\textbf{y}\right)$ while $f$ will represent a function. 

We define $EPE\left(f \right) = \Expect\left[L\left(\textbf{y} - f(\textbf{x}) \right) \right] $ where $L(y,f\left(x\right))$ is called the loss-function which is a \textbf{semimetric} chosen to penalize errors in prediction. In general a cuadratic or absolute value functions are chosen, where the first is more favored for its smoothness. 

For the the squared loss case and when conditioning on $\textbf{x}$ it results that

\begin{equation} \label{eq:expectedError}
EPE\left(f \right) = \int [y - f(x)]^2 P(x,y)dxdy

\\
= \Expect_{\textbf{x}} \left[ \Expect_{\textbf{y}|\textbf{x}} \left[  \left( \textbf{y} - f(\textbf{x})  \right)^2 \right]  \right]

%\sum_{i=0}^{\infty} a_i x^i
\end{equation}

%\textit{The generalization performance of a learning method relates to its prediction capability on independent test data. Assessment of this performance is extremely important in practice, since it guides the choice of learning method or model, and gives us a measure of the quality of the ultimately chosen model.}

\textbf{Overfitting}: 
It is said that a learner overfits a training set \mathcal{T} if it fits parameter $\hat{\theta}$ while there $\exists \theta^*$ such that
\begin{equation} \label{eq:overfitting}
Error_{train}(\hat{\theta}) < Error_{train}(\theta^*) \  and \ Error_{test}(\theta^*) < Error_{test}(\hat{\theta})   
%\sum_{i=0}^{\infty} a_i x^i
\end{equation}

\textbf{Cross Validation}:  
\textit{simplest and most widely used method... This method directly estimates the expected extra-sample error
$Error = \Expect[L(Y,\hat{f}(X))] $ i.e. the average the average generalization error when the method $\hat{f(X)}$ is applied to an independent test sample from the joint distribution of $X$ and $Y$ . As mentioned earlier, we might hope that cross-validation estimates the conditional error, with the training set. $\mathrm{T}$ held fixed. But cross-validation typically estimates well only the expected prediction error.}

\textit{Ideally, if we had enough data, we would set aside a validation set and use it to assess the performance of our prediction model. Since data are often scarce, this is usually not possible. To finesse the problem, K-fold cross-validation uses part of the available data to fit the model, and a different part to test it. We split the data into K roughly equal-sized parts; }

\textit{Let $k : \{1,..,N\} \mapsto \{1, .., K\}$ be an indexing function are more details be an indexing
function that indicates the partition to which observation i is allocated by the randomization. Denote by}


\subsection{Bengio, GrandValet - No Unbiased Estimator of the Variance of K-Fold Cross-Validation}
\textbf{Cross Validation}: 
\textit{The standard measure of accuracy for trained models is the prediction error (PE), i.e. the expected loss on future examples. Learning algorithms themselves are often compared on their average performance, which estimates expected value of prediction error (EPE) over training sets.
The hold-out technique does not account for the variance with respect to the training set, and may thus be considered inappropriate for the purpose of algorithm comparison [4]. Moreover, it makes an inefficient use of data which forbids its application to small sample sizes. In this situation, one resorts to computer intensive resampling methods such as cross-validation or bootstrap to estimate PE or EPE. We focus here on K-fold cross-validation. While it is known that cross-validation provides an unbiased estimate of EPE, it is also known that its variance may be very large.
Some distribution-free bounds on the deviations of cross-validation are available, but they are specific to locally defined classifiers, such as nearest neighbors.
We focus on the standard K-fold cross-validation procedure, with no overlap between test sets: each example is used once and only once as a test example.
}


\textbf{HyperParameters}:
Como van apareciendo en algunos algoritmos y are different from the "parameters" or coefficients of the learners. They appear as a consequence of numerical, computational and sometimes statistical fine-tuning of algorithms (give an example?). 
La relacion entre cross-validation y la busqueda de hiper parametros. 

\textit{}

\textit{}


