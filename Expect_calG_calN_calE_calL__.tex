 
$\Expect$

$\calG \calN \calE \calL$

\section{Bias, Variance, Generalization and Model Complexity}


\subsection{Hastie Tibshiranie Friedman}

\textbf{Generalization}

Given a problem setting, it is said that an algorithm's predictive power relies on its ability to correctly label samples on independent test data. Measuring and comparing this ability among models is done to select the best supervised estimator.  The best learner will be said to have the lowest \textit{predictive error}. It is common in the literature \cite{james-biasVarianceGeneral} to separate the sources for this error in two concepts: bias and variance. Controlling these errors is central to any supervised machine learning problem.

Conceptually the first type of error arises to the accuracy of the model in labeling predictions correctly. Either by assigning the sample to its correct class in classification settings or by predicting correctly the sample's target value. It is lower when models learn on the underlying structures of the data. However, as models learn more from the data i.e. the training set, they lose the ability to extend this predictive accuracy to new samples because they \textit{overfit} on the samples available. This second type of error is commonly to as increase in variance.

In the generalization error, the loss function plays an important part in determining how \textit{good} the model's prediction and quantifying these errors. Given a training set $\mathrm{T} = (X,Y)$, let $\hat{f}(X)$ be the model's predictive function for the features and denote $L( Y,\hat{f(X)} )$ the loss function. 

\cite{james-biasVarianceGeneral}

\begin{definition}{Generalization Error}
	The generalization error is also known as the prediction error. Given that any model is always built on the data, this value is the model's expected error over an independent training set $\mathrm{Ts}$:
	$$ Err_{\mathrm{T}} =  EPE(\hat{f})= \Expect_{\textbf{X},\textbf{Y}} \left[ L(Y,\hat{f}(X)) |  \mathrm{T}\right]$$
\end{definition}\footnote{Note that in this definition the expectation is taken conditional on the training set which is from where the model is fit.}

Historically the errors due to bias and variance where associated directly with the squared loss function which is defined as $L(z,w) = \left\Vert z-w \right\Vert^2_2$. This function was favored over other functions because of its analytical advantages.  The generalization for this case is

\begin{equation}\label{squaredEPE}
EPE = \Expect_X \Expect_{Y|X} \left[ \left\Vert  Y - \hat{f}(X)  \right\Vert_2^2 \right]
\end{equation}

It is not difficult to see that the model that minimizes this error is $\hat{f}(x) = \Expect \left[ Y | X=x \right] $. Suppose now that for a given problem setting, there exists a true functional relation $Y = f(X) + \epsilon$. 
We will assume that $\epsilon$ is the noise, with a fixed variance $\sigma^2$ and $0$ mean. Our model $\hat{f}$ will be an estimate of this true relation, constructed from the data. And for this loss function, its error $\Expect \left[ \left\Vert Y  - \hat{f}(X) \right\Vert_2^2 \right]$ can be decomposed in the following way:

\begin{equation}\label{squaredBiasDecomposition}
EPE( \hat{f} ) = \Expect \left[   f(X)  - \hat{f}(X) \right]^2 +  \Expect \left[ \hat{f}(X)^2  \right] - \Expect \left[ \hat{f}(X)  \right]^2  + \sigma^2
= Bias(\hat{f})^2 + Var(\hat{f}) + \sigma^2
\end{equation}

Equation \ref{squaredBiasDecomposition} is referred to as the bias-variance decomposition for the squared loss. The first term is called the square of the bias of the estimator. It measures how well off are our estimator's predictions compared to the true relational function. On the other hand, the second and third terms are the variance of the estimator. This will measure how will this random variable vary along its most expected non-random value. The noise's variance term is that part of the prediction error which is irreducible. This means this is part of the random nature of the problem and one for which the estimator cannot lower.

For more general loss functions other than the squared error, a generalized notion of bias and variance is found in \cite{james-biasVarianceGeneral}. The author defines what properties should be displayed by the bias and variance of any estimator. 
\begin{definition}{Systematic Value}
	Given a training set $\mathrm{T}$, a loss function $L(\cdot)$ and an estimator $\hat{f}$ built from the data, the systematic value or systematic component of a learner is 
	$$ S\hat(f)  =  arg \ min_u \Expect_{\textbf{X}} \left[ L(u,\hat{f}(X)) \right]$$
\end{definition}

The systematic value for the target variable $SY$ is defined in the same way, where the expectation is over the target's distribution. 

In the general setting, the criteria the bias should meet is to be a measure of the systematic difference in which a random variable differs from a particular target value and also the measure of how much this systematic difference contributes to the error. On the other hand the variance of an estimator should measure the spread of the estimator around its systematic component and also the effect this variance has on the prediction. In \cite{james-biasVarianceGeneral} variance and bias are defined in the following way:


\begin{equation}
\begin{split}
& Bias(\hat{f})^2 = L(S\hat{f},SY)) \\
& Var(\hat{f}) = \Expect_{\hat{f}} \left[  L(\hat{f}  , S\hat{f}) \right]
\end{split}
\end{equation}

Note that with this definitions the variance is an operator defined only for the estimator and which is null only when the predictor is a constant value for any training set. As such, it is unchanged by new data. The bias on the other hand is an operator built form the systemic values of $Y$ and $\hat{f}$ and is null only if both of this values are equal.
For the specific case of the squared loss, these definitions coincide with the errors we defined before.


For a  multi-class supervised problem setting, $Y$ will be taking any of the values in the class set \calG and we will denote $K = |\calG|$ as the number of classes.

In practice there are only a limited amount of samples characterizing the true underlying distribution. We must 

\begin{definition}{Expected Prediction Error}
	Related to the generalization error, the expected prediction error is the expectation of the generalization error:
	$$ EPE = \Expect \left[ L(Y,\hat{f}(X))\right] =  \Expect \left[ Err_{\mathrm{T}}  \right]$$
\end{definition}

The generalization error is also known as the prediction error. This is the model's expected error over an independent training set $\mathrm{Ts}$. Here the expectation is taken over all the random elements of this process including the samples and the model fit from these. 

In a supervised learning context, the estimation of these errors is important for the assessment of the models.  Given that we will be working only with information from the training set, then estimation of the second error won' be't possible in most cases.

By conditioning on the training set we may rewrite the above formula as 

\begin{equation}\label{classificationEPE}
	 EPE = \Expect_x\left[ \sum_{k=1}^{K} L( Y_K , \hat{f}(X) ) P(Y_k|X) \right]
\end{equation}


\begin{definition}{Training Error}
	is the average loss over the training set
	$$ \overline{err} = \frac{1}{N} \sum_i=1^N L(y_i, \hat{f}(x_i) )$$
\end{definition}


For the next part, \textbf{x} $\in \mathbb{R}^{p}$ will denote a random input variable and \textbf{y}  $\in \mathbb{R}$ will denote a random output variable with joint distribution $P\left(\textbf{x},\textbf{y}\right)$.

We define $EPE\left(f \right) = \Expect\left[L\left(\textbf{y} - f(\textbf{x}) \right) \right] $ where $L(y,f\left(x\right))$ is called the loss-function which is a \textbf{semimetric} chosen to penalize errors in prediction. In general a quadratic or absolute value functions are chosen, where the first is more favored for its smoothness. 


Models are evaluated according to their performance in these two errors in training and testing datasets. Combinations of high or low values for these, across training and test data, are used as model evaluation and highlight aspects to be improved. 

For example, a very common scenario is having a model that has high overall variance and low bias. 
It is said that a learner overfits a training set \mathcal{T} if it fits parameter $\hat{\theta}$ while there $\exists \theta^*$ such that
\begin{equation} \label{eq:overfitting}
Error_{train}(\hat{\theta}) < Error_{train}(\theta^*) \  and \ Error_{test}(\theta^*) < Error_{test}(\hat{\theta})   
%\sum_{i=0}^{\infty} a_i x^i
\end{equation}

Once we fit on the training set, the learner overfits because predictions on training samples are very accurate whilst predictions on new samples spread out. The model has learned very well from the training data but fails to be accurate on new samples.

A trivial algorithm example of overfitting would  that would 

An opposite scneario scenario occurs when


rise Typical combinations
The four possible combinations of high 

