\chapter{Conclusions}\label{ch:conclusions}

\todo{Reorder and correct this section.}

The purpose of this work was to analyze and explore howthis type of data data could be used as a surrogate to other, more traditional, methods of disease control and surveillance.
This was done starting from the hypothesis that long-term migrations 
are relevant to the spread of this disease out from the region source of vector-borne infections.
It also took advantage of the georeferenced information contained in the CDR logs, where we showed how the use of this data for reasons other than logging a user\'s calling expenses demonstrates a use which is relevant to public health policies. 


The analysis we exposed in this work allowed us to show that the Risk maps we developed are of interest to national health campaigns.
They pinpoint to likely spots of high disease prevalence, with a low cost to produce this outcome.
Added to this, the risk maps produced many interesting insights of specific antennas that have higher vulnerable interactions with the endemic region, where these are located outside of the endemic region.



From this, we intend to provide an example of how Machine Learning tools on CDR data can help authorities with disease-control strategies in way which is not intrusive to cellphone users.
Health-related measures can then be applied outside of the endemic region, and directed towards specific neighbourhoods and communities by using the recommendations output by the risk algorithm.

Next to traditional health control actions, this research may provide a very insightful piece of information for decision-makers.
Where the insights derived from the probabilistic models and the risk maps can help in understanding disease spread in other similar contexts.


We think that capturing long-term human movements will also enable better simulations of disease spreads in different contexts.
In this work, the impact of this in our predictive capacity of disease prevalence is shown.


We believe it is acceptable to say that long-term human dynamics in the form of migration patterns can be captured by the use of this data.
Since this work was performed in two Latin-American countries and in one single endemic region, we expect that this CDR usage can be extended to other similar countries and diseases, with demographic, cultural and geographical similarities beyond Chagasic spread in Mexico or Argentina.



The supervised classification experiments done on the various prediction tasks were discussed in previous sections.
The results highlighted the benefit of using this type of data for the problem.


% in the preceding chapters to our tasks.
The best learners for the following tasks were the Random Forests and Gradient Boosting models.
Due to this, we focused our work on the strengths of the ensemble learners to achieve higher predictions.
Still, we saw differences in both ensemble methods, where the former was better at this score, whilst the latter at the second.


The features most brought up in the experiments of ensemble learners where A, B, c.
These indicated this and that.
They also pointed to information in  blablabla.
Some supervised classification models realized the importance of attributes tagged with vulnerable behavior obtained from the CDR.\@

The Random Forest model played an important role in having the highest scores across all tasks and in presenting the best features of the dataset.
This information was later availed by the boosting models trained on said set of features.



With this, we showed that it is possible to use the mobile phone records of users during a bounded period (of 5 months) in order to predict whether they have lived in the endemic zone $E_Z$ in a previous time frame (of 19 months).

Combining social and geolocated information, the data at hand has been given an innovative use, different from its original billing purpose.

% To conclude, the results presented in this work show that


Currently, epidemic counter-measures include coordinating national surveillance systems with institutions and primary health care organizations, vector-centered policy interventions through fumigation of vector-infested regions and individual in-field screening of people.

These measures require costly infrastructures to set up and be run.
On the other hand, this work shows built on top of existing mobile networks would demand lower costs, taking advantage of the already available infrastructure
The potential value these results could add to health research is hereby exposed.
Finally, the results stand as a proof of concept which can be extended to other countries or to diseases with similar characteristics.


\begin{itemize}

    \item \textbf{Conclusion}

    \item This method and dataset showed a systematic approach to analyze the information contained in the dataset.
    The idea was to try to explore the human mobility patterns, as captured by the phone call records.

    \item Heatmap allowed discovery of anomalous communities with attributes from the data which were later confirmed to be relevant in helping detect long-term movement of users.

    \item We compared the strengths and weaknesses of these methods.
    The algorithms  were used to estimate the probability of each user's migrations.
    Also, we showcased the method's results differences for the tasks defined in \cref{ch:machineLearning}.

    \item Where possible, we exposed the best features of the dataset to solve the general problem.
    These exposed the relationship between the attributes that characterize vulnerable user interactions, and the target variable of the task.
    Other relevant features, such as the user's mobility size, were also found to provide relevant information for the classification task.

    \item According to our results, there are algorithms which are reliable enough to detect users which have migrated from old regions. In some instances, we have achieved high values across all scoring measures. To summarize the work done we can say that:
     % the algorithms detected that the characterization of the user's mobility provided relevant information to the task

\end{itemize}



In this work on CDRs, the Argentinean and Mexican case studies allow us to find that it is possible to characterize human movements of long duration.
Where cellphone datasets are rich enough in information to detect detailed patterns of users moving to and from a particular area, at a national level.


Results show evidence indicating that a supervised classification predictor can be built to detect large patterns of human migrations over time.
They also point to the idea that CDRs are particularly well suited for this task, where it is possible to explore this data as a mean to tag human mobility.
These tests effectively helped understand how relevant the dataset is to the question, where the complexity and scale of the data can be, up to a certain degree, assimilated by probabilistic algorithms.
% , using the models outlined before.

Large at-scale and temporal user mobility were captured by the CDRs, allowing us to understand a variety of human movements throughout the country.


% \section{In-depth discussion}
% In this work we took advantage of the large user scale and temporal range of the CDRs allowed us to capture a variety of human movements throughout the country.


Migration patterns can then be used to suggest actual epidemic spread of this disease into areas not deemed endemic.



\section{Drawbacks from our methodology}

There are several points can be stated to point to weak spots on the methodology used.
An important observation is that we can not directly jump from the human mobility insights found in this study, to conclusions from a model of disease spread.
From this work, we were looking solely at human mobility to and from endemic regions, which does not imply disease spread or prevalence.
However, on the advice of this topic's researchers, this work does bring valuable insights to the problem, reinforcing previous hypothesis that vulnerable users are not only to be found inside vector-infested regions.


Added to this, we have stated numerous times before that this methods suffers on highly skewed target classes.
The low $F1$ scores achieved on the problem of tagging user that migrated out of the endemic region, across all of the learners, pinpoint a shows a very significant negative outcome using this methodology and these attributes.

Another issue with this work is the inherent biases in our dataset.
As shown in \cref{tab:distribution_by_state}, there are significant differences in the percentage of population represented by our dataset vs.\ current state distribution estimates.
Due to this, we have to interpret results with caution, since we do not hold any real-data comparisons.
For this work we did not find any available georeferenced disease data outside of the endemic region, at municipal or regional levels.\footnote{As a matter of fact, we tried to contact health institutions and research organizations working on the matter with the purpose of enriching the original dataset.
This intent did not add any disease-related data of the kind.}

Other interesting biases stemming from the data are related to international pattern migrations.
For this work, we measured migrations only from phone owners within a single country.
Yet we have seen in \cref{ch:descr-risk}, that  epidemic regions traverse political borders.
The current dataset is limited to only capturing international migrations, and with this our analysis is limited to movements of national Telco users only.
A similar argument applies to the detection of mobility patterns for people with no cell-phone usage.
 % lack cell phone antennas.

Our dataset also presents problems when processing a user's home antenna.
To do this we had to define what we thought were the expected ``working hours'' for all users.
In doing so, we also assumed that this could be generalized to all samples. %used in a generalized case.
Even though this decision was supported from past research we referenced, the definitions might not be easily translated for datasets across other geographical and cultural regions.
% This was done by considering what we thought


Finally, we know that we can have significant time seasonality and stationarity in the data, yet we did not consider statistical methods to reduce this effect in a thorough fashion. % but only in a limited fashion.
When processing the data, we specifically tagged weekend and working hour attributes separately from the rest of the features.
The same was done for month specific attributes on user calling patterns.
However we understand that there could be other longer term time effect which was not considered under our current codebase. %used in this work.
There are possibilities of migration patterns being strongly related to seasonal factors such as, for example, in agricultural workers having seasonal migrations around different regions.% workforce.

The importance of other user specific bias, such as demographics unbalance, was not thoroughly examined.
Yet these were out of the scope of this work, due to the lack of ground truth in the data.


% discriminated in the construction of the features, yet there were no long-term time



\section{ Lines of Future Work }

The mobility and social information extracted from CDRs analysis has been shown to be of practical use for long-term human migrations and for Chagas disease research.
it adds value and information to help make data driven decisions which in turn is key to support epidemiological policy interventions in the region.
For the purpose of continuing this research line, the following is a list of possible extensions:


\begin{description}
    \item [Results validation.] Compare observed experiment results of risk maps and best features against actual serology or disease prevalence surveys.
    Data collected from fieldwork could be fed to the algorithm in order to supervise the learning towards a target variable that is defined by the disease prevalence at a certain level.


    \item [Differentiating rural antennas from urban ones.] This is important as studies show that rural areas have epidemiological conditions which are more favorable to the expansion of the disease expansion.
    The vector-borne transmission through \textit{Trypanosoma cruzi} is helped with poor housing materials and domestic animals.
    All  contribute to complete the parasite's life-cycle.
    Using the CDR data, antennas could be automatically tagged as rural by analyzing the differences between the spatial distribution of the antennas in each area.
    A similar goal could be to identify precarious settlements within urban areas, with the help of census data sources.

    \item [International and seasonal migration analysis.] Experts from the \textit{Mundo Sano} Foundation underlined that many seasonal and international migrations occur in the \textit{Gran Chaco} region.
    Workers are known to leave the endemic area for several months possibly introducing the parasite to foreign populations.
    The same happens with foreign workers, which are not part of our dataset.
    The mobility analysis on particular time periods or events, for example on holidays, or specific migrations from bordering countries, can give information on which communities have a higher influx of people from the endemic zone during a certain period.
    This additional analysis would also lower the bias of the current dataset.


    \item [Search for epidemiological data at a detailed level.] For instance, specific historical infection cases.
    Splitting the endemic region according to the infection rate in different areas, or considering particular infections.
    With the use of a high-quality epidemic dataset, we assume that a more model complex can be built, to detect infected users nationwide.
    It is of primary importance though to have a good dataset available.
    Since otherwise, it would not be possible to reach a generalizing classifier.

    \item With the above, we could add a similar systematic process as the one defined in this work, to explore features and determine for the most relevant attributes that are associated with being Chagas infected.

    \item [Test other classifiers] The probabilistic models introduced in this work are a brief list of the ones available in the literature and of these, we know that only Naive Bayes had an acceptable classification performance in \cref{target2}.
    It would be interesting to see how other well-established and known methods perform for this case and see if any of them can correct this.
    Some examples of other algorithms not shown in this work are K-Nearest Neighbors, Support Vector Machines and Classification Neural Networks.   

    \item [Test other pre-processing techniques] In line with the problem addressed by the previous point, there are multiple other data-based techniques to address the problem of learning with highly unbalanced classes.
    These tend to focus on undersampling the majority class, over-sampling the minority class, or a combination of both.


    \item [Domain Evaluation] The evaluation of Machine Learning algorithms needs to incorporate domain specific knowledge to reasses the errors accordingly. In this sense, we can say that Type I or Type II errors may not be the same at the eyes of a health researcher.
    In a future iteration, it would be interesting to evaluate again the results with the opinion of other domain experts where the algorithm's performance could be seen differently.

\end{description}

