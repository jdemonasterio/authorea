%===============================================================================
%          File: preface.tex
%        Author: Juan de Monasterio
%       Created: 08 Feb 2017
% Last Modified: 08 Feb 2017
%   Description: Resumen de la Tesis
%===============================================================================
\chapter{Resumen}
\label{cha:resumen}

En la actualidad, la enfermedad del Chagas sigue siendo una epidemia de escala global.
Ésta se encuentra extendida en gran medida a lo largo de todo el continente americano y con métodos de control que se concentran sobre las regiones infestadas por el vector de transmisión.
%Sin embargo, los patrones de movilidad humana también representan un factor importante en la transmisión congénita de esta enfermedad. Su propagación de áreas de alta a baja infestación son fortalecidas por migraciones de tipo estacionales o de larga duración.
Los patrones de movilidad humana representan un factor importante en la propagación geográfica de esta enfermedad.
El movimiento de áreas de alta a baja infestación es fortalecida por migraciones de tipo estacionales o de larga duración.
Usando datos anonimizados de llamados por telefonía celular, y en colaboración con la Fundación Mundo Sano, en este trabajo se busca evaluar la relación entre los patrones de uso celular y el de las migraciones humanas, explorando las relaciones sociales subyacentes en las interacciones de telefonía móvil.
Este análisis servirá para ayudar en los esfuerzos de prevención del Chagas en Argentina y México, donde se lograron identificar posibles focos endémicos de esta enfermedad fuera de la zona de infestación vectorial.
Para hacer esto, utilizamos diferentes modelos probabilísticos del subcampo de Aprendizaje Automático.
Para cada usuario más de 150 atributos fueron extraídos de los datos y analizados para evaluar la probabilidad de que el usuario haya migrado desde la zona endémica.
Los resultados aquí expuestos sirven para discriminar aquellos atributos más relevantes en la detección de estos movimientos, especialmente en usuarios de fuertes lazos sociales con la región endémica, de alta movilidad y en usuarios con un alto grado de movilidad.
A través de visualizaciones geográficas, logramos agregar estas relaciones sociales para mostrar locaciones tradicionalmente no endémicas que potencialmente sean focos endémicos y que aún no son como tales.

