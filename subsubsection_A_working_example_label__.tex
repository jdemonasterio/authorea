\subsubsection{A working example}\label{section-example}

For the purpose of this work we will be talking of the training set, noted by $\mathrm{T}$ as the set of data examples. The objective is to build a probabilistic model which has the capacity to correctly predict the class instance of new data objects based on having seen information of other data objects.

For concreteness, let's consider a reduced dataset built from Call Detail Records (CDRs) where samples are calls being made by users who can belong to any of following provinces : \textit{Buenos Aires}, \textit{Cordoba} and \textit{Santa Fe}. 
%\url{http://stackoverflow.com/}
%\footnote{For more information on this set, a historic review is given at \url{https://en.wikipedia.org/wiki/Iris_flower_data_set}}.
Five measurements were taken on all of the observations to account for the user's number of calls and total minute duration of calls during a week of measurements. A short overview of this dataset is printed below:

\begin{table}[ht]
	\caption{Head of the CDR dataset}
	\label{tab:sample_CDR}
	\centering
	\begin{tabular}{ l l l l l l }
		\toprule
		User & CallsWeekend & TimeWeekend & CallsWeekDays & TimeWeekday & Province \\
		\midrule
		BA343E  & 15 &  89 & 8 & 24 &  \textit{Santa Fe}\\
		73F169  & 10 &  121 & 2 & 98  &  \textit{Cordoba} \\
		EA23AD  & 12 &  43 & 5 & 154 &  \textit{Buenos Aires} \\
		\bottomrule
	\end{tabular}
\end{table}

In this form a row is representing the available data acquired from each user and the columns represent types of measurements or information on that user. In general, most machine learning problems will be associated with a training set $\mathrm{T}$ of similar form as the one shown before. Where rows represent objects or \textit{samples} and columns are measurements or \textit{features} of our samples. 

Here $\mathrm{T}$ will take the form of a paired couple of datasets $(X,Y)$ where $X \in \mathbb{R}^{n \times p}$  and $Y \in \mathbb{R}^n $. The $jth$ column of $X$ or equivalently the $jth$ feature of the data is denoted by $X^j$. Similarly, the $ith$ row or sample of $X$ is denoted by $X_i$ or $x$ when it is clear from context or when we are referring to a generic sample. A similar notation is used with the outputs, where $Y_i$ of $y$ will be used to denote the target of a specific sample depending on context. 

In this particular case, even though the last column of the dataset in \ref{tab:sample_CDR} is not a measurement \textit{per se}, it gives information on each user's province of residence. With this we can map users to \textit{classes} which are elements of the set of provinces.

Hereon, there are various questions one could try to answer. Examples of problems that a machine learning algorithm could tackle from this dataset could be: 

\begin{itemize}
	\item Predict a user's province when given information on only the first four features.
	\item Predict a user's number of calls made on weekends when given information on the last four features.
	\item Give an estimate of the probability density function for a user's calling time during weekdays.
\end{itemize}

The first two problems are examples of supervised learning where the $Y$ variables or \textit{labels} are the last and first features (columns) respectively. Even more, the first problem is a classification one since users are to be classified according to one of the three possible provinces whilst the second one is a regression problem for which the output could be any of a range of numerical values. The labels in classification problems are numerically encoded with a finite range of numbers where $\{0,1\}$ or $\{-1,1\}$ are usually used for the binary case.

Note that there are no assumptions made here about the data which we take as is given. %to us in this way. 
This is common in machine learning applications and because of this, algorithms tend to be designed to account for this situation. The type of problems and questions that could be done then depend entirely on the training set. 

The last example problem in the list before, belongs to the unsupervised learning category where there is no need to have a label on the data. Here the question is on the structure of a specific column, namely the estimation of the true probability distribution of the calling time. In this setting there is no output expected for new samples.

To illustrate further the supervised classification scenario, a brief notation outline is described below following a standard logistic regression classifier example:

Let's suppose that we want to build a predictive model to determine the origin province of a user. A classification algorithm should assign samples to provinces. Let $\{C_1,..,C_k\}$ denote the set of possible target classes. We are  interested in maximizing the probability of belonging to a certain class, given the phone data:
$$p(C_k| X) = \frac{p(x|C_k)p(C_k)}{p(x)} $$

In this interpretation, $p(C_k)$ is known as the prior probability of belonging to that certain class and $p(C_k|x)$ as the posterior probability, given the data. 

Our classifiers will partition input space into decision regions $R_k$ for which a class $C_k$ is uniquely associated to a region. It makes sense to try and \textit{minmize} the chances of assigning samples to incorrect regions. For example, take a problem with $K$ classes. Given a sample $x_i$, the probability of a correct classification is measured by 

\begin{equation}\label{eq:goodclassification-equation}
\begin{split}
=  & \sum_{k=1}^{K} p(x_i \in R_k, C_k )  \\
=  & \sum_{k=1}^{K} \int_{R_k}p(x_i,C_k) dx \\
=  & \sum_{k=1}^{K} \int_{R_k}p(C_k \mid x_i) p(x_i) dx 
\end{split}
\end{equation}

Observe that the measure is completely characterized by the posterior probabilities. The factor $p(x)$ is common to both integrals so we only need to maximize the posteriors. And even when $p(x)$ might not be known or accurately estimated, the algorithm can only modify the decision regions to improve the probability of correctly classifying samples. Thus the goal of the algorithm will then be to chose the best possible regions $R_k$ for the problem.

A more reduced problem instance would be to determine if a user belongs to the province of \textit{Cordoba} vs the rest of the provinces. This brings us to a binary classification problem since there are two possible output classes. %us to a two class learning problem. 

As a simple solution, we could build a learner $f$ from a  linear combination of the input features combined with a transformation in order to predict the target variable. In the ideal case we would have that for every sample $y \approx f(x) = h\left(\sum_{j}\theta^jx^j\right)$ \label{formula:1} \footnote{In formula \ref{formula:1} we have omitted the intercept parameter , generally noted as $\theta_0$. The reason for this is that one can include this parameter implicitly if we allow for a synthetic feature $X^0$ in $X$ which is a vector with all of its components equal to 1.  } where $\theta$ is unknown and generally referred to as the coefficient or parameter. Here the model is said to be \textit{linear} because it is linear in $X$.

If we let our decision boundary be $ t_i = \theta \cdot x_i  \ \forall \ i $,in the ideal case we will have that $\exists \theta s.t. \forall i $ 
\[
%\begin{equation}
t_i
\begin{cases}
&>0 \ \mbox{if} \ y_i=1 \\
&<0 \ \mbox{if} \ y_i=0.
\end{cases}
%\end{equation}
\]
%       \forall 1 \leq i \leq n = y^i $

As an example, we could take the heavyside step function as our transformation $h(z)$  where

\[
%\begin{equation}
h(z) =
\begin{cases}
&0 \ \mbox{if} \ z<0 \\
&\frac{1}{2} \ \mbox{if} \  z=0 \\
&1 \ \mbox{if} \  z>0.
\end{cases}
%\end{equation}
\]

If the goal of the model is to maximize $P(Y_i = y_i | X_i = x) \ \forall i$
under the assumption that $\nexists\  i \  s.t. \ t_i = 0$, the algorithm would then correctly classify all samples to their targets. However this situation is hardly the case, so the common approach would be to rely on optimization procedures to minimize the amount of misclassification cases. This would be analogous to maximizing the probability of a correct classification in  \ref{eq:goodclassification-equation}.

For analytical convenience, a more tractable function for this task is the logistic which is an approximation of the heavyside step function. Here  

\begin{equation} \label{eq:logisticFunction}
\sigma(z)  = \frac{e^{z}}{1 + e^{z}} = \frac{1}{1 + e^{-z}} 
\end{equation}

where  we use the relation

\begin{equation}
 \  H(z) = \lim_{k \to \infty} \left(\frac{1}{2} + \frac{1}{2}tanh(kz) \right) = \lim_{k \to \infty} \left(\frac{1}{1+e^{-kz}} \right)  
\end{equation}

The logistic function has the advantageof being smooth, well defined for all real numbers and the image of the funciton is $(0,1)$. This property lends itself to reading outputs as probabilities of targets belonging to a certain class. The derivative of this function can be put in terms of the function itself, where

\begin{equation} \label{eq:derivativeLogisticFunction}
\sigma '(a) = \sigma(a)( 1 - \sigma(a) )    
\end{equation}

The logistic function is also bijective, with the inverse given by the \textit{logit} function 

\begin{equation} \label{eq:logitFunction}
\sigma^{-1}(z)  = log( \frac{z}{1 - z})    
\end{equation}

For the scenario characterized in \ref{formula:1} we would have to finally assign each output to a specific class. A common approach for this is to categorize each output whether $\hat{y} > 0.5$ \label{formula:logitThreshold}. Notice that having $h(x \cdot \theta) = 0.5$ implies that $x \cdot  \theta = 0$ and thus our classifier is separating samples in feature space (the space of the inputs) with the hyper-plane defined by the parameter $\theta$. Having a higher $\hat{y}$ for a given sample implies that it is further away from the hyper-plane. The same goes for low estimated targets. If we were to read this as a probability, we can interpret that the algorithm is modeling the posterior probability $P(Y_i = y | X_i = x)$ and would correspond to having a higher confidence in the classification.

%This means that 

%It also happens to be an approximating function to the


\end{document}
