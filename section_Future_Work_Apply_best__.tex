\section{Future Work}
We have some lines for future work:
\begin{itemize}
    \item Tag rural antennas: rural areas are more vulnerable because rural constructs host triatomines more easily. Also, the presence of farm animals increase the risk of infection. The tagging could be done automatically if we prove that rural antennas have a different spatial distribution than urban ones. We also could try to tag precarious settlements within cities (with the aid of external information).

\end{itemize}
Apply best fit model to Argentinian dataset. Provide that info to a Chagas risk model. Assuming that a high influx of individuals from epidemic regions is correlated with a higher risk in that area the algorithm could highlight these points.
TAG rural antennas.

FEATURE EXPLORATION. Hypothesis testing. 

SEASONAL MIGRATIONS AND OTHER regions

Search for EPIDEMIOLOGICAL DATA at a finer grain. (such as specific infection cases, ) Differencing the epidemic region on specific infection cases rather than on
Maybe even compare model with field results from NGOs partnered in this project.

\section{Acknowledgements}
Mundo Sano. Diego, Marcelo, Marcelo2