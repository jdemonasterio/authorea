\section{Gradient Boosting Variation}

\subsection{Boosting Trees    }

$\Theta$

The objective function differences minimization objectives. 

Predictive power: how well the algorithm fits the data used for training and, hopefully, the \textit{true} underlying distribution

+ Regularization, favors parsimonious models. This is because all models approximate natural/processes to some degree, so if a simpler model can be used with the same predictive power, then this model should be used.  

\begin{equation} \label{eq:objFunction}
Obj(\Theta) \ = \ L(\Theta) + R(\Theta)
\\
%\sum_{i=0}^{\infty} a_i x^i
\end{equation}

When building a model of ensemble trees, a higher model will be learning on other \textit{weaker} decision trees. If $Tr$ is a set of tree models and $K$ the number of trees in $Tr$, then for this algorithm trees will be model's parameters and
\[ y = \sum_k f_k(x) , \ f_k \in Tr \ \forall t \in {0...K}\]

A model of a tree $f$ with $T$ amount of leaves and parameterized by leaves scores $\theta$ is defined as $f(x) = \theta_{q(x)}$  with $q : mathbb{R}^n \mapsto {1...T}$ a function assigning leaves to each observation. 

The objective function in gradient boosting would account for the relationships among trees and parameters, yielding a formula as such:
 
\[ Obj(\Theta) = \sum_i^n l(y_i,\hat{y}^i))  +  \sum_t R(f_t) \] \label{eq:boositing-objfunction} \footnote{In the formula \ref{eq:boositing-objfunction} }
%    

At a higher level we would have that $\Theta$ is a parameter encoding lower trees' parameter information. If $\theta_t$ is the parameter associated for each tree model $f_t$ then $\Theta =  \bigcup_{t \in {0...K}} \theta_k  \cup \theta_0$ where the parameter $\theta_0$ is not associated to any tree and reserved to characterize the tree assembly. An optimization routine will have to collectively fit all of the parameters in $\Theta$ to learn this model. This is prohibitively costly in practice so optimization heuristics are used instead. 

\subsubsection{Additive Training}

A first take on this optimization problem goes along the way of greedy algorithms. One tree is fit at a time and new trees are then successively added in later steps to improve on previous trees' errors.

Consider $t$ the step indexer of the algorithm, where $t \in {0,..,K}$. In step $t$, let $Obj_t(\Theta)$ be the objective function and $\hat{y}_t^i$ be the predictive target respectively. Then the $i$-eth target's value at each step would iterate in the following way:

\begin{equation} \label{eq:gb-targetSteps}
\hat{y}_0^i = 0 \\
 \ \ \ ... \\ 
 \hat{y}_t^i = \sum_{k=1}^{t} f_k(x^i) = \hat{y}_{t-1}^i +  f_t(x^i)

%\sum_{i=0}^{\infty} a_i x^i
\end{equation}
where each trees is added in such a way that we are minimizing

%\begin{equation} \label{eq:gb-objSteps1}
\[
Obj_t(\Theta) = \sum_i^n l(y^i,\hat{y_t}^i)) +   \sum_k^t R(f_k) \\
= \sum_i^n l(y^i,\hat{y}_{t-1}^i +  f_t(x^i) ) +   c(t) + R(f_t) \\
%= \sum_i^n {2(y^i -(\hat{y}_{t-1}^i +  f_t(x^i)) )} + (\hat{y}_{t-1}^i +  f_t(x^i))^2 +  f_t + c'(t)
\]
%\end{equation}

where $c(t)$ and $c'(t)$ are constants for fixed $t$. Approximating by a second order Taylor approximation yields

\[
    Obj_t(\Theta) \approx \sum_i^n l(y^i, \hat{y}_{t-1}^i +  f_t(x^i) ) + c(t) + R(f_t) \\
    = \sum_i^n {l(y^i, \hat{y}_{t-1}^i) + g^i f_t(x^i) + \frac{1}{2} h^i f_t(x^i)^2 } +  R(f_t) +  c'(t)
\]
where $g^i$ and $h^i$ are first and second order approximations of the loss function.
\[
    g^i =  \frac{\partial l(y^i, \hat{y}_{t-1}^i)}{\partial \hat{y}_{t-1}^i} \\
    h^i =  \frac{\partial^2 l(y^i, \hat{y}_{t-1}^i)}{\partial (\hat{y}_{t-1}^i)^2 }
\]

Then

\begin{equation} \label{eq:gb-objFun1}
    Obj_t(\Theta) \approx \sum_i^n {  g^i f_t(x^i) + \frac{1}{2} h^i f_t(x^i)^2 } +  R(f_t) + constant 
\end{equation} 

In a greedy optimization approach, the tree $f_t$ will minimize $\sum_i^n {  g^i f_t(x^i) + \frac{1}{2} h^i f_t(x^i)^2 } + R(f_t)$ at the $t$-th step. The assumptions with this approach are that $ \forall i \in {1...n}, \forall t \in {1..K}, \exists g^i(\hat{y}_{t}^i), h^i(\hat{y}_{t}^i) $ and that these values are effectively computable.

There is no a priori \textit{correct} or usual specification for the regularization term $R(f)$. As is the case with other algorithms, the regularization function is chosen through cross validation, among a list of possible forms. If we consider an $L_2$ norm approach, the model's complexity will be controlled by 

\begin{equation} \label{eq:gb-objFun2}
R(f) =  \gamma T + \frac{1}{2}\lambda \sum_{j=1}^T\theta_j^2 
\end{equation}

Here $T$, $\gamma$ and $\lambda$ are hyperparameters of the model. From \ref{eq:gb-objFun1} and \ref{eq:gb-objFun2} the general minimization function takes teh form

\begin{equation} \label{eq:gb-objFun}

    Obj_t(\Theta) \approx  \sum_i^n {g^i \theta_{q(x^i)} + \frac{1}{2} h^i \theta_{q(x^i)}^2 } + \gamma T 
        + \frac{1}{2}\lambda \sum_{j=1}^T\theta_j^2 \\
          =   \sum_{j=1}^T\left(  (\sum_{i \in T_j} g^i )\theta_{j} + \frac{1}{2} (\sum_{i \in T_j} h^i + \lambda ) \theta_{j}^2  \right) + \gamma T \\
         =   \sum_{j=1}^T\left(  G_j \theta_{j} + H_j \theta_{j}^2  \right) + \gamma T \\
         
\end{equation}
where $G_j = \sum_{i \in T_j} g^i $ ,  $H_j = \frac{1}{2} \sum_{i \in T_j} h^i + \lambda$ and $T_j = \{i | q(x_i)= j \}$ is the set of all samples assigned to the leaf $j$. 
it is common to choose is best fitted
is yet to be 

is  yet to be defined
$h^i$ on the existence 

where a second order Taylor expansion was used 

aggregating 
%$Obj





\subection{sklearn}


\subection{Hastie Tibshiranie Friedman}

