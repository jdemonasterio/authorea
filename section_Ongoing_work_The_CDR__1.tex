\section{Ongoing work}

The CDR logs available for Argentina span for 5 months, precluding the prediction of long-term mobilities using supervised algorithms. For the mexican dataset, users living in the ecoregion can be tested to their area of influence, looking back one year. Taking the period from January '14 to April '15 our target variable $Y $ is defined in the following way for every user $u$: 

% \eqref{PP1} reduces to
\[
    %\begin{equation}
    Y_u =
      \begin{cases}
        &1 \ \mbox{if} \ H_u \ \mbox{is from epidemic region}\\
        &0 \ \mbox{else}.
      \end{cases}
    %\end{equation}
    \]
    
\subsection{Data Attributes}
Setting the five month timespan limitation, CDRs are being processed to extract features at the user level. 

For every individual, the top ten most used antennas are logged along with the amount of use in each one. Users are tagged as 'epidemic' if their home antenna is in the the risk area and 'exposed' if any of the antennas used are in the epidemic zone. A mobility diameter is also processed from the radius of the convex hull defined by the user's logged antennas. This length is representative of the radius of influence of that individual and we are expecting to see a high correlation between a high mobility-diameter and high long-term movements of people.

Calling information is aggregated and binned according to the hour of the day and during the weekends. For every user, the duration and individual count of these calls are processed, differentiating between calls made to vulnerable and non-vulnerable users. 

Finally, social information processed from the data include the total amount of epidemic, exposed and total neighbours that any given user interacts with over the timeperiod. 

\subsection{Supervised Algorithms}
Based on 
ALGORITHMS being tested (Random Forests, Gradient Boosting, Multinomial Naive Bayes (Benchmark), Logistic Regression)


\subsection{Maps for Mexico}
MOSTRAR MAPAS O NO??