%===============================================================================
%          File: abstract.tex
%        Author: Juan de Monasterio
%       Created: 15 June 2017
%   Description: Abstract  o resumen en ingles.
%===============================================================================
\chapter{Abstract}
\label{cha:abstract}

The Chagas disease continues to represent a global epidemiological problem, particularly for the South American continent.
Current control methods focus solely on the vector-infested regions. 
%Still, human mobility patterns represent an important factor in the congenital transmission of this disease. 
%Its spread from high to low infested regions is strengthened by seasonal and long-term migrations. 
Still, human mobility patterns represent an important factor in the geographical spread of this disease, since its spread from high to low infested regions is strengthened by seasonal and long-term migrations. 
Using anonymized mobile phone data, and in collaboration with the Mundo Sano Foundation, the objective of this work is to assess the relationship of calling patterns and user behavior, with migrations. 
This analysis, in turn, helps for Chagas prevention efforts in Argentina and Mexico, where we identified possible endemic foci outside of vector-infested regions. 
To do this, we evaluated different Machine Learning techniques as probabilistic models. 
More than 150 features were extracted from the data and related to the probability of having lived or moved from an endemic region. 
The results here presented identify key features for the detection of these movements, especially in users with strong ties to the endemic regions or with high mobility patterns. 
By presenting geographic visualizations of social ties, we tag locations outside the endemic region with a hypothetical higher prevalence rate.
