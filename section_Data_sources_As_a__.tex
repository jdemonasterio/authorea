\section{Data sources}

As a general rule, privacy is assured by handling data in private servers and by differentiating users by their hashed ID, with encryption keys managed exclusively by the TelCo.

Other general characteristics are that logs are timestamped and geolocalized by the position of the antenna used to place the call. To exclude outlying users such as call-centers or dead phones, numbers whose monthly cellphone use did not surpass 5 calls and fall to at most 400 calls were automatically filtered.

\subsection{Argentina}

\subsection{Mexico}

The mexican data source is an anonymized dataset from a national mobile phone operator. Data is available for every call made within a period of 19 months from January 2014 to September 2015. The raw logs contain about 12 million calls per day for more than 8 million users that accessed the telecommunication company's (TelCo) network to place the call. This means that users from other TelCos are logged, as long as one of the users registering the call is a client of the operator. In practice, we only considered CDRs between TelCo users since geolocalization was only possible for this group.

Information logged for each call included the duration and timestamp of the call, the users participating in the call and the antenna id that transmitted the call to the TelCo client. 

(EJEMPLO DE UNA TABLA DE DATOS CRUDOS o simpleformat?)

\subsection{Data Limitations}

Although a lot of information is available in a one month CDR dataset, there maybe limitations in its representativeness. In each case, data is sourced from a single TelCo and no information is given on the distribution of its users. Thus calls not be correctly representing social interactions and movements between two given jurisdictions. Also, it is not straightforward that user movements will be captured every time by the data. A user might not be using his cellphone when he is on the move.
However, some of this limitations are offset by the datasets' sizes. We can assume that the amount of users observed in each set is enough to correlate real mobility or social links between different areas.

% Si podemos reflotemos el grafico de clustering de provincias segun comunicaciones, y repitamos para estados en Mexico.