\section{Future Work}

The mobility and social information extracted from CDRs analysis has been shown to be of practical use for Chagas disease research. Helping to make data driven decisions which in turn is key to support epidemiological policy interventions in the region. For the purpose of continuing this line of work, following is a list of possible extensions being considered:

\begin{itemize}
    \item Compare these results against actual serology or survey disease prevalence. Data collected from fieldwork could be fed to the algorithm in order to supervise the learning. 
    \item Differentiating rural antennas from urban ones. This is important as rural areas have conditions which are more vulnerable to the disease spread. \textit{T. Cruzi} transmission is favored by rural building materials and domestic animals contribute to complete the parasite's lifecycle. Antennas could be automatically tagged as rural by analyzing the differences between the spatial distribution of the antennas in each area. Another goal could be to identify precarious settlements within urban areas, with the help of census data sources.
    \item Seasonal migration analysis:  many seasonal workers might leave the endemic area for several months while still carrying the disease. Aanlyzing these migrations can give information on which communities have a high influx of people from the endemic zone.
    \item Add more regions to the analysis.
    \item Search for epidemiological data at a finer grain, such as specific historical infection cases. Splitting the endemic region according to the infection rate in different areas, or considering particular infections.
    \item Feature exploration, to search for correlations with being infected.
    \item Apply best fit model to Argentinian dataset. Provide that info to a Chagas risk model. Assuming that a high influx of individuals from epidemic regions is correlated with a higher risk in that area the algorithm could highlight these points. % revisar
\end{itemize}

\section{Acknowledgements}
Mundo Sano. Diego, Marcelo, Marcelo2