\section{Meta Section: Writing plan for this Chapter}

\textbf{Primero}, dar una introduccion a Machine Learning, que es, que hace, notacion, como se usa, supervised learning, clasificadores y algunos aspectos sobre el "cambio de paradigma" o "filosofia" o como quieran llamarlo. Comentar algo del foco en la data, el training error, predictive power, generalizaciones,etc . Ir llevando todo con un ejemplo sobre el Iris dataset y un logistic classifier.

\textbf{Despues} cross validation, variance, bias, errors, metrics, scores y similar.

\textbf{Finalizar} con 4/5 clasificadores usuales como: 
\begin{itemize}
	\item multinomial naive bayes (por su simplicidad y eficiencia computacional)
	\item "full" logistic regression (a diferencia de la intro anterior con la parte de regularizacion, con comentarios a sgd y su eficiencia, etc. )
	\item Random Forests
	\item Extension a gradient boosting
	\item Comentarios sobre Boltzmann machines ( sin adentrarnos mucho en redes neuronales ) y Bernoulli RestrictedBM.
\end{itemize}
