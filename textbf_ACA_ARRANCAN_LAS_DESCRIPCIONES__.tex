\textbf{ACA ARRANCAN LAS DESCRIPCIONES TECNICAS Y TEORICAS DE CADA CLASIFICADOR}

Notar que todo lo que esta en italica es tipo highlighter de textos de otros autores :)
%NOTAR QUE TO LO QUE ESTA EN ITALICA ES TIPO HIGHLIGHTER DE OTROS TEXTOS DE OTROS AUTORES}

\section{Classifier : Random Forests}

Random Forests are a special type of classifier in which a set of weaker learners are used together to build a stronger classifier. The idea is to have \textit{Decision Trees} as week learners and to ensemble a group of trees together.

\subsection{Decision Tree Learners}

Decision Trees will not be described in detail in this work, but a brief overview can be given as follows. 

The algorithm builds a tree in the graph-theory sense where each node has a rule based on set belonging. Each rule defines a linear (or similar) partition of the training set where rules are built from the sample's features. At each node we would have a decision to include or exclude a sample $s$ from that partition by checking if $X_i(s) \in U$ where $X_i$ might be any given feature of the data and $U$ is a subset of said feature's space.

For numerical features, $U$ will be of the form $(-\infty,c]$ where $c \in \mathbb{R}$ is any given number predefined by each rule. For categorical features, $U$ will be a subset of possible values of that feature.

A tree is generally grown in an iterative way from the top down \footnote{In this context the \textit{top} of a tree refers to the root of it.}. All samples in the training set would enter the tree at the root node and then travel down according to their fulfillment of each node's rule. 

The most common variations of this learner build rules at each nodes in a greedy fashion, where a metric is locally optimized in each node to decide on the best splitting decision. The metrics used to build each rule score, for the resulting sets, are the \textit{Gini impurity measure} and the \textit{entropy} or \textit{information gain} criterion. The former optimizes for misclassification error in the resulting sets, if all samples were to be tagged with one predicted target-label. The latter optimizes for information entropy, which is analogous to minimizing \textit{Kullback-Liebler divergence} of the resulting sets with respect to the original set. This partitioning continues iteratively until a predefined optimization or iteration threshold is met. 

To illustrate this method, an instance is show in figure \ref{rf-treeFigure}. This classification tree example is built for the two class problem of gender prediction using data from CDRs:
%[.{\textit{Woman}}]
\smallskip
\begin{figure}[h]\label{rf-treeFigure}
	\Tree[.{ $Calling\_Volume \leq 23$ } [.{$Province \in \{ San Luis, Chubut \} $} [.{$Time\_Weekend \geq 16$} [.{\textit{M}} ] [.{\textit{F}} ]  ]
	[.{$Calls\_Weekdays \leq 48$} 
	[.{ $Time\_Weekday \geq 17$} [.{\textit{M}} ] [.{\textit{F}} ]] [.{\textit{F}} ] ]  ]
	[.{$Calls\_Mondays \geq 2$} [.{$Province \in \{ Chubut, Cordoba \} $}  [.{\textit{M}} ] [.{\textit{F}} ] ]
	[.{\textit{M}}  ]]]
		
\end{figure}

\smallskip

%[.{\textit{M}} ] [.{\textit{F}} ]

Given a sample from training set $\mathrm{T}$, a decision tree would predict its target class by traveling the sample until a leaf node is reached. In this method hyper-parameters of the model include the length of the tree and the metric used to decide on the \textit{best} split. Parameters are limited to the features selected at each node along with the splitting rule threshold. Note that the construction of optimal binary decision trees is NP-Complete \cite{decisionTreesNP}, thus it is computationally prohibitive to fit all of these parameters at once. However, once a learner is fit, predicting targets for new samples is straightforward. A tree defines a partition of feature space in disjoint sets $A_1,...,A_J$ such that the sample's predicted output $\hat{y}$ is $c_j$ if the sample belongs to $A_j$. Here $c_j$ is one of the possible values taken by the target variable $y$ in the training set. And by the way trees were built, each $A_j$ is a hyper-rectangles in feature space.

For a more complete explanation of a decision tree for classification or regression problems, please refer to \cite{breiman-cart84}.
