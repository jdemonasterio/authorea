\textbf{ACA ARRANCAN LAS DESCRIPCIONES TECNICAS Y TEORICAS DE CADA CLASIFICADOR}

Notar que todo lo que esta en italica es tipo highlighter de textos de otros autores :)
%NOTAR QUE TO LO QUE ESTA EN ITALICA ES TIPO HIGHLIGHTER DE OTROS TEXTOS DE OTROS AUTORES}


\section{Classifier : Random Forests}


\section{Leo Breiman}


\textit{
Random forests are a combination of tree predictors
such that each tree depends on the values of a random
vector sampled independently and with the same
distribution for all trees in the forest.}

\textit{The generalization error for forests converges a.s. to a limit
as the number of trees in the forest becomes large}

\textit{The generalization error of a forest of tree classifiers depends on the strength of the individual trees in the
forest and the correlation between them. Using a random selection of features to split each node yields
error rates that are more with respect to noise.}

\textit{Internal estimates monitor error, strength, and correlation and these are used to show
the response to increasing the number of features used in the splitting. Internal estimates are also used to
measure variable importance.
d}



Random subspace: each tree is grown using a random selection of features.

Given $K$ number of trees, let $\Theta_k$ encode the parameters for the $k$-th tree and $h(\textbf{x},\Theta_k)$ the corresponding classifier. Let $N$ be the number of samples in the training set. The creation of a random forests involves an iterative procedure where at each $k$ step $\Theta_k$ is drawn from the same distribution but independently of the previous parameters $\Theta_1, \ ..., \ \Theta_{k-1}$ created at previous steps. 

Bagging: to grow each tree a random selection (without replacement) on the training set is used. 

Random split selection: choose a random split among the $K$ best splits. $\Theta$ will be encoded by a vector of randomly drawn integers from 1 to $M$ which is part of the model's hyperparameters.

Let $\{h_k(\textbf{x})\}_{i=1}^K$  \footnote{There is an abuse of notation by noting trees as $h_k(\textbf{x}$ and not $h(\textbf{x}, \Theta_k)$ } be a set of classifying trees and let $I$ denote the indicator function.  Define the margin function as

$$mg(\textbf{x},\textbf{y}) =  \frac{1}{K}   \sum_{k=1}^K I(h_k(\textbf{x}) = \textbf{y})  
- max_{j\neq \textbf{y}}\left(\frac{1}{K} \sum_{k=1}^K I(h_k(\textbf{x}) = j) \right) $$ \label{eq:rf-marginFun}

%\end{equation}
$ P_{\textbf{x}, \textbf{y} }(mg(\textbf{x}, \textbf{y}) < 0) $


The function measures, in average, how much do the trees vote for the correct class in comparison to all other classes and it can be shown that when $K$ is large then the prediction error converges almost surely to 

$$ P_{\textbf{x}, \textbf{y} } ( P_{\Theta} (h(\textbf{x}, \Theta) = \textbf{x}) - max_{j \neq \textbf{y}} (\textbf{x}, \textbf{y}) < 0) $$

\subsubsection{Proof}
The proof follows from seeing that given a training set, a parameter $\Theta$ and a class $j$ then 
$$\forall \textbf{x} \lim_{K\to\infty} \frac{1}{K} \sum_{k=1}^K I(h_k(\textbf{x}) = j) \ =   \ P_\Theta(h(\theta,\textbf{x}) = j) $$
 almost surely.

By looking at the nature of each tree, $\{\textbf{x} / h_k(\textbf{x}, \Theta) = j \}$ denotes a union of hyper-rectangles partitioning feature space. And given the finite size of the training set, there can only be a finite set of these unions. Let $S_1, ..., S_K$ be an indexation of these unions and define $\phi(\Theta) = k $ if $\{\textbf{x} / h(\textbf{x}, \Theta) = j \} = S_k$. 

We denote by $N_k$ the number of times that $\phi(\Theta_n) =k $, where $n \in {1...N}$ and $N$ is the total of trials.

It is immediate that 

$$ \frac{1}{N} \sum_{n=1}^N I(h_n(\textbf{x}) = j) \ = \  \frac{1}{N} \sum_{k=1}^K N_k I(\textbf{x} \in S_k)  $$

and that there is a convergence almost everywhere of $$ \frac{N_k}{N} = \sum_{n=1}^N  I(\phi(\Theta_n) = k)  \xrightarrow[N \to \infty]   P_{\Theta}(\phi(\Theta)= k)$$. 

If we let $C = $ $\bigcup\limits_{k=1}^{K} C_{k}$ where each $C_k$ are zero-measured sets representing the points where the sequence is not converging. We will finally have that  outside of $C$, 

$$ \frac{1}{N} \sum_{n=1}^N I(h_n(\textbf{x}) = j) \xrightarrow[N \to \infty]{} \sum_k^K    P_{\Theta}(\phi(\Theta)= k) I(\textbf{x} =j ) \ = \ P_{\Theta}(h(\textbf{x}, \Theta) = j)  $$ 



\begin{lemma}
Given two line segments whose lengths are $a$ and $b$ respectively there is a 
real number $r$ such that $b=ra$.
\end{lemma}

For the case of a random forest classifier, the margin function will be defined as

$$mr(\textbf{x},\textbf{y}) =  P_Theta(h(\textbf{x}) = \textbf{y})  
- max_{j\neq \textbf{y}}\left(P_{Theta}(h_k(\textbf{x}) = j) \right) $$ \label{eq:rf-marginFunRf}

and the strength will be defined as 

$$s =  \Expect_{\textbf{x},\textbf{y}} \left[ mr(\textbf{x},\textbf{y} ) \right] $$. \label{eq:rf-strength}

Assuming that $s \geq 0$ we have that the generalization error $PE \leq var(mr)/sˆ2$ by Chebyshev's inequality. 

Define $\hat{j}(\textbf{x},\textbf{y}) $ 
\b{x}


