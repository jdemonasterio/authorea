\textbf{ACA ARRANCAN LAS DESCRIPCIONES TECNICAS Y TEORICAS DE CADA CLASIFICADOR}

\section{Classifier : Random Forests}

\section{Hastie Tibshiranie Friedman}



Chagas disease is a tropical parasitic epidemic of global reach, spread mostly across 17 Latin American countries. The World Health Organization (WHO) estimates more than six million infected people worldwide~\cite{who2016}. The disease is caused by the \textit{Trypanosoma cruzi} parasite. Most transmissions occur in the endemic regions in America, where \textit{T. cruzi} is spread to humans by the \textit{Triatomine} insect family (also called "kissing bug", and known by many local names such as "vinchuca" in Argentina, Bolivia, Chile and Paraguay, and "chinche" in Central America). In recent years and due to globalization and migrations, the disease has become a health issue in other continents, particularly in countries who receive Latin American immigrants such as Spain and the United States~\cite{schmunis2010chagas}, making it a global health problem.

A crucial characteristic of the infection is that it may last 10 to 30 years in an individual without being detected~\cite{rassi2012american}, which greatly complicates effective detection and treatment. In effect, about 70\% of individuals with chronic Chagas disease will never develop symptoms, whereas the remaining 30\% will develop life-threatening heart and/or digestive disorders.
Long-term human mobility, particularly seasonal and permanent rural-urban migration, thus plays a key role in the spread of the epidemic~\cite{briceno2009chagas}. Relevant routes of transmission also include blood transfusion and congenital transmission, with an estimated 14,000 newborns infected each year in the Americas~\cite{OPS2006chagas}.
% \begin{comment}  en el drive estan las ppt del min salud \end{comment}. 
The spatial dissemination of a congenitally transmitted disease sidesteps the available measures to control risk groups, and shows that individuals who have not been exposed to the disease vector should also be included in detection campaigns.


\section{Leo Breiman}

In this work we discuss the use of Call Detail Records (CDRs) for the analysis of mobility patterns and the detection of possible risk zones of Chagas disease in two Latin America countries. This project was performed in collaboration with the \textit{Mundo Sano} Foundation, who provided key health expertise on the subject. We generate predictions of population movements between different regions, providing a proxy for the epidemic spread. Our objective is to show that geolocalized call records are rich in social and individual information, which can be used to determine whether an individual has lived in an epidemic area. We present two case studies, in Argentina and in Mexico, using data provided by mobile phone companies from each country. A discussion of how mobile data was processed is included. 

Mobile phone records contain information about the movements of large subsets of the population of a country, and make them very useful to understand the spreading dynamics of infectious diseases. They have been used to understand the diffusion of malaria in Kenya~\cite{wesolowski2012quantifying} and in Ivory Coast~\cite{enns2013human}, including the refining of infection models~\cite{chunara2013large}. 
The referenced works on Ivory Coast were performed using the D4D (Data for Development) challenge datasets released in 2013. Additional studies based on the Ivory Coast dataset are reviewed in \cite{naboulsi2015mobile}.
However, to the best of our knowledge, this is the first work that leverages mobile phone data to better understand the diffusion of the Chagas disease.


\section{scikit-docs}

\textit{The goal of ensemble methods is to combine the predictions of several base estimators built with a given learning algorithm in order to improve generalizability / robustness over a single estimator.
Two families of ensemble methods are usually distinguished:
• In averaging methods, the driving principle is to build several estimators independently and then to average their predictions. On average, the combined estimator is usually better than any of the single base estimator because its variance is reduced.
Here: Forests of randomized trees, ...
• By contrast, in boosting methods, base estimators are built sequentially and one tries to reduce the bias of the combined estimator. The motivation is to combine several weak models to produce a powerful ensemble.

As they provide a way to reduce overfitting, bagging methods work best with strong and complex models (e.g., fully developed decision trees), in contrast with boosting methods which usually work best with weak models (e.g., shallow decision trees).

Bagging methods come in many flavours but mostly differ from each other by the way they draw random subsets of the training set. As a general rule, randomness is applied either to the ‘rows’ or ‘columns’ of the dataset i.e. samples or features:    
• When random subsets of the dataset are drawn as random subsets of the samples, then this algorithm is known as Pasting [L. Breiman, “Pasting small votes for classification in large databases and on-line”, Machine Learning, 36(1), 85-103, 1999.].
• When samples are drawn with replacement, then the method is known as Bagging [L. Breiman, “Bagging predictors”, Machine Learning, 24(2), 123-140, 1996.].
• When random subsets of the dataset are drawn as random subsets of the features, then the method is known as Random Subspaces [H1998].
• Finally, when base estimators are built on subsets of both samples and features, then the method is known as
Random Patches [LG2012].
L. Breiman, “Bagging predictors”, Machine Learning, 24(2), 123-140, 1996.

Random forests use a perturb-and-combine technique [Breiman, “Arcing Classifiers”, Annals of Statistics 1998.] specifically designed for trees. This means a diverse set of classifiers is created by introducing randomness in the classifier construction. The prediction of the ensemble is given as the averaged prediction of the individual classifiers.

each tree in the ensemble is built from a sample drawn with replacement (i.e., a bootstrap sample) from the training set. In addition, when splitting a node during the construction of the tree, the split that is chosen is no longer the best split among all features. Instead, the split that is picked is the best split among a random subset of the features. As a result of this randomness, the bias of the forest usually slightly increases (with respect to the bias of a single non-random tree) but, due to averaging, its variance also decreases, usually more than compensating for the increase in bias, hence yielding an overall better model. [Breiman, “Random Forests”, Machine Learning, 45(1), 5-32, 2001.]

The main parameters to adjust when using these methods is n_estimators and max_features. The former is the number of trees in the forest. The larger the better, but also the longer it will take to compute. In addition, note that results will stop getting significantly better beyond a critical number of trees. The latter is the size of the random subsets of features to consider when splitting a node. The lower the greater the reduction of variance, but also the greater the increase in bias. Empirical good default values are max_features=n_features for regression problems, and max_features=sqrt(n_features) for classification tasks (where n_features is the number of features in the data). Good results are often achieved when setting max_depth=None in combination with min_samples_split=1 (i.e., when fully developing the trees). Bear in mind though that these values are usually not optimal, and might result in models that consume a lot of ram. The best parameter values should always be cross-validated. In addition, note that in random forests, bootstrap samples are used by default (bootstrap=True) while the default strategy for extra-trees is to use the whole dataset (bootstrap=False). When using bootstrap
sampling the generalization error can be estimated on the left out or out-of-bag samples. This can be enabled by setting oob_score=True.
}



\textit{

}





