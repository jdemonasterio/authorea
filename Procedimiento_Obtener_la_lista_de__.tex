% Procedimiento:
	% Obtener la lista de antenas de GC
	% Determinar la casa de los usuarios
	% Determinar, para cada usuario, si se comunicó con un habitante de GC
	% Determinar las agregaciones por antena
	% Visualizar (agrego alfa, min_volume y zoom en regiones)

\section{Procedure}
\subsection{Home detection}
    The first step is to determine the area in which each user lives. The granularity of the geolocated data we have is at the antenna level, so the objective for this part is to match each user $u$ with its \textit{home antenna} $H_u$.
    
    We predict $H_u$ as the antenna in which user $u$ spends most of the time during weekday nights. This, according to our categorization of types of days of the week, corresponds to Monday to Thursday nights, from 8pm to 6am of the following day.
    % Speak about the min number of calls needed to say that we can know the home.
    
    Note that users for which the inferred home antenna is located in \textit{Gran Chaco} (the risk area) will be considered the set of \textit{residents of Gran Chaco}.

\subsection{Detection of vulnerable users}
    Given the set of inhabitants of the risk area, we want to find those with a high communication with the residents of the endemic zone.
    
    We get the list of calls for each user, and, using this data, we determine the set of neighbors in the social graph. Then, we tag the neighbors of any resident of the endemic zone as potentially vulnerable.
	
		\begin{block}{Antenna aggregation}
		\begin{itemize}
			\item  We aggregated vulnerable users and total users (residents) per antenna.
			%Agregamos usuarios vulnerables y usuarios totales (habitantes) para cada antena.
			
			\item We also aggregated the total volume of outgoing calls from every antenna and from these we extracted
			every call that had a user whose home is in the \textit{Gran Chaco} area as a receiver (\textit{vulnerable calls}).
			%Tambien agregamos el volumen total de llamados salientes de cada antena,
			%y de estos, los que tenían un habitante de Gran Chaco como destinatario (\textit{llamados vulnerables}).
			
			%\bigskip
			%Por lo tanto se obtuvieron, para cada antena, cuatro indicadores:
			%\begin{itemize}
			%	\item Cantidad de usuarios habitantes
			%	\item Cantidad de usuarios vulnerables entre los habitantes.
			%	\item Volumen de llamados salientes
			%	\item Volumen de llamados vulnerables salientes.
			%\end{itemize}
		\end{itemize}
		
	\end{block}
\subsection{Heatmap generation}

		We generated heat maps from the processed data where each cell was represented by its communications with
		the ecoregion.
		%A partir de los datos obtenidos, generamos mapas de calor que representan las comunicaciones de la celda con el Gran Chaco.
		
		We generated a circle for each cell where:
		%Generamos un c\'irculo por cada celda donde:
		\begin{itemize}
			\item the \textbf{area} depends on the on the volume of use.
			%el \textbf{\'area} depende de la cantidad de usuarios (habitantes),
			\item the \textbf{color} corresponds to the percentage of use which is vulnerable in that antenna.
			%el \textbf{color} corresponde al porcentaje de usuarios vulnerables que viven en esa antena.
		\end{itemize}
		
		\begin{block}{Antenna Filter}
		We built two filtering parameters $\beta$ and $min\_volumen$ which will control the antennas to be plotted.
		Every antenna will be plotted if:
		% Notar que en ningún lado antes de esto dijimos que íbamos a pintar un mapa de Argentina con las antenas.
		%Dos par\'ametros $\beta$ y $min\_volumen$  filtran la lista de antenas a ser visualizadas.
		%Cada antena se grafica si:
		\begin{itemize}
			\item the percentage of vulnerable users/calls is bigger than  $\beta$.
			%el porcentaje de usuarios vulnerables es mayor que $\beta$,
			\item the volume of vulnerable users/calls is bigger than $min\_volumen$. 
			%el volumen de uso vulnerable es mayor que $min\_volumen$ de llamados o usuarios.
		\end{itemize}
		
		%
		%\bigskip
		%Por lo tanto, el modelo permitiría ver las antenas donde existen mayores lazos familiares con el Gran Chaco.	% Creo que 'familiares' es muy fuerte
		
		%\bigskip
		%A\'un con los par\'ametros anteriores, fue necesario separar el filtrado por distintas regiones:
		%
		%\smallskip
		%Una antena de baja proporción a nivel nacional ($\beta=10\%$) puede ser llamativa, por ejemplo, en Tierra del Fuego.
		%%aca se inserta imagen x ejemplo?? %No, la imagen es demasiado grande me parece. Lo podemos discutir.
		
	\end{block}