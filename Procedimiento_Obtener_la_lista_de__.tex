% Procedimiento:
	% Obtener la lista de antenas de GC
	% Determinar la casa de los usuarios
	% Determinar, para cada usuario, si se comunicó con un habitante de GC
	% Determinar las agregaciones por antena
	% Visualizar (agrego alfa, min_volume y zoom en regiones)

\section{Procedure}
\subsection{Home detection}
	\begin{block}{Home Prediction}
		\begin{itemize}
			\item As a first step, we determined each user's residence antenna.
			%Como primer paso, determinamos para cada usuario, su lugar de residencia.
			%No es la ubicación de su casa exactamente, pero es una buena aproximación. % Aclarar más esto? Aclararlo menos? %creo que esta bien
			
			\item This was chosen to be the most used frequently used antenna, considering only calls made in week evenings.
			%La antena elegida como hogar es la más frecuentemente usada, 
			%considerando llamados nocturnos en días de semana.
			
			\item The hypothesis: most of people are at home on any given weeknight. 
			%La hipótesis: la mayoría de los días, las personas se encuentran en sus casas durante la noche.

			%Así, cada antena queda asociada a un conjunto de usuarios: sus \textit{habitantes}.
			\item Note: users for which the inferred home antenna is located in \textit{Gran Chaco} (the risk area) will
			be considered the set of \textit{residents of Gran Chaco}. 
			%Nota: Los usuarios cuya antena inferida pertenece al Gran Chaco (la zona de riesgo)
			%se consideran el conjunto de \textit{habitantes de Gran Chaco}.
			
		\end{itemize}
	\end{block}

\subsection{Detection of vulnerable users}
	\begin{block}{Vulnerable users}
		\begin{itemize}
			\item For every user, we listed all of the call receivers in a given month.
			%Para cada usuario, obtuvimos la lista de todos los usuarios con los que se comunicó. %ojo que esto podria cambiar en versiones posteriores cuando cambie la definicion de 'vulnerable'
			
			\item If a given user communicated with the \textit{Gran Chaco}, we considered him to have higher risk
			of having Chagas and we tagged him as potentially \textit{vulnerable}.
			%Si un usuario tuvo comunicaciones con la zona endémica (Gran Chaco), se considera que tiene mayor riesgo de tener Mal de Chagas, y lo marcamos como potentialmente \textit{vulnerable}.
		\end{itemize}
	\end{block}
	
		\begin{block}{Antenna aggregation}
		\begin{itemize}
			\item  We aggregated vulnerable users and total users (residents) per antenna.
			%Agregamos usuarios vulnerables y usuarios totales (habitantes) para cada antena.
			
			\item We also aggregated the total volume of outgoing calls from every antenna and from these we extracted
			every call that had a user whose home is in the \textit{Gran Chaco} area as a receiver (\textit{vulnerable calls}).
			%Tambien agregamos el volumen total de llamados salientes de cada antena,
			%y de estos, los que tenían un habitante de Gran Chaco como destinatario (\textit{llamados vulnerables}).
			
			%\bigskip
			%Por lo tanto se obtuvieron, para cada antena, cuatro indicadores:
			%\begin{itemize}
			%	\item Cantidad de usuarios habitantes
			%	\item Cantidad de usuarios vulnerables entre los habitantes.
			%	\item Volumen de llamados salientes
			%	\item Volumen de llamados vulnerables salientes.
			%\end{itemize}
		\end{itemize}
		
	\end{block}
\subsection{Heatmap generation}

		We generated heat maps from the processed data where each cell was represented by its communications with
		the ecoregion.
		%A partir de los datos obtenidos, generamos mapas de calor que representan las comunicaciones de la celda con el Gran Chaco.
		
		We generated a circle for each cell where:
		%Generamos un c\'irculo por cada celda donde:
		\begin{itemize}
			\item the \textbf{area} depends on the on the volume of use.
			%el \textbf{\'area} depende de la cantidad de usuarios (habitantes),
			\item the \textbf{color} corresponds to the percentage of use which is vulnerable in that antenna.
			%el \textbf{color} corresponde al porcentaje de usuarios vulnerables que viven en esa antena.
		\end{itemize}
		
		\begin{block}{Antenna Filter}
		We built two filtering parameters $\beta$ and $min\_volumen$ which will control the antennas to be plotted.
		Every antenna will be plotted if:
		% Notar que en ningún lado antes de esto dijimos que íbamos a pintar un mapa de Argentina con las antenas.
		%Dos par\'ametros $\beta$ y $min\_volumen$  filtran la lista de antenas a ser visualizadas.
		%Cada antena se grafica si:
		\begin{itemize}
			\item the percentage of vulnerable users/calls is bigger than  $\beta$.
			%el porcentaje de usuarios vulnerables es mayor que $\beta$,
			\item the volume of vulnerable users/calls is bigger than $min\_volumen$. 
			%el volumen de uso vulnerable es mayor que $min\_volumen$ de llamados o usuarios.
		\end{itemize}
		
		%
		%\bigskip
		%Por lo tanto, el modelo permitiría ver las antenas donde existen mayores lazos familiares con el Gran Chaco.	% Creo que 'familiares' es muy fuerte
		
		%\bigskip
		%A\'un con los par\'ametros anteriores, fue necesario separar el filtrado por distintas regiones:
		%
		%\smallskip
		%Una antena de baja proporción a nivel nacional ($\beta=10\%$) puede ser llamativa, por ejemplo, en Tierra del Fuego.
		%%aca se inserta imagen x ejemplo?? %No, la imagen es demasiado grande me parece. Lo podemos discutir.
		
	\end{block}