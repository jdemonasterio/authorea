\section{Conclusions}

The heatmaps expose a 'temperature' descent from the core regions outwards. The heat is concentrated in the ecoregion and gradually descends as we move further away. This expected behavior could be explained by the fact that calls are in general of a local nature and limited to 3 or 4 antennas used per user. 

A more surprising event is the finding of communities atypical to their neighbouring region. They stand out for their strong communication ties with the studied region, showing significantly higher links of vulnerable communication. The detection of these antennas through the visualizations is of great value to health campaign managers. Tools that target specific areas help to prioritize resources and calls to action more effectively.

The results presented in this work show that it is possible to explore CDRs as a mean to tag human mobility. Combining social and geolocated information, the data at hand has been given a new innovative use different to the end for which the data was created (billing).

Epidemic counter-measures nowadays include setting national surveillance systems, vector-centered policy interventions and individual screenings of people. These measures require costly infrastructures to set up and be run. However, systems built on top of existing mobile networks would demand lower costs, taking advantage of the already available infrastructure. The potential value these results could add to health research is hereby exposed.

Finally, the results stand as a proof of concept which can be extended to other countries or to diseases with similar characteristics.


