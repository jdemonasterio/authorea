\documentclass[xcolor=x11names]{beamer}
  \mode<presentation> {
    \usetheme{Frankfurt}
  }

\usepackage{times}
\usepackage{amsmath,amssymb}
\usepackage[english]{babel}
\usepackage[utf8]{inputenc}
% \usepackage[latin1]{inputenc}
\usepackage{array}
% % !BIB TS-program = biber
% !BIB program = biber
%===============================================================================
%          File: preamble.tex
%        Author: Pedro Ferrari
%       Created: 08 Feb 2017
% Last Modified: 08 Feb 2017
%   Description: Preamble for Thesis File
%===============================================================================
%---------------------------------------+
% Source code, programming and patching |
%---------------------------------------+
\usepackage{etoolbox}    % Toolbox of programming tools
\usepackage{xpatch}      % Extension of etoolbox patching commands

%--------------------------------------+
% Language, hyphenation, encoding, etc |
%--------------------------------------+
\usepackage[english]{babel}
\usepackage{lmodern}             % Use Latin Modern fonts
\usepackage[T1]{fontenc}         % Better output when a diacritic/accent is used
\usepackage[utf8]{inputenc}      % Allows to input accented characters
\usepackage{textcomp}            % Avoid conflicts with siunitx and microtype
\usepackage{microtype}           % Improves justification and typography
\usepackage[svgnames]{xcolor}    % Svgnames option loads navy (blue) colour

%---------------------------------------------+
% Page style: titles, margins, footnotes, etc |
%---------------------------------------------+
% A4 page layout:
\usepackage[width=14cm,left=3.5cm,marginparwidth=3cm,marginparsep=0.35cm,
height=21cm,top=3.7cm,headsep=1cm,footskip=1.1cm]{geometry}

\usepackage[pagestyles,outermarks]{titlesec}  % Customize titles and headers
\newpagestyle{main}[\scshape]{%
  \headrule
  \sethead
  [\thepage][][\chaptertitlename\space\thechapter. \chaptertitle]
  {\ifthesection{\thesection\space\,\sectiontitle}
  {\chaptertitlename\space\thechapter. \chaptertitle}}{}{\thepage}
}
\newpagestyle{special}[\scshape]{%
  \headrule
  \sethead
  [\thepage][][\chaptertitle]
  {\ifthesection{\sectiontitle}{\chaptertitle}}{}{\thepage}
}
% The following pagestyle is needed because titlesec isn't compatible with
% refsegment=chapter
\newpagestyle{bibatend}[\scshape]{
  \headrule
  \sethead
  [\thepage][][\chaptertitle]
  {\sectiontitle}{}{\thepage}
}
\pagestyle{special}
\appto{\mainmatter}{\pagestyle{main}}
\appto{\backmatter}{\pagestyle{bibatend}}
\appto{\printindex}{\pagestyle{special}}

% Use empty page style instead of plain in parts and chapters title pages
\patchcmd{\part}{plain}{empty}{}{}
\patchcmd{\chapter}{plain}{empty}{}{}

\usepackage{emptypage}  % Empty blank pages created by \cleardoublepage

% Change chapter heading style to match titlepage
\titleformat{\chapter}[display]
{\bfseries\filcenter}
{\titlerule[1.5pt]\vspace{4ex}%
\LARGE{\chaptertitlename\space\thechapter}}{0.5cm}{\huge}
[\vspace{2ex}{\titlerule[1.5pt]}\vspace{0.3cm}]
% Do the same for unnumbered chapters (TOC, preface, etc)
\titleformat{name=\chapter,numberless}[display]
{\bfseries\filcenter}
{\titlerule[1.5pt]\vspace{4ex}}{0.5cm}{\huge}
[\vspace{2ex}{\titlerule[1.5pt]}\vspace{0.3cm}]

\usepackage[stable,multiple]{footmisc}  % Customizations of footnotes
\renewcommand*{\footnoterule}{\vspace*{0.3cm}\hrule width 2.5cm\vspace*{0.3cm}}
\makeatletter
  \renewcommand\@makefntext[1]{
  \setlength{\parindent}{15pt}\mbox{\@thefnmark.\space}{#1}}
\makeatother

%-------------------------------+
% Math symbols and environments |
%-------------------------------+
\usepackage{amsmath}               % Load new math environments
\numberwithin{equation}{section}
\usepackage{amssymb}               % Defines most math symbols (such as \mathbb)
\usepackage{mathtools}             % Extension and bug fixes for amsmath package
\usepackage{mathrsfs}              % Math script like font
\usepackage{breqn}                 % Automatic line breaking of math expressions
\renewcommand*{\intlimits}{\displaylimits}  % Fix breqn clash with intlimits


%---------------------+
% Floats and captions |
%---------------------+
\usepackage{graphicx}           % To include graphics files
%\graphicspath{{/Users/Pedro/OneDrive/programming/Latex/logos/}{figures/}{tables/}}
\usepackage{pdflscape}
\usepackage[font=small,labelfont=bf]{caption}
\captionsetup*[figure]{format=plain,justification=centerlast,labelsep=quad}
\captionsetup*[table]{justification=centering,labelsep=newline}
\numberwithin{figure}{section}
\numberwithin{table}{section}

% Use subcaption for subfigures (to work properly with hyperref)
\usepackage{subcaption}
\captionsetup*[subfigure]{subrefformat=simple,labelformat=simple}
\renewcommand*{\thesubfigure}{(\alph{subfigure})}

% Further modifications of float layout
\usepackage[captionskip=5pt]{floatrow}  % We set caption skip here
\floatsetup[table]{style=Plaintop,font=small,footnoterule=none,footskip=2.5pt}
 
% Add ability to show rotated figures in horizontal or vertical alignment - sidewaysfigure
\usepackage{rotating}

%----------------+
% Table packages |
%----------------+
\usepackage{array}          % Flexible column formatting
% \usepackage{spreadtab}  % Spreadsheet features
\usepackage{multirow}       % Allows table cells that span more than one row
\usepackage{booktabs}       % Enhance quality of tables
\setlength{\heavyrulewidth}{1pt}

% \usepackage{longtable}        % Allows to break tables through pages
% \floatsetup[longtable]{margins=centering,LTcapwidth=table}

% to build master table
\newcommand{\ra}[1]{\renewcommand{\arraystretch}{#1}}

%---------------------------------------------------------+
% Miscellaneous packages: lists, setspace, todonotes, etc |
%---------------------------------------------------------+
\usepackage[shortlabels,inline]{enumitem}   % Customize lists
\setlist[itemize,1]{label=$\bullet$}
\setlist[itemize,2]{label=\footnotesize{$\blacktriangleright$}}
\setlist[itemize,3]{label=\tiny{$\blacksquare$}}
\setlist[itemize,4]{label=\bfseries{\large{--}}}
% \setlist[enumerate,2]{label=\emph{\alph*})}
\newlist{steps}{enumerate}{1}               % List of steps to be used in proofs
\setlist[steps,1]{leftmargin=*,label=\textit{Step \arabic*.},ref=\arabic*}

\usepackage{setspace}  % Commands for double and one-and-a-half spacing
\setstretch{1.2}       % 1.2 spacing

% \usepackage{listings}  % Useful for inserting code (no unicode support)
% \lstset{basicstyle=\small\ttfamily}

% \usepackage[colorinlistoftodos,textsize=small,figheight=5cm,
% figwidth=10cm,color=red!85]{todonotes}

% \usepackage{lipsum}    % Dummy text generator

\usepackage{algorithm2e} % to write pseudo-code algorithms
\usepackage{qtree} % To draw trees
\usepackage{todonotes} % To draw trees

%----------------------------------+
% Appendix, bibliography and index |
%----------------------------------+
% Solve bad interaction between titlesec and \
\preto{\appendix}{\cleardoublepage}

% titletoc will prefix the word `Appendix` for each appx chapter.
\usepackage[titletoc,page]{appendix}
% this might insert blank pages here and there but now compilation works correctly with biber
\usepackage[style=american]{csquotes}  % Language sensitive quotation facilities
\usepackage[style=authoryear-comp,backref=true,refsection=chapter,backend=biber]{biblatex}


% \usepackage[
% 	backend=biber,
% 	sortlocale=us_EN,
% 	natbib=true,
% 	style=authoryear-comp,backref=true,refsection=chapter,
% 	url=false,
% 	doi=true,
% 	eprint=false
% 	]{biblatex}

\addbibresource{biblio.bib}

% Bibliography format
\usepackage{mybibformat} % Modifications to authoryear-comp style and hyperlinks
\setlength{\bibitemsep}{0.1cm}

\usepackage{imakeidx}  % Creation and formatting of indexes
\indexsetup{level=\chapter,firstpagestyle=empty,othercode=\small}
\makeindex[title=Alphabetical Index]

%------------------------------------------------------+
% Hyperlinks, bookmarks, theorems and cross-references |
%------------------------------------------------------+
\usepackage[hyperfootnotes=false]{hyperref}
\hypersetup{colorlinks=true, allcolors=Navy, linktoc=page,
pdfstartview={XYZ null null 1}, pdfcreator={Vim LaTeX},
pdfsubject={Machine Learning},
pdftitle={Math thesis},
pdfauthor={Juan Mateo De Monasterio},
pdfkeywords={machine learning}
}
\usepackage[numbered,open,openlevel=1]{bookmark}

\usepackage{amsthm,amsmath}        % We load theorem environments here to avoid warnings

\usepackage[noabbrev,capitalise]{cleveref}

%-----------------------------------------------------------+
% Comment out the math equations, figurest,etc                |
% This will help us spell checking the doc with online tools |
%-----------------------------------------------------------+

% These allow one to compile the pdf without any figures, equations
% etc. which will let us copy paste the output for spell checking.

%\usepackage{comment}
%\excludecomment{figure}
%\let\endfigure\relax
%\excludecomment{equation}
%\let\endequation\relax

%\excludecomment{lstlisting}
%\let\endlstlisting\relax



%------------------------------------+
% Definition of theorem environments |
%------------------------------------+
% Declare theorem styles that remove final dot and use bold font for notes
\newtheoremstyle{plaindotless}{\topsep}{\topsep}{\itshape}{0pt}{\bfseries}{}%
{5pt plus 1pt minus 1pt}{\thmname{#1}\thmnumber{ #2}\bfseries{\thmnote{ (#3)}}}
\newtheoremstyle{definitiondotless}{\topsep}{\topsep}{\normalfont}{0pt}%
{\bfseries}{}{5pt plus 1pt minus 1pt}%
{\thmname{#1}\thmnumber{ #2}\bfseries{\thmnote{ (#3)}}}
\newtheoremstyle{remarkdotless}{0.5\topsep}{0.5\topsep}{\normalfont}{0pt}%
{\itshape}{}{5pt plus 1pt minus 1pt}%
{\thmname{#1}\normalfont\thmnumber{ #2}\itshape{\thmnote{ (#3)}}}

% Define style dependent environments and number them consecutively per section
\theoremstyle{plaindotless}
\newtheorem{theorem}{Theorem}[section]
\newtheorem*{theorem*}{Theorem.}
\newtheorem{proposition}[theorem]{Proposition}
\newtheorem*{proposition*}{Proposition.}
\newtheorem{lemma}[theorem]{Lemma}
\newtheorem*{lemma*}{Lemma.}
\newtheorem{corollary}[theorem]{Corollary}
\newtheorem*{corollary*}{Corollary.}

\theoremstyle{definitiondotless}
\newtheorem{definition}[theorem]{Definition}
\newtheorem*{definition*}{Definition.}
\newtheorem{examplex}[theorem]{Example}
\newtheorem*{examplestarred}{Example.}
\newtheorem*{continuedex}{Example \continuedexref\space Continued.}
\newtheorem{exercise}[theorem]{Exercise}
\newtheorem*{exercise*}{Exercise.}
\newtheorem*{solution*}{Solution.}
\newtheorem{problem}{Problem}

\theoremstyle{remarkdotless}
\newtheorem{remark}[theorem]{Remark}
\newtheorem*{remark*}{Remark.}
\newtheorem*{notation*}{Notation.}

% Define numbered, unnumbered and continued examples with triangle end mark
\newcommand{\myqedsymbol}{\ensuremath{\triangle}}

\newenvironment{example}
  {\pushQED{\qed} \renewcommand{\qedsymbol}{\myqedsymbol}\examplex}
  {\popQED\endexamplex}

\newenvironment{example*}
  {\pushQED{\qed}\renewcommand{\qedsymbol}{\myqedsymbol}\examplestarred}
  {\popQED\endexamplestarred}

\newenvironment{examcont}[1]
  {\pushQED{\qed}\renewcommand{\qedsymbol}{\myqedsymbol}%
    \newcommand{\continuedexref}{\ref*{#1}}\continuedex}
  {\popQED\endcontinuedex}


%-----------------------------------------------+
% Cross-references settings (cleveref settings) |
%-----------------------------------------------+

\crefname{exercise}{Exercise}{Exercises}
\crefname{enumerate}{Enumeration}{Enumerations}
\crefname{stepsi}{Step}{Steps}
\crefname{problem}{Problem}{Problems}

% Uncommenting this brings problems when referrencing equations.

%\crefname{equation}{}{}
%\crefformat{equation}{#2(#1)#3}
%\crefrangeformat{equation}{#3(#1)#4 to #5(#2)#6}
%\crefmultiformat{equation}{#2(#1)#3}{ and #2(#1)#3}{, #2(#1)#3}{ and #2(#1)#3}
%\crefrangemultiformat{equation}{#3(#1)#4 to #5(#2)#6}{ and #3(#1)#4 to #5(#2)#6}%
%{, #3(#1)#4 to #5(#2)#6}{ and #3(#1)#4 to #5(#2)#6}

%----------------------------------------------+
% Half-title, titlepage and copyright settings |
%----------------------------------------------+

\newcommand*{\halftitlepg}{%
  \begingroup
    \begin{center}
      \textbf{\huge{Math Thesis}}
    \end{center}
  \endgroup
  \thispagestyle{empty}\cleardoublepage
}

\newcommand*{\titlepg}{%
  \begingroup
    \vspace{0.01\textheight}
    \begin{center}
      \textbf{\LARGE{Universidad de Buenos Aires}}\\
      \vspace{0.02\textheight}
      \textbf{\Large{Facultad de Ciencias Exactas}}\\
      \vspace{0.3\textheight}
      \rule{\textwidth}{1.5pt}\par
      \vspace{\baselineskip}

% {\bfseries\Huge{Exploring Latinoamerican chagasic migrations with machine learning }\par
 {\bfseries\Huge{Exploring Migrations and Spread of Chagas Disease in Latin America with Machine Learning}\par

% 	\bigskip\Large{Explorando migraciones chagásicas en latinoamérica con aprendizaje automático}}\\
 	\bigskip\Large{Explorando migraciones y difusión del mal de Chagas en América Latina con aprendizaje automático}}\\

      \vspace{\baselineskip}
      \rule{\textwidth}{1.5pt}\par
      \vfill
      \textsc{\huge{Juan Mateo De Monasterio}}
      \vfill
      \textbf{\Large{\today}}
    \end{center}
  \endgroup
  \thispagestyle{empty}\clearpage
}

\newcommand*{\copyrightpg}{%
  \begingroup
    \footnotesize
    \parindent 0pt
    \null
    \vfill
    \textcopyright{} 2017 Juan Mateo De Monasterio. All rights reserved.\par
    \vspace{\baselineskip}
    This document is free; you can redistribute it and/or modify it under the
    terms of the GNU General Public License as published by the Free Software
    Foundation; either version 2 of the License, or (at your choice) any later
    version.\par
    \vspace{\baselineskip}
    This document was typeset in Latin Modern font using \LaTeX.\par
  \endgroup
  \thispagestyle{empty}\clearpage
}

\newcommand*{\dedication}{%
  \begingroup
    \vspace*{0.3\textheight}
    \begin{center}
      \emph{\large{To all.}}
      \end{center}
    \endgroup
%  \thispagestyle{empty}\cleardoublepage
  \thispagestyle{empty}\clearpage
}

%-------------------+
% Table of contents |
%-------------------+
% Add bookmark for table of contents and increase spacing of items
% \preto{\tableofcontents}{\cleardoublepage\pdfbookmark[0]{\contentsname}{toc}%  \setstretch{1.1}}
\preto{\tableofcontents}{\clearpage\pdfbookmark[0]{\contentsname}{toc}%
  \setstretch{1.1}}
\appto{\tableofcontents}{\singlespacing}

%---------------------------------------------------+
% (Re)Definition of new commands and math operators |
%---------------------------------------------------+
% Numbers
\DeclareMathOperator{\N}{\mathbb{N}}
\DeclareMathOperator{\Z}{\mathbb{Z}}
\DeclareMathOperator{\Q}{\mathbb{Q}}
\DeclareMathOperator{\R}{\mathbb{R}}
% Probability
\DeclareMathOperator{\E}{\mathbb{E}}
\DeclareMathOperator{\Expect}{\mathbb{E}}
\DeclareMathOperator{\Var}{\mathrm{Var}}
\DeclareMathOperator{\Cov}{\mathrm{Cov}}
% Delimiters
\DeclarePairedDelimiter{\abs}{\lvert}{\rvert}
\DeclarePairedDelimiter{\norm}{\lvert\lvert}{\rvert\rvert}
% Miscellaneous
\renewcommand{\d}{\ensuremath{\operatorname{d}\!}}  % Differential
\renewcommand{\L}{\ensuremath{\operatorname{\mathcal{L}}}}  % Lagrangian
\DeclareMathOperator{\calN}{\mathcal{N}}
\DeclareMathOperator{\calG}{\mathcal{G}}
\DeclareMathOperator{\calL}{\mathcal{L}}
\DeclareMathOperator{\calE}{\mathcal{E}}
% argmin and max operators
\DeclareMathOperator*{\argmin}{argmin} % no space, limits underneath in displays
% \DeclareMathOperator{\argmin}{argmin} % no space, limits on side in displays
\DeclareMathOperator*{\argmax}{argmax} % no space, limits underneath in displays
% \DeclareMathOperator{\argmax}{argmax} % no space, limits on side in displays



\newcommand\MyBox[2]{
	\fbox{\lower0.75cm
		\vbox to 1.7cm{\vfil
			\hbox to 1.7cm{\hfil\parbox{1.4cm}{#1\\#2}\hfil}
			\vfil}%
	}%
}



\usepackage{fancybox}
\usefonttheme[onlymath]{serif}
\boldmath

\usepackage{scalefnt}
\usepackage{ragged2e}
\usepackage{graphics}
\usepackage[export]{adjustbox} % aligns figures
\usepackage{bbm}
\usepackage{graphicx}
\usepackage{mathrsfs}
\usepackage{amsfonts}
\usepackage[authoryear,round]{natbib}
\usepackage{epigraph}

\setlength\epigraphwidth{8cm}
\setlength\epigraphrule{0pt}

\graphicspath{ {./figures/} }

%---------------------------------------------------------+
% Paquetes misceláneos de listas, set space, notas, etc   |
%---------------------------------------------------------+
\usepackage[shortlabels,inline]{enumitem}   % customiza liastas lists
\setlist[itemize,1]{label=$\bullet$}
\setlist[itemize,2]{label=\footnotesize{$\blacktriangleright$}}
\setlist[itemize,3]{label=\tiny{$\blacksquare$}}
\setlist[itemize,4]{label=\bfseries{\large{--}}}
% \setlist[enumerate,2]{label=\emph{\alph*})}
\newlist{steps}{enumerate}{1}               % Lista de pasos para usar en demostraciones
\setlist[steps,1]{leftmargin=*,label=\textit{Step \arabic*.},ref=\arabic*}

%----------------+
% Table packages |
%----------------+
\usepackage{array}          % Flexible column formatting
% \usepackage{spreadtab}  % Spreadsheet features
\usepackage{multirow}       % Allows table cells that span more than one row
\usepackage{booktabs}       % Enhance quality of tables
% to build master table
\newcommand{\ra}[1]{\renewcommand{\arraystretch}{#1}}

\def\calE{\mathcal{E}}
\def\calF{\mathcal{F}}
\def\calG{\mathcal{G}}
\def\calN{\mathcal{N}}
\def\calP{\mathcal{P}}
\def\calR{\mathcal{R}}
\def\calW{\mathcal{W}}
\DeclareMathOperator{\Expect}{\mathbb{E}}
\DeclareMathOperator{\trainsetn}{\mathcal{T}^{(n)}}
\DeclareMathOperator{\trainset}{\mathcal{T}}
\DeclareMathOperator{\testsetn}{\mathcal{T_s}^{(n)}}
\DeclareMathOperator{\testset}{\mathcal{T_s}}

\newtheorem{teorema}{Teorema}[section]
\newtheorem{definicion}[teorema]{Definición}
\newtheorem{lema}[teorema]{Lema}
\newtheorem{proposicion}[teorema]{Proposición}
\newtheorem{corolario}[teorema]{Corolario}
\newtheorem{observacion}[teorema]{Observación}
% \newtheorem{ejercicio}[teorema]{Ejercicio}
\newtheorem{ejemplo}[teorema]{Ejemplo}

\newcolumntype{L}[1]{>{\raggedright\let\newline\\\arraybackslash\hspace{0pt}}m{#1}}
\newcolumntype{C}[1]{>{\centering\let\newline\\\arraybackslash\hspace{0pt}}m{#1}}
\newcolumntype{R}[1]{>{\raggedleft\let\newline\\\arraybackslash\hspace{0pt}}m{#1}}

\beamertemplatenavigationsymbolsempty

\newcommand\MyBox[2]{
	\fbox{\lower0.75cm
		\vbox to 1.7cm{\vfil
			\hbox to 1.7cm{\hfil\parbox{1.4cm}{#1\\#2}\hfil}
			\vfil}%
	}%
}
%%%%%%%%%%%%%%%%%%%%%%%%%%%%%%%%%%%%%%%%%%%%%%%%%%%%%%%%%%%%%%%%%%%%%%%%%%%%%%%%%%%%%%%%%%%%%%%%%%%%%%%%%%%%
%%%%%%%%%%%%%%%%%%%%%%%%%%%%%%%%%%%%%%%%%%%%%%%%%%%%%%%%%%%%%%%%%%%%%%%%%%%%%%%%%%%%%%%%%%%%%%%%%%%%%%%%%%%%

  \title[Chagas \& Big Data]{Explorando migraciones y difusi\'on del mal de Chagas en América Latina con aprendizaje automático}
% \title[Chagas \& Big Data]{Explorando migraciones y difusi\'on del mal de Chagas en América Latina con aprendizaje automático}

  \author[Sarraute,Salles,de Monasterio]{Juan de Monasterio\inst{1}
  \and Carlos Sarraute\inst{3}
  \and Alejo Salles\inst{1}
  }

  \institute[]{
  \and \inst{1} Universidad de Buenos Aires
  \and \inst{3} GranData Labs

  }

  \date{ FCEyN \\ Diciembre, 2017}


%\setbeamertemplate{footline}[text line]{\bf \insertshortauthor \hfill \insertshorttitle \hfill \insertframenumber/29}
\setbeamertemplate{footline}{\hspace*{0.95\textwidth}\vspace{0.1cm} \insertframenumber/\inserttotalframenumber\null}

\AtBeginSection[]
{
  \begin{frame}<beamer>{Agenda}
    \tableofcontents[currentsection]
  \end{frame}
}

%%%%%%%%%%%%%%%%%%%%%%%%%%%%%%%%%%%%%%%%%%%%%%%%%%%%%%%%%%%%%%%%%%%%%%%%%%%%%%%%%%%%%%%%%%%%%%%%%%%%%%%%%%%%
%%%%%%%%%%%%%%%%%%%%%%%%%%%%%%%%%%%%%%%%%%%%%%%%%%%%%%%%%%%%%%%%%%%%%%%%%%%%%%%%%%%%%%%%%%%%%%%%%%%%%%%%%%%%
%%%%%%%%%%%%%%%%%%%%%%%%%%%%%%%%%%%%%%%%%%%%%%%%%%%%%%%%%%%%%%%%%%%%%%%%%%%%%%%%%%%%%%%%%%%%%%%%%%%%%%%%%%%%
\begin{document}


\begin{frame}
\titlepage\
\end{frame}


%%%%%%%%%%%%%%%%%%%%%%%%%%%%%%%%%%%%%%%%%%%%%%%%%%%%%%%%%%%%%%%%%%%%%%%%%%%%%%%%%%%%%%%%%%%%%%%%%%%%%%%%%%%%
\section{Introducción}
%%%%%%%%%%%%%%%%%%%%%%%%%%%%%%%%%%%%%%%%%%%%%%%%%%%%%%%%%%%%%%%%%%%%%%%%%%%%%%%%%%%%%%%%%%%%%%%%%%%%%%%%%%%%

\begin{frame}{Presentación  del proyecto}

	\begin{block}{Contexto}
		\begin{itemize}
			\item Comenzado en agosto 2015
			\item Investigación sobre la enfermedad del Chagas.
			\item Trabajo multidisciplinario de biología, computación, estadística
			\item ``Big Data'' + Aprendizaje Automático integrando datos de telefonía celular
			\item Buscamos caracterizar y predecir migraciones de usuarios
		\end{itemize}
	\end{block}

	%\pause

\end{frame}


%%%%%%%%%%%%%%%%%%%%%%%%%%%%%%%%%%%%%%%%%%%%%%%%%%%%%%%%%%%%%%%%%%%%%%%%%%%%%%%%%%%%%%%%%%%%%%%%%%%%%%%%%%%%

\begin{frame}{Problemática}

			El Chagas es una enfermedad causada por el \textit{tripanosoma cruzi}, un parásito que se extiende por todo el continente americano
			% y tambi\'en predominante en algunas comunidades que han migrado de esta zona. .

			\medskip  El insecto \textit{triatoma infestans} es el mayor transmisor de esta enfermedad

			\medskip  La Organización Mundial de la Salud (OMS) estima 65 millones de personas expuestas a esta enfermedad que es endémica en más de 21 países latino americanos
			%Esta zona comprende el norte y noroeste de la Argentina
			%así como varios países limítrofes (Brasil, Bolivia, Paraguay, Perú).

			\medskip El acceso a tratamientos es extremadamente limitado para las personas infectadas

\end{frame}

%%%%%%%%%%%%%%%%%%%%%%%%%%%%%%%%%%%%%%%%%%%%%%%%%%%%%%%%%%%%%%%%%%%%%%%%%%%%%%%%%%%%%%%%%%%%%%%%%%%%%%%%%%%%

\begin{frame}{Epidemiología del Chagas}
			\includegraphics[height=.9\textheight]{slides/triatomine-map.jpg}
\end{frame}

%%%%%%%%%%%%%%%%%%%%%%%%%%%%%%%%%%%%%%%%%%%%%%%%%%%%%%%%%%%%%%%%%%%%%%%%%%%%%%%%%%%%%%%%%%%%%%%%%%%%%%%%%%%%
\begin{frame}{Chagas en Argentina}
	\begin{columns}
		\begin{column}{0.45\textwidth}
			% In Argentina, the disease is endemic in the \textit{Gran Chaco} region and
			% also predominant in communities with migrations from this area.
			La enfermedad es endémica de la zona del Gran Chaco
			% y tambi\'en predominante en algunas comunidades que han migrado de esta zona.

			\medskip Las estimaciones de usuarios infectados oscilan alrededor de 1.5 millones de personas con solo 1 a 2 mil tratamientos realizados por año. En 2009 los tests de seroprevalencia en embarazadas alcanzaba el 4.2\% de casos.

			\medskip
			 % infected users and only 1 to 2 thousand treatments done yearly, where 30\% of all infected will develop a cardiopathy.
			% % % %DECIR ORAL: Seroprevalence tests in pregnant women reached 4.2% in 2009 which predicted 1.300 infected newborns that year.

		\end{column}
		\begin{column}{0.45\textwidth}
			\includegraphics[height=.7\textheight]{slides/Ambientes_GranChaco_TNC-Argentina.png}
		\end{column}
	\end{columns}
\end{frame}

%%%%%%%%%%%%%%%%%%%%%%%%%%%%%%%%%%%%%%%%%%%%%%%%%%%%%%%%%%%%%%%%%%%%%%%%%%%%%%%%%%%%%%%%%%%%%%%%%%%%%%%%%%%%
\begin{frame}{Chagas en México}
	\begin{columns}
		\begin{column}{0.3\textwidth}

			La enfermedad es endémica en algunas regiones y estados particulares
			 % is endemic in some particular states of the country, shown in the map.
			%The disease is endemic in the states of Jalisco, Oaxaca, Veracruz, Guerrero, Morelos, Puebla, Hidalgo and Tabasco.
			%This selection was based on the top 25\% prevalance states.

			% \medskip Non official reports estimate 5.5 million of potentially affected people and studies indicate that less than 0.5\% of infected have access to treatments.

			\medskip No oficialmente, los reportes estiman 5.5 millones de personas potencialmente afectados y donde menos del 0.5\% de los infectados tienen acceso a tratamientos
			 % official reports estimate 5.5 million of potentially affected people and studies indicate that less than 0.5\% of infected have access to treatments.


		\end{column}
		\begin{column}{0.7\textwidth}
			\includegraphics[width=\textwidth]{slides/Ambientes_Gran_Chaco-Mexico_original.png}
		\end{column}
	\end{columns}
\end{frame}
%%%%%%%%%%%%%%%%%%%%%%%%%%%%%%%%%%%%%%%%%%%%%%%%%%%%%%%%%%%%%%%%%%%%%%%%%%%%%%%%%%%%%%%%%%%%%%%%%%%%%%%%%%%%

\begin{frame}{Puntos claves}
	% Chagas is a disease:
	El Chagas es una enfermedad:
	\begin{itemize}
		\item \ldots con diferentes vías de transmisión y donde las \textbf{migraciones de largo plazo} juegan un papel clave
		%explicar vertical, congenita y transfusion/transplante
		\item \ldots donde un pequeño número de la población infectada sabe que la está padeciendo
		% for which a low proportion of the infected population knows that they have the disease.
		\item \ldots con una fase asintomática que puede extenderse más de 10 años
		% % ORALLY SAY: 30 years of vector mitigation activities in GC.
		\item \ldots que es epidémica y desatendida
		% \item ... which is epidemic and unattended.
	\end{itemize}
	En particular, la fundación busca encontrar chagásicos en zonas no tradicionalmente endémicas
	%Es decir, gente de fuera de Gran Chaco infectada con \textit{Trypanosoma Cruzi} (el parásito que causa el Chagas).
\end{frame}


%%%%%%%%%%%%%%%%%%%%%%%%%%%%%%%%%%%%%%%%%%%%%%%%%%%%%%%%%%%%%%%%%%%%%%%%%%%%%%%%%%%%%%%%%%%%%%%%%%%%%%%%%%%%%
\begin{frame}{Objetivo}
	\begin{block}{\ldots atacar problemáticas de América Latina relaciónadas con el Chagas}

	% The purpose of this work is to support ongoing national health campaigns
	% by analyzing mobility information contained in Call Detail Records.
	Este trabajo propone un modelo que utiliza información de registros (logs) de llamados telefónicos
	como forma de observar en qué regiones del país se espera encontrar una alta proporción de personas chagásicas

	\bigskip
	Sabiendo que las migraciones humanas tienen un rol crucial en la diseminación de la enfermedad, el modelo busca
	detectar migraciones a partir de los patrones de uso celular
	% , donde, idealmente, se vea que una fluida comunicación con el lugar de orígen es indicador

	\bigskip
	Idealmente podremos encontrar aquellos indicadores de usuarios que mejor determinen las zonas del país tradicionalmente no-endémicas, donde exista más intercambio migratorio con la zona endémica del país
	% We strongly assume that long-term migrations are related to fluid communications with the past area of residence.
	\end{block}
\end{frame}

%% HABLAR de pre trabajos en vih y malaria en el D4D? .
%

%%%%%%%%%%%%%%%%%%%%%%%%%%%%%%%%%%%%%%%%%%%%%%%%%%%%%%%%%%%%%%%%%%%%%%%%%%%%%%%%%%%%%%%%%%%%%%%%%%%%%%%%%%%%
\section{Clasificación Binaria Supervisada}
%%%%%%%%%%%%%%%%%%%%%%%%%%%%%%%%%%%%%%%%%%%%%%%%%%%%%%%%%%%%%%%%%%%%%%%%%%%%%%%%%%%%%%%%%%%%%%%%%%%%%%%%%%%%

\begin{frame}{Objetivo Principal}
		\centering

		Sea $\textbf{X} \in \mathbb{R}^{n \times p}$ matriz aleatoria compuesta por muestras $\textbf{X}_i \in \mathbb{R}^{p}$\\
		$\textbf{Y} \in \mathbb{R}^n$ es el vector aleatorio de salida \\
		Se busca una función $f_\theta: \mathbb{R}^{p} \rightarrow  \mathbb{R}$ que optimice u ``aprenda''

	\begin{align*}%\label{expectedPredictionError}
		\begin{split}
		\operatorname{argmin}_{\theta} \  EPE(\theta) = & \Expect_{\textbf{X},\textbf{Y}} \left[ L(Y,f_\theta(X))\right]  \\
			= & \int L(y,f_\theta (x)) p(x,y) dx dy
		\end{split}
	\end{align*}

	para generar \textit{predicciones} para cualquier muestra

	\smallskip
	$f_\theta$ es un clasificador o algoritmo indexado por $\theta \in \Theta$

	\smallskip
	$L: \mathbb{R}^{2} \rightarrow  \mathbb{R}_{\geq 0}$ se denomina la función de pérdida


\end{frame}
%%%%%%%%%%%%%%%%%%%%%%%%%%%%%%%%%%%%%%%%%%%%%%%%%%%%%%%%%%%%%%%%%%%%%%%%%%%%%%%%%%%%%%%%%%%%%%%%%%%%%%%%%%%%

\begin{frame}{Consideraciones}

	$f_\theta^*$ intenta aproximar la relación desconocida entre $\textbf{X}$ y $\textbf{Y}$

	\bigskip

	$f(\cdot)$, $L(\cdot)$ y el espacio de busqueda $\Theta$ de los \textit{aproximantes} son predeterminados

	\bigskip\
	
	Desconocemos $P(\textbf{X},\textbf{Y})$ pues nos basamos en datos finitos\\

	$X = \trainsetn  \sqcup \testsetn $ con $\trainsetn$ conjunto de aprendizaje/entrenamiento

	$$\testsetn$$ conjunto de aprendizaje

	\bigskip\
	En Clasificación: $y_i \in \{0,1\} \ \forall  i$ y $L(z,w) =  \mathbb{I}(z \neq w ) $

\end{frame}

%%%%%%%%%%%%%%%%%%%%%%%%%%%%%%%%%%%%%%%%%%%%%%%%%%%%%%%%%%%%%%%%%%%%%%%%%%%%%%%%%%%%%%%%%%%%%%%%%%%%%%%%%%%%

\begin{frame}{Aproximaciones de $EPE$}

Varios tipos de errores a minimizar en el aprendizaje con \textbf{finitos} datos observados
    \begin{block}{Error de entrenamiento y de generalización}

        $$  Err_{\trainsetn}(\theta) =
        \Expect_{ \textbf{X}, \textbf{Y} } \left[ L(Y,f_{\theta}(X))  |  \trainsetn \right] \approx
        \dfrac{1}{n} \sum_{i=1}^{n} \mathbb{I} \left(f_\theta (X_{i})\neq Y_{i} \right)  $$

        $$Err_{\testsetn}(\theta) \approx
        \dfrac{1}{n} \sum_{i=1}^{n} \mathbb{I} \left(f_\theta (\tilde{X_{i}})\neq \tilde{Y_{i}} \right)  $$

	\end{block}

	Un modelo se dice sobreajustado (``overfit'') cuando $\mid Err_{\testset}(\theta) - Err_{\trainset}(\theta)  \mid \ >> 0$

	Para esto es importante controlar la \textbf{complejidad} del modelo

\end{frame}

%%%%%%%%%%%%%%%%%%%%%%%%%%%%%%%%%%%%%%%%%%%%%%%%%%%%%%%%%%%%%%%%%%%%%%%%%%%%%%%%%%%%%%%%%%%%%%%%%%%%%%%%%%%%

\begin{frame}{ Ajuste Poly-M vs Ajuste Poly-2 }
\center
\includegraphics[width=0.9\textheight]{slides/wiki_overfit_complex_classifier_demo.png}
\end{frame}


%%%%%%%%%%%%%%%%%%%%%%%%%%%%%%%%%%%%%%%%%%%%%%%%%%%%%%%%%%%%%%%%%%%%%%%%%%%%%%%%%%%%%%%%%%%%%%%%%%%%%%%%%%%%

\begin{frame}{Convergencia a $EPE$ en clasificadores binarios}
Bajo ciertas condiciones sobre $\Theta$ tenemos
	\begin{block}{Desigualdad de Vapnik–Chervonenkis}

		\begin{align*}
			\begin{split}
				P\left(\sup_{\theta\in \Theta}\left|Err_{\trainsetn}(\theta)-EPE (\theta)\right|>\varepsilon \right) & \leq
				8S (\Theta,n) e^{{-n \varepsilon^{2}/32}}\\
				\Expect\left[\sup_{\theta \in \Theta}\left| Err_{\trainsetn}(\theta)-EPE (\theta)\right|\right] &
				\leq 2\sqrt{\dfrac{\log S(\Theta,n)+\log2}{n}}
			\end{split}
		\end{align*}

	\end{block}

	Donde $S(\Theta)$ es el coeficiente de ``shattering'' de la clase
\end{frame}

%%%%%%%%%%%%%%%%%%%%%%%%%%%%%%%%%%%%%%%%%%%%%%%%%%%%%%%%%%%%%%%%%%%%%%%%%%%%%%%%%%%%%%%%%%%%%%%%%%%%%%%%%%%%

\begin{frame}[shrink=5]{Regresión Logística}
% = \frac{P(x|C_1)p(C_1) }{P(x|C_1)p(C_1) + P(x|C_0)p(C_0)}
Con $\sigma(z) = \frac{1}{1 + e^{-z}}$ y $logodds(x)$ como $\log \left(  \frac{ P(C_1 \mid x)}{P(C_0 \mid x ) } \right)$
	\begin{block}{Idea}
	\footnotesize
	\begin{align*}
	% 	% \begin{split}
		% $$
		P(C_1| X_i)  = \frac{1}{1 + \exp \left(- \log \left(  \frac{ P(X_i|C_1)p(C_1)}{P(X_i|C_0)p(C_0)} \right) \right)} = \sigma\left(logodds(X_i)\right)
		% $$
	% 	% \end{split}
	\end{align*}

	El modelo propone $\exists \theta \ t.q.$
	$$logodds(X_i) = \sum_{j=1}^p \theta^j \ X_i^j  \ , \forall X_i \in X $$
	$$Y_i \mid X_i \ \sim \operatorname{Bernoulli}(p_i)$$

	\end{block}

	\begin{block}{Función de pérdida (MLL)}
		$$\operatorname{argmin}_{\theta} \ \sum_{i=1}^N \big(Y_i ( \theta \cdot X_i ) + \log(1 + e^{1- \theta \cdot X_i} ) \big) + \lambda { \| \theta \|_{2}}^2$$
	\end{block}
	% $\gamma$ es un hiperparámetro para la regularizacion del modelo
\end{frame}

%%%%%%%%%%%%%%%%%%%%%%%%%%%%%%%%%%%%%%%%%%%%%%%%%%%%%%%%%%%%%%%%%%%%%%%%%%%%%%%%%%%%%%%%%%%%%%%%%%%%%%%%%%%%

\begin{frame}[shrink=2]{Árbol de decisión}

Algoritmo recursivo y aleatorio que particióna el espacio en forma binaria $\mathcal{T}_{iz}(d,j) = \{x \in \mathcal{T} \ / \ x^j \leq d \}$ y $\mathcal{T}_{de}(d,j)$

	\begin{block}{Idea}
		% $$$		
		\begin{align*}
			\min_{d, j} \big[ \min_{c_{iz} }  \frac{1}{N_{iz}}\sum_{x \in \mathcal{T}_{iz}(d,j) } Imp(y,c_{iz}) \
			+  \min_{c_{de}}  \frac{1}{N_{de}}\sum_{x \in \mathcal{T}_{de}(d,j) } Imp(y,c_{de}) \big]
		\end{align*}
			donde $Imp$ es una función de \textit{impureza} 
		% \begin{align*}
		% asdf
		% \end{align*}
	\end{block}

	\begin{block}{Función de pérdida}
	\small
	{
	\center
	$h_{\theta}(X_i) = \sum_{l=1}^L c_l I(X_i \in \mathcal{T}_l)$ \\
	}
 	con $L$ el número de regiones. Aqui $\theta = \bigcup\limits_{l=1}^{L} (d_l,j_l)$

	% \begin{align*}			
	% 			% h_{\theta}(X_i) = \sum_{l=1}^L c_l I(X_i \in \mathcal{T}_l)
	% 	\end{align*}

	\end{block}
\end{frame}

%%%%%%%%%%%%%%%%%%%%%%%%%%%%%%%%%%%%%%%%%%%%%%%%%%%%%%%%%%%%%%%%%%%%%%%%%%%%%%%%%%%%%%%%%%%%%%%%%%%%%%%%%%%%

\begin{frame}{Árbol de ejemplo}
		\includegraphics[width = 0.9 \paperwidth, height = 0.7 \paperheight,
										trim = 0.2 80 0.2 0.2cm, left, clip = true]{decision_tree/full_tree_map.png}
\end{frame}

%%%%%%%%%%%%%%%%%%%%%%%%%%%%%%%%%%%%%%%%%%%%%%%%%%%%%%%%%%%%%%%%%%%%%%%%%%%%%%%%%%%%%%%%%%%%%%%%%%%%%%%%%%%%

\begin{frame}{Sobreajuste}
\center\
\includegraphics[width=1\textheight]{figure-biasVariance/dtree_overfit_problem_2.png}
\end{frame}

%%%%%%%%%%%%%%%%%%%%%%%%%%%%%%%%%%%%%%%%%%%%%%%%%%%%%%%%%%%%%%%%%%%%%%%%%%%%%%%%%%%%%%%%%%%%%%%%%%%%%%%%%%%%

\begin{frame}{Clasificadores de Ensamble }
	
	Sea $\bigcup\limits_{m=1}^{M} h_{\theta_m}$ un conjunto de pequeños arboles
	
	\begin{block}{Bosques Aleatorios} 
		\begin{align*}
			Y_i \approx f_{\theta}(X_i) = \dfrac{1}{M}\sum_{m=1}^M h_{\theta_m}(X_i) 
		\end{align*}

	\end{block}
	$ h_{\theta_m}$

	\begin{block}{``Gradient Tree Boosting''}
	Mejora iterativa de la pérdida 
		\begin{align*}
			Y_i \approx f_{\theta}(X_i) = \dfrac{1}{M}\sum_{m=1}^M h_{\theta_m}(X_i) 
		\end{align*}
	\end{block}


\end{frame}



\begin{frame}{FOOOO}
	Mejora iterativa de arboles $h_{\theta}$

	\begin{block}{Idea} 
		\begin{align*}
			Y_i \approx f_{\theta}(X_i) = \dfrac{1}{M}\sum_{m=1}^M h_{\theta}(X_i) 
		\end{align*}

	\end{block}

\end{frame}


%%%%%%%%%%%%%%%%%%%%%%%%%%%%%%%%%%%%%%%%%%%%%%%%%%%%%%%%%%%%%%%%%%%%%%%%%%%%%%%%%%%%%%%%%%%%%%%%%%%%%%%%%%%%


\begin{frame}{``Naive Bayes''}

Suponemos una distribución conocida para $p(\cdot)$
	\begin{block}{Idea}
		Asumir \textit{inocentemente}
		$$p(X^j |X^1,\ldots,\overline{X^j},\ldots,X^p, C_k) \ = \ p(X^j | C_k) \ \forall k \in \{0,1\}$$
	\end{block}

	\begin{block}{Función de decisión }

		\begin{align*}
			p(C_1) \propto f(X) = \hat{p}(C_{1}) \  \prod_{i=1}^{n} \hat{p}(X_{i}\mid C_{1})
		\end{align*}

		\begin{align*}
			\hat{Y} \ = \ \mathbb{I}\left[  f(X)  >= \ \frac{1}{2} \right]
		\end{align*}

	\end{block}
	% $\gamma$ es un hiperparámetro para la regularizacion del modelo
\end{frame}

%%%%%%%%%%%%%%%%%%%%%%%%%%%%%%%%%%%%%%%%%%%%%%%%%%%%%%%%%%%%%%%%%%%%%%%%%%%%%%%%%%%%%%%%%%%%%%%%%%%%%%%%%%%%

\begin{frame}{estimación de Fronteras de Decision}
		\includegraphics[width = 0.9 \paperwidth, height = 0.7 \paperheight, 
										trim = 0.2 0.2 0.2 0.2cm, left, clip = true]{slides/plot_classifier_comparison.png}
\end{frame}

%%%%%%%%%%%%%%%%%%%%%%%%%%%%%%%%%%%%%%%%%%%%%%%%%%%%%%%%%%%%%%%%%%%%%%%%%%%%%%%%%%%%%%%%%%%%%%%%%%%%%%%%%%%%

\begin{frame}{$K$ - Validación cruzada}

Optimización de configuraciones de hiperparámetros $\mathcal{A}$ via estimacion del $EPE$

	\begin{block}{Idea}
		Dado un $\alpha \in \mathcal{A}$, particiónar las muestras en $K$ conjuntos al azar
		\\
		Entrenar $\hat{f}_{\alpha}^{-k}, \ k \in \{1, \ldots, K\}$
		\\
		donde cada clasificador $\hat{f}^{-k}$ aprende sin usar las muestras de la partición $k$
		\\
		Computar $CV(\alpha) = \frac{1}{n} \sum^n_{i=1} L\left( y_i, \hat{f}_{\alpha}^{-k_i}(x_i) \right)$
	\end{block}

\end{frame}
%%%%%%%%%%%%%%%%%%%%%%%%%%%%%%%%%%%%%%%%%%%%%%%%%%%%%%%%%%%%%%%%%%%%%%%%%%%%%%%%%%%%%%%%%%%%%%%%%%%%%%%%%%%%

\begin{frame}{Métricas de clasificadores binarios}

	\center
	Conteo de las combinaciones de $\hat{Y}$ y $Y$

		\begin{columns}
			\begin{column}{0.55 \textwidth}
			\scriptsize
				\noindent
				\renewcommand\arraystretch{1}
				\setlength\tabcolsep{0pt}
				\begin{tabular}{c >{\bfseries}r @{\hspace{0.7em}}c @{\hspace{0.4em}}c @{\hspace{0.7em}}l}
				\multirow{10}{*}{\parbox{1.1cm}{\bfseries\raggedleft\ Valor real $y$}} &
				& \multicolumn{2}{c}{\bfseries Valor predicho $\hat{y}$} & \\
				& & \bfseries \^{P} \ $(0)$ & \bfseries \^{N} \ $(1)$  \\
				& P \ $(0)$ & \MyBox{Verdadero}{Positivo (TP)} & \MyBox{Falso}{Negativo (FN)} & \\[2.4em]
				& N \ $(1)$ & \MyBox{Falso}{Positivo (FP)} & \MyBox{Verdadero}{Negativo (TN)} & \\
				%& total & P \ $(0)$ & &
				\end{tabular}

			\end{column}

			\begin{column}{0.45 \textwidth}
			\scriptsize
				\begin{align*}
					\begin{split}
						& Accuracy \ =\ \frac{ TP + TN }{ TP + TN + FP + FN }\\
						& Precision \ =\ \frac{ TP }{TP + FN}\\
						& Recall \  =\ \frac{TP }{TP + FP}\\
					  	& F1 = \mbox{media armónica de } Precision \mbox{ y } Recall \\
					  	& Curva \ ROC: \\
					  	& \sigma(\pi) = (Recall(\pi), FPR(\pi)) \ t.q. \ \pi \in (0,1)
					\end{split}
				\end{align*}
			\end{column}
		\end{columns}

	% \end{block}
\end{frame}



%%%%%%%%%%%%%%%%%%%%%%%%%%%%%%%%%%%%%%%%%%%%%%%%%%%%%%%%%%%%%%%%%%%%%%%%%%%%%%%%%%%%%%%%%%%%%%%%%%%%%%%%%%%%
\section{ Descripción de los datos}
%%%%%%%%%%%%%%%%%%%%%%%%%%%%%%%%%%%%%%%%%%%%%%%%%%%%%%%%%%%%%%%%%%%%%%%%%%%%%%%%%%%%%%%%%%%%%%%%%%%%%%%%%%%%

\begin{frame}{Visualización Social: El grafo de comunicaciones}
		\center\
		\includegraphics[scale=0.24]{slides/Graph-screenshot.jpg}
\end{frame}

%%%%%%%%%%%%%%%%%%%%%%%%%%%%%%%%%%%%%%%%%%%%%%%%%%%%%%%%%%%%%%%%%%%%%%%%%%%%%%%%%%%%%%%%%%%%%%%%%%%%%%%%%%%%
\begin{frame}{Registros de Llamados Celulares (CDRs)}

Los conjuntos de datos consisten de registros detallados de llamados (CDRs) para Mexico y Argentina provistos por una compañía de telecomunicaciones (TelCo) de cada país. El primero tiene 24 meses de datos mientras que el segundo es de 5 meses.

\medskip

En total, cada dataset tiene mas de 10 mil millones de registros realizados por 8 millones y 2 millones de usuarios, respectivamente para Argentina y México\footnote{No tenemos ninguna información personal de los usuarios como nombre, dirección, teléfono y su privacidad está asegurada mediante criptografía.}

\medskip
Para los dos países tenemos más de cuatro mil antenas celulares geolocalizadas

% \end{block}
\end{frame}

%%%%%%%%%%%%%%%%%%%%%%%%%%%%%%%%%%%%%%%%%%%%%%%%%%%%%%%%%%%%%%%%%%%%%%%%%%%%%%%%%%%%%%%%%%%%%%%%%%%%%%%%%%%%

\begin{frame}{Definiciones 1}

	Podemos representar cada registro en una tupla $r =\left < u, w, t, d, a \right >$ que contiene:
	\begin{itemize}
		\item Usuarios anonimizados $u$ y $w$ en esa llamada
		\item Tiempo de inicio y duración de la llamada $t$
		\item Dirección $d$ de la llamada (entrante o saliente, con respecto a $u$)
		\item $a$ es el ID de la antena telefónica usada por $u$ en esa comunicación
	\end{itemize}

	\medskip
	$\forall u \  \exists \  H_u$ el lugar de residencia de un usuario que es determinado como la antena $a$ más utilizada de lunes a viernes y fuera del horario laboral. La endemicidad de un usuario está determinado por la condición $H_u \in E_Z$

	\medskip

	El conjunto de antenas puede ser particiónado a partir de su pertenencia a la zona endémica $A \ = E_Z \cup {E_Z}^{\complement} $

\end{frame}



\begin{frame}{Definiciones 2}
	Los registros definen el grafo de comunicaciones
	$\calG = \left< \calN, \calE \right> $ donde $\calN$ es el conjunto de nodos (usuarios) y $\calE$ el de interacciones

	\medskip
	Dividimos los 24 meses del dataset mexicano dos períodos:
	\begin{enumerate*}[label={\alph*)},]
			\item $T_0$ desde Enero 2014 hasta Julio 2015 es el \textit{pasado} y
			\item $T_1$ desde Agosto 2015 hasta Diciembre 2015, considerado el \textit{presente}
			% . Este va a ser nuestro dataset principal a procesar.
	\end{enumerate*}

	\medskip
	Diremos que un usuario $u$ es vulnerable en un mes dado si $\exists \ r = \ \left < u, w, t, d, a \right > $ tal que $H_w \in E_z$ para ese mes

\end{frame}


%%%%%%%%%%%%%%%%%%%%%%%%%%%%%%%%%%%%%%%%%%%%%%%%%%%%%%%%%%%%%%%%%%%%%%%%%%%%%%%%%%%%%%%%%%%%%%%%%%%%%%%%%%%%
%%%%%%%%%%%%%%%%%%%%%%%%%%%%%%%%%%%%%%%%%%%%%%%%%%%%%%%%%%%%%%%%%%%%%%%%%%%%%%%%%%%%%%%%%%%%%%%%%%%%%%%%%%%%

% Procedimiento:
	% Obtener la lista de antenas de GC
	% Determinar la casa de los usuarios
	% Determinar, para cada usuario, si se comunicó con un habitante de GC
	% Determinar las agregaciones por antena
	% Visualizar(agrego alfa, min_volume y zoom en regiones)

\begin{frame}{Agregación a nivel de antenas}

	%\pause\
	\begin{block}{Usuarios o llamados vulnerables}
	Para un mes dado y una antena $a$ tendremos la tupla $\left< N_a, V_a, C_a, VC_a \right>$ donde
		\begin{itemize}
			\item El número total de usuarios residentes $N_a$
			\item El número total de residentes vulnerables $V_a$
			\item El volumen total de llamados salientes $C_a$
			\item El volumen de llamados salientes en una interacción \textit{vulnerable} $VC_a$
		\end{itemize}
	% Esta  definen las propiedas are the properties we extracted for each antenna in the studied countries.
	\end{block}
\end{frame}

%%%%%%%%%%%%%%%%%%%%%%%%%%%%%%%%%%%%%%%%%%%%%%%%%%%%%%%%%%%%%%%%%%%%%%%%%%%%%%%%%%%%%%%%%%%%%%%%%%%%%%%%%%%%
\begin{frame}{Exploración con mapas de calor }
		Generamos mapas de calor, graficando un círculo por antena sobre un mapa geopolítico

		\medskip
		El color del círculo considera o la tasa de llamados vulnerables $\frac{V_a}{N_a}$ o la de llamados vulnerables $\frac{C_a}{VC_a}$

		\medskip
		El \textbf{\'area} del círculo depende de la cantidad de usuarios o llamados de $a$


	\begin{block}{Filtros de antenas}
		Definimos a $\beta$ como nuestro parámetro de cota inferior para que una antena sea graficada

		\medskip
		% Buscamos poder encontrar aquellas antenas donde existen mayores interacciones con las zonas endémicas

	\end{block}
\end{frame}

%%%%%%%%%%%%%%%%%%%%%%%%%%%%%%%%%%%%%%%%%%%%%%%%%%%%%%%%%%%%%%%%%%%%%%%%%%%%%%%%%%%%%%%%%%%%%%%%%%%%%%%%%%%%
% \section{Exploración de los datos}
%%%%%%%%%%%%%%%%%%%%%%%%%%%%%%%%%%%%%%%%%%%%%%%%%%%%%%%%%%%%%%%%%%%%%%%%%%%%%%%%%%%%%%%%%%%%%%%%%%%%%%%%%%%%



%%%%%%%%%%%%%%%%%%%%%%%%%%%%%%%%%%%%%%%%%%%%%%%%%%%%%%%%%%%%%%%%%%%%%%%%%%%%%%%%%%%%%%%%%%%%%%%%%%%%%%%%%%%%
\begin{frame}
	\frametitle{Mapa de calor: Argentina, $\beta$ = 1 \%}
	\center\
	\includegraphics[height=.9\textheight,width = .9\columnwidth, keepaspectratio]
	{slides/201112_hi_res_argentina_usuarios_proporcion_circulos_beta1.png}
\end{frame}

%%%%%%%%%%%%%%%%%%%%%%%%%%%%%%%%%%%%%%%%%%%%%%%%%%%%%%%%%%%%%%%%%%%%%%%%%%%%%%%%%%%%%%%%%%%%%%%%%%%%%%%%%%%%
\begin{frame}
	\frametitle{Mapa de calor: Argentina, $\beta$ = 15 \%}
	\center\
	\includegraphics[height=.9\textheight,width = .9\columnwidth, keepaspectratio]
	{slides/201112_hi_res_argentina_usuarios_proporcion_circulos_beta15.png}
\end{frame}

%%%%%%%%%%%%%%%%%%%%%%%%%%%%%%%%%%%%%%%%%%%%%%%%%%%%%%%%%%%%%%%%%%%%%%%%%%%%%%%%%%%%%%%%%%%%%%%%%%%%%%%%%%%%
\begin{frame}
	\frametitle{Mapa de calor: Argentina, $\beta$ = 30 \%}
	\center\
	\includegraphics[height=.9\textheight,width = .9\columnwidth, keepaspectratio]
	{slides/201112_hi_res_argentina_usuarios_proporcion_circulos_beta30.png}
\end{frame}

%%%%%%%%%%%%%%%%%%%%%%%%%%%%%%%%%%%%%%%%%%%%%%%%%%%%%%%%%%%%%%%%%%%%%%%%%%%%%%%%%%%%%%%%%%%%%%%%%%%%%%%%%%%%
\begin{frame}
	\frametitle{Mapa de calor: Argentina, $\beta$ = 50 \%}
	\center\
	\includegraphics[height=.9\textheight,width = .9\columnwidth, keepaspectratio]
	{slides/201112_hi_res_argentina_usuarios_proporcion_circulos_beta50.png}
\end{frame}

%%%%%%%%%%%%%%%%%%%%%%%%%%%%%%%%%%%%%%%%%%%%%%%%%%%%%%%%%%%%%%%%%%%%%%%%%%%%%%%%%%%%%%%%%%%%%%%%%%%%%%%%%%%%

\setbeamercolor{background canvas}{bg=white}

\begin{frame}{Regiones focalizadas}
	A partir de estas primeras visualizaciones, con la gente de la fundación decidimos focalizar la granularidad en regiones específicas y fuera del Gran Chaco.
	%A partir de estos primeros mapas se decidi\'o enfocar la visualizaci\'on y mejorar la precisi\'on en ciertas regiones fuera del Gran Chaco.
		\begin{itemize}
			\item Tierra del Fuego
			\item Este de Río Negro
			\item Buenos Aires
			\item Capital Federal, Sur y Norte de CABA y AMBA
			\item Córdoba Central
		\end{itemize}
\end{frame}

%%%%%%%%%%%%%%%%%%%%%%%%%%%%%%%%%%%%%%%%%%%%%%%%%%%%%%%%%%%%%%%%%%%%%%%%%%%%%%%%%%%%%%%%%%%%%%%%%%%%%%%%%%%%
\begin{frame}
	\frametitle{Mapa de calor: AMBA, $\beta$ = 2 \%}
	\centering
	\includegraphics[height=.9\textheight,width = .9\columnwidth,keepaspectratio]
	{slides/201112_hi_res_amba_usuarios_proporcion_circulos_beta2.png}
\end{frame}
%%%%%%%%%%%%%%%%%%%%%%%%%%%%%%%%%%%%%%%%%%%%%%%%%%%%%%%%%%%%%%%%%%%%%%%%%%%%%%%%%%%%%%%%%%%%%%%%%%%%%%%%%%%%
\begin{frame}
	\frametitle{Mapa de calor: AMBA, $\beta$ = 10 \%}
	\centering
	\includegraphics[height=.9\textheight,width = .9\columnwidth,keepaspectratio]
	{slides/201112_hi_res_amba_usuarios_proporcion_circulos_beta10.png}
\end{frame}
%%%%%%%%%%%%%%%%%%%%%%%%%%%%%%%%%%%%%%%%%%%%%%%%%%%%%%%%%%%%%%%%%%%%%%%%%%%%%%%%%%%%%%%%%%%%%%%%%%%%%%%%%%%%
\begin{frame}
	\frametitle{Mapa de calor: AMBA, $\beta$ = 20 \%}
	\centering
	\includegraphics[height=.9\textheight,width = .9\columnwidth,keepaspectratio]
	{slides/201112_hi_res_amba_usuarios_proporcion_circulos_beta20.png}
\end{frame}
%%%%%%%%%%%%%%%%%%%%%%%%%%%%%%%%%%%%%%%%%%%%%%%%%%%%%%%%%%%%%%%%%%%%%%%%%%%%%%%%%%%%%%%%%%%%%%%%%%%%%%%%%%%%
\begin{frame}
	\frametitle{Heat Map AMBA, $\beta$ = 28 \%}
	\centering
	\includegraphics[height=.9\textheight,width = .9\columnwidth,keepaspectratio]
	{slides/201112_hi_res_amba_usuarios_proporcion_circulos_beta28.png}
\end{frame}

%%%%%%%%%%%%%%%%%%%%%%%%%%%%%%%%%%%%%%%%%%%%%%%%%%%%%%%%%%%%%%%%%%%%%%%%%%%%%%%%%%%%%%%%%%%%%%%%%%%%%%%%%%%%
\begin{frame}{Comunidades resaltadas}
	Luego de filtrar los mapas y explorar con distintas visualizaciones, encontramos antenas especificas que se resaltan por su nivel de vulnerabilidad
	%Como resultado del filtrado en cada zona, nos hemos quedado con las antenas destacadas por su nivel de riesgo y, utilizando su ubicaci\'on, las emparejamos con localidades o comunidades argentinas.

	\bigskip

	\begin{block}{Ejemplos de comunidades en Argentina}
		\begin{itemize}
			\item Cordoba: Freyre, La Tordilla, Balnearia
			%hay muchas mas pero no se si esta bien poner esto
			\item AMBA: Avellaneda, Parque Patricios, San Isidro
			\item Provincia de Buenos Aires: Lima, San Nicolas
			\item La Rioja: Chamical y Malanz\'an
			\item Salta: Tartagal
		\end{itemize}

	\end{block}
\end{frame}


%%%%%%%%%%%%%%%%%%%%%%%%%%%%%%%%%%%%%%%%%%%%%%%%%%%%%%%%%%%%%%%%%%%%%%%%%%%%%%%%%%%%%%%%%%%%%%%%%%%%%%%%%%%%

\begin{frame}{Definición de problemas de migración A}

	Sean $EU_{1}$ y $EU_{0}$ los conjuntos de usuarios que vivieron en la región endémica durante $T_1$ y $T_0$ respectivamente
	% así como también $H^0_u$ y $H^1_u$ para referirnos al hogar de $u$ a través de los períodos
	\begin{block}{Problema 1}
	Cuales usuarios vivían en la región endémica en el pasado:
		\begin{align*}
		Y^{(1)}_u =
		\begin{cases}
		&1 \ \mbox{if} \ u \in EU_{0} \\
		&0 \ \mbox{if not}.
		\end{cases}
		\end{align*}
	\end{block}


	\begin{block}{Problema 2}
		Cuales vivían en la región endémica y luego migraron:
		\begin{align*}
			Y^{(2)}_u =
			\begin{cases}
				&1 \ \mbox{if} \ u \in EU_{0} \cap { EU_{1} }^{\complement}  \\
				&0 \ \mbox{if not}.
			\end{cases}
		\end{align*}
	\end{block}

\end{frame}

%%%%%%%%%%%%%%%%%%%%%%%%%%%%%%%%%%%%%%%%%%%%%%%%%%%%%%%%%%%%%%%%%%%%%%%%%%%%%%%%%%%%%%%%%%%%%%%%%%%%%%%%%%%%

\begin{frame}{Definición de problemas de migración B}
	\begin{block}{Problemas 3}

		Predecir aquellos usuarios que migraron entre regiones, en cualquier dirección
		\begin{align*}
			Y^{(3)}_u =
			\begin{cases}
				&1 \ \mbox{if} \ u \in (EU_{0} \cap { EU_{1} }^{\complement}) \cup (EU_{1} \cap { EU_{0} }^{\complement}) \\
				&0 \ \mbox{if not}.
			\end{cases}
		\end{align*}
	\end{block}

	\begin{block}{Problema 4}
		Determinar, de todos los usuarios actualmente no endémicos, cuales migraron fuera de la región endémica
		\begin{align*}
			Y^{(4)}_u =
			\begin{cases}
				& 1 \ \mbox{if} \ u \in ( EU_{0} \cap { EU_{1} }^{\complement})    \\
				& 0 \ \mbox{if} \ u \in ( { EU_{1} }^{\complement} / EU_{0}).
			\end{cases}
		\end{align*}

	\end{block}

\end{frame}

%%%%%%%%%%%%%%%%%%%%%%%%%%%%%%%%%%%%%%%%%%%%%%%%%%%%%%%%%%%%%%%%%%%%%%%%%%%%%%%%%%%%%%%%%%%%%%%%%%%%%%%%%%%%

\begin{frame}{ Proceso de experimentación }

Dado $(X,Y)$ y un clasificador $f_\theta \ t.q. \ \theta \in \Theta$
	\begin{enumerate}[I]
		% \item a
		% \item b
		\item Preprocesar $X$ y separar en $\trainset$ + $\testset$
		\item Cross validar hiperparámetros de $f$ c/ $\trainset$
		\item Aprender $f_\theta^*$ con $\trainset$
		\item Comparar  $Err_{\testset}(\theta^*)$ y  $Err_{\trainset}(\theta^*)$ en las métricas
		\item Evaluar mejoras en complejidad, varianza e iterar en (I)
	\end{enumerate}

\end{frame}

%%%%%%%%%%%%%%%%%%%%%%%%%%%%%%%%%%%%%%%%%%%%%%%%%%%%%%%%%%%%%%%%%%%%%%%%%%%%%%%%%%%%%%%%%%%%%%%%%%%%%%%%%%%%

\begin{frame}{Atributos en los datos}
	Procesamos los CDRs de $T_1$ para generar un conjunto de entrenamiento $(X,Y)$ donde $X \in \mathbb{R}^{n \times p}$ es una matriz con un usuario por fila e $Y \in \mathbb{R}^n$ es el vector de respuesta para la movilidad endémica de este usuario

	\bigskip

	\begin{block}{Diámetro de Movilidad}
		Tomamos el diametro en kilómetros de la cápsula convexa generada por todas las antenas que un usuario utiliza en $T_1$. Hicimos la distinción para los registros de horarios no laborales.
	\end{block}

\end{frame}

%%%%%%%%%%%%%%%%%%%%%%%%%%%%%%%%%%%%%%%%%%%%%%%%%%%%%%%%%%%%%%%%%%%%%%%%%%%%%%%%%%%%%%%%%%%%%%%%%%%%%%%%%%%%
\begin{frame}{Atributos en los datos 2}
	Procesamos los CDRs de $T_1$ para generar un conjunto de entrenamiento $(X,Y)$ donde $X \in \mathbb{R}^{n \times p}$ es una matriz con un usuario por fila e $Y \in \mathbb{R}^n$ es el vector de respuesta para la movilidad endémica de este usuario

	\bigskip

	\begin{block}{Atributos del grafo $\calG$ y de comunicaciones}
		En cada arista $\left< u, w \right> \in \calE$, desagregamos la interacción en base a múltiples criterios. Para cada mes generamos atributos a partir de la tupla  $\left< tiempo_{uw}, q\_llamados_{uw}, direcci on, vulnerabilidad, momento \right>$ donde $tiempo$ es el tiempo total en todos los llamados, $q\_llamados$ es la cantidad de llamados, $direccion$ es una variable booleana que indica si el llamado es entrante o saliente para $u$, $vulnerabilidad$ depende de si $w \in E_Z$ y $momento$ corresponde a una partición de la semana entre los días laborales, los fines de semana y los horarios no laborales
	\end{block}

\end{frame}


%%%%%%%%%%%%%%%%%%%%%%%%%%%%%%%%%%%%%%%%%%%%%%%%%%%%%%%%%%%%%%%%%%%%%%%%%%%%%%%%%%%%%%%%%%%%%%%%%%%%%%%%%%%%
\begin{frame}{Ejemplo de atributos}
Un ejemplo de dos atributos de $X$ procesdos a partir del grafo de comunicaciones
	\begin{table}[ht]
		\label{tab:data_example}
		\footnotesize
		\centering
		\resizebox{\textwidth}{!}{
			\begin{tabular} {|p{1.5cm}|p{1.5cm}|p{2cm}|p{1.5cm}|p{2cm}|p{1.5cm}|p{1cm}}
			% \begin{tabular} {|p{1.5cm} p{1.5cm} p{2cm} p{1.5cm} p{2cm} p{1.5cm} p{1cm}}
				%{l r r r r r r }
				% \toprule
				\hline
				\textbf{Nombre del atributo} & \textbf{Q\_llamados / Tiempo} & \textbf{Momento} & \textbf{Dirección} &
				\textbf{Vulnerabilidad} & \textbf{Mes} \\
				% \midrule
				\hline
				LLamados Fds EntrVul08    & Conteo & Final de Semana & Entrante & Interacciones vulnerables únicamente & August\\
				% \midrule
				\hline
				Tiempo LuVierNoche Sal12 & Tiempo de duración (s) & Durante la semana fuera de horario laboral & Saliente & Sin filtro endémico   & Diciembre \\
				\hline
				% \bottomrule
			\end{tabular}
		}
	\end{table}

\end{frame}




%%%%%%%%%%%%%%%%%%%%%%%%%%%%%%%%%%%%%%%%%%%%%%%%%%%%%%%%%%%%%%%%%%%%%%%%%%%%%%%%%%%%%%%%%%%%%%%%%%%%%%%%%%%%
\section{Resultados y Conclusiones}
%%%%%%%%%%%%%%%%%%%%%%%%%%%%%%%%%%%%%%%%%%%%%%%%%%%%%%%%%%%%%%%%%%%%%%%%%%%%%%%%%%%%%%%%%%%%%%%%%%%%%%%%%%%%


%%%%%%%%%%%%%%%%%%%%%%%%%%%%%%%%%%%%%%%%%%%%%%%%%%%%%%%%%%%%%%%%%%%%%%%%%%%%%%%%%%%%%%%%%%%%%%%%%%%%%%%%%%%%
\begin{frame}{Tabla de Experimentos A}

	\begin{table}[htp]\centering
	\footnotesize
		%\hskip-4.0cm
		\ra{1.3}
		%\hskip-4.0cm
		\begin{tabular}{@{}rr@{\hskip 0.3cm}r@{\hskip 0.3cm}r@{\hskip 0.3cm}rc@{}} \toprule
			&  \multicolumn{4}{c}{Problema 1} \\
			\cmidrule{2-5}
			& $Accuracy$ & $ROC AUC$ & $F1$ & $Runtime  (m)$ \\ \midrule
			$Logistic \ Regression$     & 0.893 & 0.857 & 0.9   & 96\\
			$Random \ Forest$            & 0.878 & 0.857 & 0.9  & 33 \\
			$GTB$ & 0.974 & 0.978 & 0.952 & 41  \\
			$Naive \ Bayes$               & 0.84  & 0.82  & 0.75  & 2  \\

			\bottomrule
		\end{tabular}

		%\addtocounter{table}{-1} % This dirty hack will fake both tables as if they were the same one.

		% \medskip
		%\hskip-4.0cm
		\ra{1.3}
		%\hskip-4.0cm
		\begin{tabular}{@{}rr@{\hskip 0.3cm}r@{\hskip 0.3cm}r@{\hskip 0.3cm}rc@{}} \toprule
			&  \multicolumn{4}{c}{Problema 2} \\
			\cmidrule{2-5}
			& $Accuracy$ & $ROC AUC$ & $F1$ & $Runtime  (m)$ \\ \midrule
			$Logistic \ Regression$     & 0.714 & 0.726 & 0.248 & 119 \\
			$Random \ Forest$            & 0.79  & 0.776 & 0.278 & 45  \\
			$GTB$ & 0.838 & 0.819 & 0.291 & 54 \\
			$Naive \ Bayes$               & 0.64  & 0.61  & 0.31  & 2   \\

			\bottomrule
		\end{tabular}
	\end{table}

\end{frame}


\begin{frame}{Tabla de Experimentos B}

	\begin{table}[htp]\centering
	\footnotesize
		\ra{1.3}
		%\hskip-4.0cm
		\begin{tabular}{@{}rr@{\hskip 0.3cm}r@{\hskip 0.3cm}r@{\hskip 0.3cm}rc@{}} \toprule
			&  \multicolumn{4}{c}{Problema 3} \\
			\cmidrule{2-5}
			& $Accuracy$ & $ROC AUC$ & $F1$ & $Runtime  (m)$ \\ \midrule
			$Logistic \ Regression$     & 0.705 & 0.754 & 0.307 & 115 \\
			$Random \ Forest$            &  0.792 & 0.845 & 0.346 & 21 \\
			$GTB$ & 0.811 & 0.855 & 0.359 & 33 \\
			$Naive \ Bayes$               & 0.65  & 0.63  & 0.45  & 1  \\

			\bottomrule
		\end{tabular}
		%\hskip-4.0cm
		% \medskip
		\ra{1.3}
		%\hskip-4.0cm
		\begin{tabular}{@{}rr@{\hskip 0.3cm}r@{\hskip 0.3cm}r@{\hskip 0.3cm}rc@{}} \toprule
			&  \multicolumn{4}{c}{Problema 4} \\
			\cmidrule{2-5}
			& $Accuracy$ & $ROC AUC$ & $F1$ & $Runtime  (m)$ \\ \midrule
			$Logistic \ Regression$     & 0.883 & 0.85  & 0.331 & 106 \\
			$Random \ Forest$            & 0.898 & 0.853 & 0.393 & 19 \\
			$GTB$ & 0.885 & 0.873 & 0.384 & 47 \\
			$Naive \ Bayes$               & 0.85  & 0.76  & 0.62 & 1 \\

			\bottomrule
		\end{tabular}
	\end{table}

\end{frame}



\begin{frame}{Resultados}
	
	\begin{block}{Facil Problema 1}
		Todas las metricas $\geq 75\%$ con la mejor en $97\%$
	\end{block}

	\begin{block}{Muy bajo $F1$}
		\begin{enumerate*}[label={\alph*)},]
			\item Desbalance de clases deriva en sobreconfianza
			\item Precision y ``recall'' asimetricos
			\item Sorpresa en ``Naive Bayes'' para ser hasta $50\%$ mejor en el Problema 4
			\item El Problema 2 es de mayor desbalance
		\end{enumerate*}
	\end{block}

\begin{block}{Inconvenientes en la ``Regresión Logística''}
	\begin{enumerate*}[label={\alph*)},]
		\item Lento: 110 min promedio frente a 1 min para ``Naive Bayes''
		\item Hasta un $12\%$ peor que $GBT$ 
	\end{enumerate*}
\end{block}

\end{frame}

%%%%%%%%%%%%%%%%%%%%%%%%%%%%%%%%%%%%%%%%%%%%%%%%%%%%%%%%%%%%%%%%%%%%%%%%%%%%%%%%%%%%%%%%%%%%%%%%%%%%%%%%%%%%

\begin{frame}{Mejores Atribuos}
	
	% \begin{block}{Facil Problema 1}
	% 	Todas las metricas $\geq 75\%$ con la mejor en $97\%$
	% \end{block}
	% \begin{block}{Muy bajo $F1$}
		Heuristica para algoritmos de ensamble
		
		\begin{enumerate*}[label={\alph*)},]
			\item Interacciones en el mes de Diciembre o en Agosto
			\item Interacciones vulnerables
			\item Diametro de Mobilidad
		\end{enumerate*}
% 	\end{block}

% \begin{block}{}
% 	\begin{enumerate*}[label={\alph*)},]
% 	\end{enumerate*}
% \end{block}

\end{frame}

%%%%%%%%%%%%%%%%%%%%%%%%%%%%%%%%%%%%%%%%%%%%%%%%%%%%%%%%%%%%%%%%%%%%%%%%%%%%%%%%%%%%%%%%%%%%%%%%%%%%%%%%%%%%

%%%%%%%%%%%%%%%%%%%%%%%%%%%%%%%%%%%%%%%%%%%%%%%%%%%%%%%%%%%%%%%%%%%%%%%%%%%%%%%%%%%%%%%%%%%%%%%%%%%%%%%%%%%%
% Conclusiones:


\begin{frame}{Conclusiones Esperadas}

	Descenso de temperatura desde el ``Gran Chaco'' hacia afuera
	% , indicando que las antenas descienden su porcentaje de usuarios \textit{vulnerables}

	\medskip
	Las interacciones vulnerables eran atributos relevantes 
	% de llamados desde y hacia la región endémica son relevantes para predecir migraciones
	% , lo cual confirma nuestra hipótesis utilizada para construir los mapas de calor

	\medskip
	% Fue posible construir clasificadores de 
	Alta performance en $Accuracy$ y $ROC AUC$ 
	% para detectar cuáles usuarios migraron desde la región endémica
	% , sobre todos los usuarios actualmente no endémico

	\medskip

	Utilizaci\'on novedosa de los CDRs 
	% es novedosa de los CDRs y se puede extender hacia otras epidemias

\end{frame}

%%%%%%%%%%%%%%%%%%%%%%%%%%%%%%%%%%%%%%%%%%%%%%%%%%%%%%%%%%%%%%%%%%%%%%%%%%%%%%%%%%%%%%%%%%%%%%%%%%%%%%%%%%%%
%%%%%%%%%%%%%%%%%

\begin{frame}{Conclusiones Inesperadas}

	Mapas correlacionan con migraciones 
	\medskip
	++ ensamble
	% en estrategias que no son intrusivas a las personas
	\medskip
	-- precision y $F1$ en general
	\medskip
	Los meses tambien fueron atributos relevantes

	% tres tipos de atributos más relevantes para la clasificaición de los migrantes estaban relaciónadas con:
	% \begin{enumerate*}[label={\alph*)},]
	% 	\item la movilidad del usuario
	% 	\item las interacciones y patrones de llamados con vecinos vulnerables, y
	% 	\item uso celular en los meses más cercanos a $T_0$, así como también la actividad en el mes de Diciembre.
	% \end{enumerate*}

\end{frame}

%%%%%%%%%%%%%%%%%%%%%%%%%%%%%%%%%%%%%%%%%%%%%%%%%%%%%%%%%%%%%%%%%%%%%%%%%%%%%%%%%%%%%%%%%%%%%%%%%%%%%%%%%%%%
%%%%%%%%%%%%%%%%%

% \begin{frame}{Inconvenientes}

% 		\begin{block}{Sesgos en los datos}
% 			\begin{itemize}
% 			\item Basados únicamente en datos de una sola TelCo por país que no capturan migraciones internacionales
% 			%La caracterizaci\'on actual de la vulnerabilidad viene dada por contacto telefónicos con usuarios de antenas del Gran Chaco.
% 			\item Concentración geoespacial de usuarios

% 			\item Mobilidad no implica prevalencia

% 			\item Estacionalidades en las migraciones

% 			\end{itemize}
%  	 	\end{block}
% \end{frame}


  %%%%%%%%%%%%%%%%%%%%%%%%%%%%%%%%%%%%%%%%%%%%%%%%%%%%%%%%%%%%%%%%%%%%%%%%%%%%%%%%%%%%%%%%%%%%%%%%%%%%%%%%%%%%

%  \begin{frame}{Lineas Futuras de Trabajo}

	\begin{block}{Extensiones}
		
		\begin{enumerate*}[label={\alph*)},]
			\item + clasificadores 
			\item Metodos para desbalance
			\item + ``best features'' 
			\item Feature Engineering
			\item Estacionaridad
			\item datos de serologia u ruralidad
		\end{enumerate*}

	\end{block}
% 	\medskip
% 	\begin{block}{Nuevas fuentes de datos}
% 		Agreagar informacion para mayores analisis:
% 		\begin{itemize}
% 			\item estimación de resultados de movilidad con serologia por region
% 			\item Categorizacion de antenas por grado de ruralidad del ambiente
% 			\item infected newborns
% 			\item acute Chagas cases.
% 			\item serological data per community, amongst others.
% 		\end{itemize}
% 	\end{block}

 \end{frame}

%%%%%%%%%%%%%%%%%%%%%%%%%%%%%%%%%%%%%%%%%%%%%%%%%%%%%%%%%%%%%%%%%%%%%%%%%%%%%%%%%%%%%%%%%%%%%%%%%%%%%%%%%%%%
% ==============================================

\begin{frame}{Gracias! }
			\begin{block}{Companeros de proyecto}
					\center\
					Alejo Salles (Instituto de Cálculo-UBA) \\
					Carlos Sarraute (GranData Labs) \\
					Carolina Lang (UBA, GranData Labs) \\
					 Diego Weinberg (Fundación Mundo Sano) \\
	\end{block}

		% \begin{column}{0.55 \textwidth}
				\center\
				Juan Mateo de Monasterio \\
				\textit{laterio@gmail.com} \\
				Preguntas? \\
		% \end{column}

\end{frame}



% ==============================================

\justifying%
\scalefont{0.7}
% \bibliographystyle{unsrtnat}
% \bibliography{./}

% \end{columns}
% ==============================================

\vfill

\end{document}




%   ___  _     ___         ___________ __ __ _____ _____
%  /   \| |   |   \       / ___/      |  |  |     |     |
% |     | |   |    \     (   \_|      |  |  |   __|   __|
% |  O  | |___|  D  |     \__  |_|  |_|  |  |  |_ |  |_
% |     |     |     |     /  \ | |  | |  :  |   _]|   _]
% |     |     |     |     \    | |  | |     |  |  |  |
%  \___/|_____|_____|      \___| |__|  \__,_|__|  |__|

%%%%%%%%%%%%%%%%%%%%%%%%%%%%%%%%%%%%%%%%%%%%%%%%%%%%%%%%%%%%%%%%%%%%%%%%%%%%%%%%%%%%%%%%%%%%%%%%%%%%%%%%%%%%

%\begin{frame}{Objective: Chagas and Migrations}
%	Our goal is to find those traditionally non-endemic places which have most of the migratory exchange with the ecoregion.
%	El objetivo es encontrar las zonas del país donde exista más intercambio migratorio con la zona del Gran Chaco.
%
%	\bigskip
%	We aim to find infected people living in areas which are traditionally non-endemic.
%	Living outside \textit{Gran Chaco} and carrying the disease parasite \textit{trypanosoma cruzi} that causes Chagas.
%
%	\medskip
%	Having a disease with long asymptomatic phases imply that long-term migrations are relevant to analyze.
%
%	El período durante el cual se puede estar enfermo sin síntomas es tan alto
%	que se consideraron interesantes las migraciones largas.
%
%	\medskip
%
%	\begin{block}{The goal is to find...}
%		\begin{itemize}
%			\item individuals which have been infected in endemic areas and have later migrated.
%			gente que haya contraído la enfermedad en zonas endémicas y luego haya migrado.
%					\item ...personas que hayan nacido con ella (cuyos familiares/contactos pueden haber sido migrantes).	% Polémico? %naa
%			\item seasonal workers whose residence varies during the year.
%			trabajadores golondrina, que cambien su residencia\\ de acuerdo a la estación del año.
%			\item outlying communities with probability of having a higher risk of Chagas prevalence.
%			comunidades destacadas donde posiblemente exista una alta prevalencia de Chagas.
%		\end{itemize}
%	\end{block}
%
%\end{frame}

%%%%%%%%%%%%%%%%%%%%%%%%%%%%%%%%%%%%%%%%%%%%%%%%%%%%%%%%%%%%%%%%%%%%%%%%%%%%%%%%%%%%%%%%%%%%%%%%%%%%%%%%%%%%
