\section{Introduction}

Chagas is a tropical parasitic disease found in mo disease mainly epidemicAfter more than 30 years of Chagas vector  disease is an epidemic pedro said this 

Aun despues de mas de 30 años de aplicar programas de control vectorial, el Mal de Chagas

sigue siendo una enfermedad con altísima prevalencia en la Argentina. Según datos del

Ministerio de Salud se estiman mas de un millón de personas infectadas y siete millones de

individuos expuestos a esta enfermedad a nivel nacional. Hoy en día son especialmente

relevantes el contagio por transfusión de sangre y el de madre a hijo durante el embarazo,

estimándose que nacen cada año al menos 1300 niños infectados.

Por otra parte, existen en la literatura varios trabajos (ver por ejemplo (6,7)) que analizan

movimientos migratorios humanos a partir de registros de telefonía celular utilizando la

geolocalización de las llamadas. Estos registros, llamados CDRs por sus siglas en inglés (Call

Detail Record), también son utilizados para inferir propiedades y características sociales de

cada usuario tales como la zona de residencia, el entorno social, las zonas por las que transita,

su edad y género, etc.

El objetivo de este trabajo consiste en generar una predicción para la localización de posibles

infectados por el Trypanosoma cruzi (parásito causante del Mal de Chagas) a nivel nacional

basada en un gran volumen de registros de llamados celulares. Estos datos serán utilizados

para aportar información acerca de la dinámica social de la población Argentina, tanto a nivel

2

individual como a nivel agregado. Los datos contienen cinco meses de llamados desde

noviembre de 2011 para mas de 40 millones de usuarios.

1

