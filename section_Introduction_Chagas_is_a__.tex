\section{Introduction}

Chagas is a global tropical parasitic disease mostly epidemic Latin America. WHO \ref{http://www.who.int/mediacentre/factsheets/fs340/en/} estimate more than six million infected people worldwide. Most transmissions occurs in the Americas via the \textit{Trypanosoma cruzi} parasite, vector-borne by the \textit{Triatomine} insect family.

In Argentina vector control campaigns have been ongoing for more than thirty years as the main epidemic counter-measure. National estimates still indicate that there exist at least one million people carrying the parasite with more than seven million exposed.
Other relevant routes of transmission include blood transfusion and congenital transmission, estimating 1300 newborns infected each year \ref{trabajos_de_}.\begin{comment}  en el drive estan las ppt del min salud \end{comment}

The dynamic interaction of the triatomine infested areas and the human mobility patterns presents a difficult scenario to track down individuals or spots with high prevalence of infected people. Other than complex and costly individual screenings, the authyors of this work 


Por otra parte, existen en la literatura varios trabajos (ver por ejemplo (6,7)) que analizan

movimientos migratorios humanos a partir de registros de telefonía celular utilizando la

geolocalización de las llamadas. Estos registros, llamados CDRs por sus siglas en inglés (Call

Detail Record), también son utilizados para inferir propiedades y características sociales de

cada usuario tales como la zona de residencia, el entorno social, las zonas por las que transita,

su edad y género, etc.

El objetivo de este trabajo consiste en generar una predicción para la localización de posibles

infectados por el Trypanosoma cruzi (parásito causante del Mal de Chagas) a nivel nacional

basada en un gran volumen de registros de llamados celulares. Estos datos serán utilizados

para aportar información acerca de la dinámica social de la población Argentina, tanto a nivel

individual como a nivel agregado. Los datos contienen cinco meses de llamados desde

noviembre de 2011 para mas de 40 millones de usuarios.

1