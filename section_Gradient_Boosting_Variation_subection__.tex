\section{Gradient Boosting Variation}

\subection{Boosting Trees    }

$\Theta$

The objective function differences minimization objectives. 

Predictive power: how well the algorithm fits the data used for training and, hopefully, the \textit{true} underlying distribution

+ Regularization, favors parsimonious models. This is because all models approximate natural/processes to some degree, so if a simpler model can be used with the same predictive power, then this model should be used.  

\begin{equation} \label{eq:objFunction}
Obj(\Theta) \ = \ L(\Theta) + R(\Theta)
\\
%\sum_{i=0}^{\infty} a_i x^i
\end{equation}

When building a model of ensemble trees, a higher model will be learning on other \textit{weaker} decision trees. If $\calG$ is a set of tree models, then 
\[    y = \sum_k f_k(x) , \ f \in \calG \]

\subection{sklearn}


\subection{Hastie Tibshiranie Friedman}

