%===============================================================================
%     File: ch4_evaluation_results.tex
%    Author: Juan de Monasterio
%    Created: 15 Feb 2017
%  Description: Chapter: Evaluation and Results
%===============================================================================

\chapter{Summary of Results and Conclusion}\label{ch:results_conclusion}


The idea of this chapter is to present a combined overview of the work done and results we encountered in \cref{ch:descr-risk, ch:machineLearning, ch:modelSelection, ch:ensembleMethods}. All in all, we
 and Conclusion at the end.

As a summary of the work done we can say we have discussed issues on the most of the following items:

%\cref{ch:descr-risk}
%\cref{ch:machineLearning}
%\cref{{ch:modelSelection}
%\cref{ch:ensembleMethods}

%\cref{}

\begin{itemize}
    \item
    Predict a user's province when given information on only the first four features.
    \item Predict a user's number of calls made on weekends when given information on the last four features.
    \item Give an estimate of the probability density function of user's calls duration, during weekdays.
\end{itemize}

\todo{Reword sentences. Expand, correct and finish them in a cohesive structure.}


\subsection{Some nice sentences}



The overall picture shows that the hypothesis stipulates that there are social interactions captured by the cellphone usage logs, which is informative on long-term migrations and disease spread.


The experiments discussed in previous sections highlight the benefit of using these types of data.
The Argentinean and Mexican case studies allow to reach an inisght on to whether it is possible to characterize human movements of long durations.
CDRs are rich enough in information to detect detailed patterns of users moving to and from a particular area, at a national level.



Some supervised classification models realized the importance of attributes tagged with vulnerable behavior obtained from the CDR.



CDRs can be extensively used for non business purposes and, specifically in this case, there is light evidence that they might capture well the fluxes of diseases among humans.
This data can be used as a surrogate to other, more traditional, methods of disease control and surveillance.
This work intends to help authorities with disease-control strategies in way which is not intrusive for cellphone users.
Measures can then be applied outside of the endemic region, and directed towards specific neighbourhoods and communities by using the recommendations output by the risk algorithm.





extensive research of machine learning literature and CDR usage in health problems.
handling data, models and features with acumen
focus on the strengths of the ensemble learners to achieve higher predictions
try and draw the greates prediction performance from the features


Through a number of experiments, we demonstrated how the supervised classifaction models performed across different prediction tasks.
These case test sheds light on how relevant this dataset is to the question on complex long-term migrations.
Their complexity and scale of these movements can be, up to a certain degree, assimilated by probabilistic algorithms, using the models outlined before.
In this owrk we took advantatge of the large user scale and temporal range of the CDRs allowed us to capture a variety of human movements throughout the country.
Migration patterns can then be used to suggest actual epidemic spread of this disease into areas not deemed endemic.


There are also weak points with this methodology. We can not jump directly to a disease spread conclusion when we are only looking

Bias in the sample of the country's population that is part of our dataset.
Still, these results should be interpreted with caution, since we have no real comparisons with actual disease data, outside of the endemic region and at a municipal or regional level.\footnote{As a matter of fact, we have tried to contact hearlth institutions and organizations working on the matter to try and enrich the dataset with more info. This intent did not reach any health data of the kind.}

There are other interesting limitations concerning international pattern migrations.
For this work, we measured migrations only in national phone owners.
And it was showed in \cref{ch:descr-risk}, that  epidemic regions extend political borders.
With the current dataset we cannot capture international migrations and our analysis is limited to movements of national Telco users only.
The same applies for detecting movement patterns of people which lack cell phone antennas.

In extracting each user's home antenna, based on the user's ``working hours'', we also made the assumption that this can be used in a generalized case.
Even though this decision was supported by past research, it might not be easily translated into datasets of other geographical and cultural regions.

Time seasonalities and sationarity was considered but only in a limited fashion.
Weekends and working hours were separately tagged in attributes.
The same with month specific attributes for the user's accalling patterns.
However there are mor long-term time segmentation which were not used in this work.
It is possible that migration patterns be strongly related to seasonal factors such as, for example, in the agriculture workforce.

% discriminated in the construction of the features, yet there were no long-term time


The resulting analysis, allowed to show that the Risk maps we developed are of interest to national health campaigns.
They pintpoint to likely spots of high disease prevalence, with a low cost to produce this outcome.

Next to traditional health control actions, this research may provide a very insightful piece of information for decision-makers.
Capturing long-term human movements will also enable better simiulations of disease spreads in different contexts.
In this work, the impact of this in our predictive capacity of disease prevalence is shown.





We believe it is acceptable to say that most human dynamics in the form of migration patterns are captured by the use of this data.
This work was performed in two Latin-American countries and in one single endemic region.
We expect that this CDR usage can be extended to other similar countires and diseases, with demographic, cultural and geographical similarities.


Feature selection of ensemble algorithms were concurring in the choice of the best features for the prediction task.
Human mobility, vulnerable calling patterns, interactions in earlier months and in december, etc.
Model regularization and hyperparameter cross validation improved the models predictive power.
Also, the effects of overfitting the models were visible in some of the experiments where deep trees and complex configurations underpowered the resulting model's generalization error.



High discrepancies exist in the scores of the different machine learning models.
Some are better at capturing the relevant information to predict human movements in the long-term.
This is becaause not all social information is presented directly in the dataset, and feature interactions are helpful.


The importance of other user specific features, such as demographics, should be examined as well.
Yet these are out of the scope of this work, for the lack of data.


This work may give us some insights into future disease spreads in other similar contexts.
Where
It also helps to showcase interesting ways of using the CDRs for reasons other than logging a user\'s calling expenses.


\subsection{Descriptions of key results}


\todo{Rewrite all key results of the experiments.}

The features most brought up in the experiments of ensembel learners where A, B, C .
These indicated this and that.
They also pointed to information in.. bla


The best learners for the following tasks where A,B.
We saw differences in these and that type of learners, where the former was better at this score, whilst the latter at the second.

The Random Forest model played an important role in having the highest scores across all tasks and in presenting the best features of the datset.
This information was later availed by the boosting models trained on said set of features.


We wrote an introdcutory review for a list of supervised classification algorithms and then applied them in the preceding chapters to our tasks.
These models presented are some of the most common techniques found in the literature for the classification task we were trying to solve.



The heatmaps shown in Section~\cref{ch:descr-risk} expose a `temperature' descent from the core regions outwards.
The heat is concentrated in the ecoregion and gradually descends as we move further away.
 This expected behavior could be explained by the fact that calls are in general of a local nature and limited to 3 or 4 main antennas used per user.

Another fact found in this work is of certain communities atypical to their neighboring region.
They stand out for their strong communication ties with the studied region, showing significantly higher links of vulnerable communication.
The detection of these antennas through the visualizations was of great value to the health researchers participating in this project.

In \cref{ch:ensembleMethods, ch:modelSelection, ch:machineLearning}, we introduced and evaluated a systematic process that tackles the problem of predicting long-term migrations.
In particular, we showed that it is possible to use the mobile phone records of users during a bounded period (of 5 months) in order to predict whether they have lived in the endemic zone $E_Z$ in a previous time frame (of 19 months).
The results obtained demonstrate that CDRs are particularly well suited for this task.

To conclude, the results presented in this work show that it is possible to explore CDRs as a mean to tag human mobility.
Combining social and geolocated information, the data at hand has been given an innovative use, different from its original billing purpose.

Epidemic counter-measures nowadays include setting national surveillance systems, vector-centered policy interventions and individual screenings of people. These measures require costly infrastructures to set up and be run. However, systems built on top of existing mobile networks would demand lower costs, taking advantage of the already available infrastructure. The potential value these results could add to health research is hereby exposed.
Finally, the results stand as a proof of concept which can be extended to other countries or to diseases with similar characteristics.





% Algorithms: SVM, Random Forest and Logistic Regression, Multinomial Naive Bayes



% L-BFGS is the optimization routine. limited memory Broyden–Fletcher–Goldfarb–Shanno
%(reference the limited memory Broyden–Fletcher–Goldfarb–Shanno (BFGS) algorithm \url{https://en.wikipedia.org/wiki/Limited-memory_BFGS} )
%% es un algoritmo de optimizacion numerica para computar minimos de funciones


% % With a running time of 80s, 10 steps are generally needed to achieve a 0.977 validation accuracy score on
% The best model was a Logistic Regression Classifier with an $L2$-penalty value of 0.01.
% % Table~\cref{ts}
% The following table
% shows the scores obtained by the selected model on the out-of-sample set.

% \begin{center}
% 	\begin{tabular}{ l l }
% 		\toprule
% 		Score & Value \\
% 		\midrule
% 		F1 score & 0.964537  \\
% 		Accuracy & 0.980670  \\
% 		AUC    & 0.991593  \\
% 		Precision & 0.970838  \\
% 		Recall  & 0.958316  \\
% 		\bottomrule
% 	\end{tabular}
% \end{center}


% \begin{table}\label{tab:results}[ht]
% 	\caption{Resulting scores.}

% 	\centering
% 	\begin{tabular}{ l l }
% 		\toprule
% 		Score & Value \\
% 		\midrule
% 		F1 score & 0.964537  \\
% 		Accuracy & 0.980670  \\
% 		AUC    & 0.991593  \\
% 		Precision & 0.970838  \\
% 		Recall  & 0.958316  \\
% 		\bottomrule
% 	\end{tabular}
% \end{table}

% High values across all scoring measures are achieved. % with the best estimator, with the lowest metric starting at a value of 0.958.
% These results can be explained by the fact that
% %Scores results become less surprising when looking closer at the dataset.
%
%Si del test\_set ahora me quedo con lo users que Y\_target ==1 == EPIDEMIC\_gt pero que hoy en dia son epidemic ==0 (se mudaron). Los scores bajan mucho:
%
%+--------------+-----------------+-------+
%| target\_label | predicted\_label | count |
%+--------------+-----------------+-------+
%|   1    |    1    | 1944 |
%|   1    |    0    | 3525 |
%+--------------+-----------------+-------+
%'f1\_score': 0.5244840145690004, ''accuracy': 0.3554580,, 'precision': 1.0, 'recall': 0.35545803,



\section{Conclusions}\label{section:conclusions}

ASDF ASDF


\subsection{ Future Work Lines}

The mobility and social information extracted from CDRs analysis has been shown to be of practical use for long-term human migrations and for Chagas disease research.
it adds value and information to help make data driven decisions which in turn is key to support epidemiological policy interventions in the region.
For the purpose of continuing this line of work, the following is a list of possible extensions:

\begin{description}
    \item [Results validation.] Compare observed experiment results of risk maps and best features against actual serology or disease prevalence surveys.
    Data collected from fieldwork could be fed to the algorithm in order to supervise the learning towards a target variable that is defined by the disease prevalence at a certain level.

    \item [Differentiating rural antennas from urban ones.] This is important as studies show that rural areas have epidemiological conditions which are more favorable to the expansion of the disease expansion.
    The vector-borne transmission through \textit{Trypanosoma cruzi} is helped with poor housing materials and domestic animals.
    All  contribute to complete the parasite's lifecycle.
    Using the CDR data, antennas could be automatically tagged as rural by analyzing the differences between the spatial distribution of the antennas in each area.
    A similar goal could be to identify precarious settlements within urban areas, with the help of census data sources.

    \item [Seasonal migration analysis.] Experts from the \textit{Mundo Sano} Foundation underlined that many seasonal migrations occur in the \textit{Gran Chaco} region.
    Workers are known to leave the endemic area for several months possibly introducing the parasite to foreign populations.
    The analysis of these movements can give information on which communities have a higher influx of people from the endemic zone during a certain period.
    %\item Add more regions to the analysis.

    \item [Search for epidemiological data at a detailed level.] For instance, specific historical infection cases.
    Splitting the endemic region according to the infection rate in different areas, or considering particular infections.
    With the use of a high-quality epidemic dataset, we assume that a more model complex can be built, to detect infected users nationwide.
    It is of primary importance though to have a good dataset availabe.
    Since otherwise, it would not be possible to reach a generalizing classifier.

    \item Feature exploration to search for correlations with being infected.

\end{description}



% \subsection{Supervised Algorithms}
\subsection{Brief Technical Notes }


Results were satisfactory in most algorithms such as Gradient Boosting, Random Forest and Logistic Regression.
The Multinomial Naive Bayes method was used solely for benchmarking reasons and for its fast runtime and low number of parameters required to configure.
All algorithms were run on a 16-core Linux machine with 72GB of RAM.\@ Processing and learning scripts were run on Python.
When available, we showed how the feature importance methods qualified the attributes of the dataset, selecting the most relevant for the experiment.

\todo{add python, graphlab, sklearn, pandas citations}.

The project involved working with more than 2 terabytes of raw data which had to be extracted, processed, transformed, loaded and classified as demanded.
To do this, most of the project's code was built using existing Machine Learning and Data Analysis frameworks for Python.
For the latter part of the job, we used Pandas and Graphlab whilst for the classification part it was mostly implemented with Scikit-learn~\textcite{scikit-learn} and, in some cases, with Graphlab~\textcite{graphlab}.

We can not recommend highly enough all the aforementioned python modules.
All were critical in the development of a working experimentation pipeline.




