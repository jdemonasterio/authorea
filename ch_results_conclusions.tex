%!TEX root = main.tex

%===============================================================================
%     File: ch4_evaluation_results.tex
%    Author: Juan de Monasterio
%    Created: 15 Feb 2017
%  Description: Chapter: Evaluation and Results
%===============================================================================

\chapter{Summary of Results and Conclusion}\label{ch:results_conclusion}

\todo{This is work in progress.}

%\cref{ch:descr-risk}
%\cref{ch:machineLearning}
%\cref{{ch:modelSelection}
%\cref{ch:ensembleMethods}
%\cref{}

% handling data, models and features with acumen
% After extensive research of machine learning literature and CDR usage in health problems.

The idea of this chapter is to present a combined overview of the work done and results we encountered in
\cref{ch:descr-risk,ch:machineLearning,ch:modelSelection,ch:ensembleMethods}.
All in all, we intend to showcase most of the relevant information from the preceding sections.
The overall picture of this work shows that there is evidence to confirm the hypothesis which stipulates that there are social interactions captured by the cellphone usage logs.
This is because the data-set is informative enough on long-term migrations and human mobility.
To summarize the work done we can say that:
% A summary of the work done we can say we have discussed issues on the most of the following items:


\begin{itemize}

    \item We presented a visualization of the datasets with heat maps that showed georeferenced call interactions and colored cellphone antennas according to their level of vulnerable interactions.
    The maps produced showed results that were coherent with the expert's knowledge of the endemic zone's migration patterns.
    Added to this, they produced many interesting insights of specific antennas that have higher vulnerable interactions with the endemic region, where these are located outside of the endemic region.

    \item We applied a systematic approach to analyze the information contained in the dataset.
    The idea was to try to explore the human mobility patterns, as captured by the phone call records.

    \item We broke the problem of long-term human migrations down to four tasks which were analyzed separately.
    To perform this analysis, we introduced machine learning theory and methods.
    A discussion of these methods' applications to these kinds of tasks was elaborated.
    The models presented are some of the most common techniques found in the literature for the classification task we were trying to solve.

    \item We compared the strengths and weaknesses of these methods.
    The algorithms  were used to estimate the probability of each user's migrations.
    Also, we showcased the method's results differences for the tasks defined in \cref{ch:machineLearning}.

    \item Where possible, we exposed the best features of the dataset to solve the general problem.
    These exposed the relationship between the attributes that characterize vulnerable user interactions, and the target variable of the task.
    Other relevant features, such as the user's mobility size, were also found to provide relevant information for the classification task.
     % the algorithms detected that the characterization of the user's mobility provided relevant information to the task
It was important in detecting the long-term movement of users.

    \item According to our results, there are algorithms which are reliable enough to detect users which have migrated from old regions
In some instances, we have achieved high values across all scoring measures.

\end{itemize}

From our findings, we can say that there is no single method that can be single-handedly applied to all problems.
Each of the tasks will have different scores across the algorithms used.
% and the same applies to the metrics used to evaluate them.
However, the results show that the performance of Random Forests or Gradient Boosting models is best over other models such as Logistic Regression or Naive Bayes algorithms.
The best models can find complex interactions in the data whilst taking care not to lose generalization power.

\todo{Reword sentences
Expand, correct and finish them in a cohesive structure.}

\section{Key Results}

\todo{Rewrite all key results of the project.}


In this work on CDRs, the Argentinean and Mexican case studies allow us to find that it is possible to characterize human movements of long duration.
Where cellphone datasets are rich enough in information to detect detailed patterns of users moving to and from a particular area, at a national level.




The heatmaps shown in Section~\cref{ch:descr-risk} expose a `temperature' descent from the core regions outwards.
The heat is concentrated in the ecoregion and gradually descends as we move further away.
This expected behavior could be explained by the fact that calls are in general of a local nature.
Where users, in general, have an antenna usage limited to 3 or 4 Telco antennas used during the period analyzed.

The risk maps highlight that the interactions from non-endemic antennas towards those in the endemic region are non-homogeneous.
As an example, \cref{fig:amba_map} outlines various antennas with higher vulnerability, which would help detect communities with higher probability of disease prevalence.
These anomalies found could point to potential communities atypical of their neighboring region.
They stand out for their strong communication ties with the regions studied, showing significantly higher links of vulnerable communication.
In the images presented, the differences in the vulnerable interactions are clear.
The detection of these antennas through the visualizations was of great value to the health Mundo Sano Foundation researchers participating in this project.






In \cref{ch:machineLearning,ch:modelSelection,ch:ensembleMethods}, we introduced and evaluated a systematic process that tackles the problem of predicting long-term migrations.
As a start, we wrote an introductory review for a list of supervised classification algorithms.

Then, through a number of experiments, we demonstrated how the supervised classification models performed across different prediction tasks.
For each model, we tried to draw the greatest prediction performance from the features extracted from the dataset.

Results evidence indicating that a supervised classification predictor can be built to detect large patterns of human migrations over time.
They also point to the fact that CDRs are particularly well suited for this task.
Where it is possible to explore CDRs as a mean to tag human mobility.


The supervised classification experiments done on the various prediction tasks were discussed in previous sections.
The results highlighted the benefit of using this type of data for the problem.


% in the preceding chapters to our tasks.
The best learners for the following tasks were the Random Forests and Gradient Boosting models.
Due to this, we focused our work on the strengths of the ensemble learners to achieve higher predictions.
Still, we saw differences in both ensemble methods, where the former was better at this score, whilst the latter at the second.


The features most brought up in the experiments of ensemble learners where A, B, c.
These indicated this and that.
They also pointed to information in  blablabla.
Some supervised classification models realized the importance of attributes tagged with vulnerable behavior obtained from the CDR.\@

The Random Forest model played an important role in having the highest scores across all tasks and in presenting the best features of the dataset.
This information was later availed by the boosting models trained on said set of features.



With this, we showed that it is possible to use the mobile phone records of users during a bounded period (of 5 months) in order to predict whether they have lived in the endemic zone $E_Z$ in a previous time frame (of 19 months).

Combining social and geolocated information, the data at hand has been given an innovative use, different from its original billing purpose.

% To conclude, the results presented in this work show that



Epidemic counter-measures nowadays include setting national surveillance systems, vector-centered policy interventions and individual screenings of people
These measures require costly infrastructures to set up and be run
However, systems built on top of existing mobile networks would demand lower costs, taking advantage of the already available infrastructure
The potential value these results could add to health research is hereby exposed.
Finally, the results stand as a proof of concept which can be extended to other countries or to diseases with similar characteristics.



\section{In-depth discussion}


% - hacer explicita la respuesta a esta pregunta: Que tanto se puede predecir long term migrations usando datos de CDRs?
% - cuales fueron las mejores tecnicas? Que tuvieron a favor y en contra?
% - repasar los 5 problemas y resumir los mejores resultados de c/ algoritmo
Que scorings tuvieron?
% - Enganchar los resultados con lo que se vio en los mapas
Hablar de como los features del algoritmo estan relacionados con el armado del mapa y como se "valida" la hipotesis.





These test case shed light on how relevant this dataset is to the question on complex long-term migrations.
Their complexity and scale of these movements can be, up to a certain degree, assimilated by probabilistic algorithms, using the models outlined before.


In this work we took advantage of the large user scale and temporal range of the CDRs allowed us to capture a variety of human movements throughout the country.
Migration patterns can then be used to suggest actual epidemic spread of this disease into areas not deemed endemic.




\section{Drawbacks on the methodology}

Several weak points can be made on our methodology.
An important observation is that we can not directly jump these conclusions on-to a disease spread model.
In this work, we are looking solely at human mobility to and from endemic regions.
This does not imply disease spread or prevalence.
Yet, on the advice of this topic's researchers, this work does bring valuable insights to the problem.


Another issue is the representation bias of our dataset.
As shown in \cref{tab:distribution_by_state}, there are differences in the percentage of population represented by our dataset vs.\ current state distribution estimates.
With this, results should be interpreted with caution, since we have no real comparisons with actual disease data, outside of the endemic region and at a municipal or regional levels.\footnote{As a matter of fact, we have tried to contact health institutions and organizations working on the matter to try and enrich the dataset with more info
This intent did not reach any health data of the kind.}

There are other interesting limitations concerning international pattern migrations.
For this work, we measured migrations only in national phone owners.
And it was showed in \cref{ch:descr-risk}, that  epidemic regions extend political borders.
With the current dataset we cannot capture international migrations and our analysis is limited to movements of national Telco users only.
The same applies for detecting movement patterns of people which lack cell phone antennas.

In extracting each user's home antenna, based on the user's ``working hours'', we also made the assumption that this can be used in a generalized case.
Even though this decision was supported by past research, it might not be easily translated into datasets across other geographical and cultural regions.

Finally, time seasonalities and stationarity, were considered but only in a limited fashion.
Weekends and working hours were separately tagged in attributes.
The same with month specific attributes for the user's calling patterns.
However there are more long-term time segmentation which were not used in this work.
It is possible that migration patterns be strongly related to seasonal factors such as, for example, in the agriculture workforce.

The importance of other user specific features, such as demographics, should be examined as well.
Yet these are out of the scope of this work, for the lack of data.


% discriminated in the construction of the features, yet there were no long-term time






The resulting analysis, allowed to show that the Risk maps we developed are of interest to national health campaigns.
They pinpoint to likely spots of high disease prevalence, with a low cost to produce this outcome.

Next to traditional health control actions, this research may provide a very insightful piece of information for decision-makers.
Capturing long-term human movements will also enable better simulations of disease spreads in different contexts.
In this work, the impact of this in our predictive capacity of disease prevalence is shown.



We believe it is acceptable to say that most human dynamics in the form of migration patterns are captured by the use of this data.
This work was performed in two Latin-American countries and in one single endemic region.
We expect that this CDR usage can be extended to other similar countries and diseases, with demographic, cultural and geographical similarities beyond Chagasic spread in Mexico or Argentina.


Feature selection of ensemble algorithms were concurring in the choice of the best features for the prediction task.

Model regularization and hyperparameter cross validation improved the models predictive power.

Also, the effects of overfitting the models were visible in some of the experiments where deep trees and complex configurations underpowered the resulting model's generalization error.

Feature importance with tree ensemble gave us insights into which CDR information provided
predictive value.
The best features can be broken down into the following categories.

\begin{itemize}

    \item Calls made in older months and in December: Interactions of different type, that occured closer to the split month between both periods $T_0$ and $T_1$ (July), were also distinguished.
    The model frequently used interactions from the months of August and September, as well as December.
    We conjecture that this last result can be due to the fact that December is a month where user\'s increase their mobile activity with their families.

    \item Vulnerable calling patterns: As we first suspected in \cref{ch:descr-risk}, the vulnerability of the mobile interaction between users was relevant.
    Measurements such as the duration and volume of calls at different time periods where selected.
    We make the reminder here that these measurments were the basis of our construction of the heatmaps in that same chapter.
    Where the antennas plotted concentrated users with vulnerable calls by calling volume or time, for a given month.

    \item Mobility size: Finally, different measures of the distance in human mobility were
determined to be important by the model.
Both general mobility diameters and weekday or weeknight specific mobility endemic were relevant.

\end{itemize}

These gruops identify strong predictors in long-term human mobility and provide additional insights into our original hypothesis of which predictors were most valuable for the task.
Understanding these factors can further provide information into the problem of chagasic disease spread in the long run.
A more thorough examination of these results can be seen in \cref{tab:random_forest_big_experiment_best_features, tab:boosting_big_experiment_best_features}.


High discrepancies exist in the scores of the different machine learning models.
Some are better at capturing the relevant information to predict human movements in the long-term.
This is because not all social information is presented directly in the dataset, and feature interactions are helpful.


This work may give us some insights into future disease spreads in other similar contexts.
Where
It also helps to showcase interesting ways of using the CDRs for reasons other than logging a user\'s calling expenses.



The implications of these results argue that CDRs can be extensively used for non business purposes and, specifically in this case, there is light evidence that they might capture well the fluxes of diseases among humans.
This data was used as a surrogate to other, more traditional, methods of disease control and surveillance.
This work intends to help authorities with disease-control strategies in way which is not intrusive for cellphone users.
Measures can then be applied outside of the endemic region, and directed towards specific neighbourhoods and communities by using the recommendations output by the risk algorithm.


\cref{tab:all_results} shows a compendium of the results obtained from the CDR dataset, using all classification models.
Due to time constraints limited by the use of full cross validation fitting procedures, not all models were run for all of the tasks.
However, where available, we show the resulting classifier's test performance across three scores: $Accuracy$, $F1$ and $ROC AUC$.

\begin{table}
\caption{Table comparing cross validated results for all of the classifiers run in this work
For each task, we show the triplet of $Accuracy$, $F1$ and $ROC AUC$ test scores after a full cross validation procedures on the learner.}
\label{tab:all_results}
\centering
\begin{tabular*}{0.9\textwidth}{@{\extracolsep{\fill} }  l l l l l }
%{|p{2cm}|p{2cm}|p{1.5cm}|p{1cm}|p{1.5cm}}
\toprule
Measure & Problem 1 & Problem 2 & Problem 3 & Problem 4  \\
\midrule
Naive Bayes     & (0.84,0.75,0.82)  & (0.64,0.31,0.61)  &  (0.65,0.45,0.63)   & (0.85,0.62,0.76)   \\
Logistic Classifier   & (0.97,0,96,0.99)  & (0.35,0.52,0.67)  &  (,,)   & (,,)   \\
Random Forest   & (,,)  & (,,)  &  (,,)   & (,,)   \\
Gradient Boosting   & (,,)  & (,,)  &  (,,)   & (,,)   \\
% CV $Accuracy$ score     & 0.842  & 0.649  &  0.658   &  0.852   \\
% CV $F1$ score           & 0.753  & 0.313  &  0.457   &  0.626   \\
% CV $ROC AUC$ score      & 0.827  & 0.617  &  0.635   &  0.768   \\
\bottomrule
\end{tabular*}
\end{table}




% % With a running time of 80s, 10 steps are generally needed to achieve a 0.977 validation accuracy score on
% The best model was a Logistic Regression Classifier with an $L2$-penalty value of 0.01.
% % Table~\cref{ts}
% The following table
% shows the scores obtained by the selected model on the out-of-sample set.

% \begin{table}\label{tab:results}[ht]
% 	\caption{Resulting scores.}

% 	\centering
% 	\begin{tabular}{ l l }
% 		\toprule
% 		Score & Value \\
% 		\midrule
% 		F1 score & 0.964537  \\
% 		Accuracy & 0.980670  \\
% 		AUC    & 0.991593  \\
% 		Precision & 0.970838  \\
% 		Recall  & 0.958316  \\
% 		\bottomrule
% 	\end{tabular}
% \end{table}


% High values across all scoring measures are achieved
% with the best estimator, with the lowest metric starting at a value of 0.958.
% These results can be explained by the fact that
% %Scores results become less surprising when looking closer at the dataset.
%
%Si del test\_set ahora me quedo con lo users que Y\_target ==1 == EPIDEMIC\_gt pero que hoy en dia son epidemic ==0 (se mudaron)
Los scores bajan mucho:
%
%+--------------+-----------------+-------+
%| target\_label | predicted\_label | count |
%+--------------+-----------------+-------+
%|   1    |    1    | 1944 |
%|   1    |    0    | 3525 |
%+--------------+-----------------+-------+
%'f1\_score': 0.5244840145690004, ''accuracy': 0.3554580,, 'precision': 1.0, 'recall': 0.35545803,



\section{Conclusions}\label{section:conclusions}

ASDF ASDF

\subsection{ Lines of Future Work }

The mobility and social information extracted from CDRs analysis has been shown to be of practical use for long-term human migrations and for Chagas disease research.
it adds value and information to help make data driven decisions which in turn is key to support epidemiological policy interventions in the region.
For the purpose of continuing this line of work, the following is a list of possible extensions:

\begin{description}
    \item [Results validation.] Compare observed experiment results of risk maps and best features against actual serology or disease prevalence surveys.
    Data collected from fieldwork could be fed to the algorithm in order to supervise the learning towards a target variable that is defined by the disease prevalence at a certain level.

    \item [Differentiating rural antennas from urban ones.] This is important as studies show that rural areas have epidemiological conditions which are more favorable to the expansion of the disease expansion.
    The vector-borne transmission through \textit{Trypanosoma cruzi} is helped with poor housing materials and domestic animals.
    All  contribute to complete the parasite's life-cycle.
    Using the CDR data, antennas could be automatically tagged as rural by analyzing the differences between the spatial distribution of the antennas in each area.
    A similar goal could be to identify precarious settlements within urban areas, with the help of census data sources.

    \item [International and seasonal migration analysis.] Experts from the \textit{Mundo Sano} Foundation underlined that many seasonal and international migrations occur in the \textit{Gran Chaco} region.
    Workers are known to leave the endemic area for several months possibly introducing the parasite to foreign populations.
    The same happens with foreign workers, which are not part of our dataset.
    The mobility analysis on particular time periods or events, for example on holidays, or specific migrations from bordering countries, can give information on which communities have a higher influx of people from the endemic zone during a certain period.
    This additional analysis would also lower the bias of the current dataset.


    \item [Search for epidemiological data at a detailed level.] For instance, specific historical infection cases.
    Splitting the endemic region according to the infection rate in different areas, or considering particular infections.
    With the use of a high-quality epidemic dataset, we assume that a more model complex can be built, to detect infected users nationwide.
    It is of primary importance though to have a good dataset available.
    Since otherwise, it would not be possible to reach a generalizing classifier.

    \item Feature exploration to search for correlations with being infected.

\end{description}



% \subsection{Supervised Algorithms}
% \subsection{Brief Technical Notes }


% Results were satisfactory in most algorithms such as Gradient Boosting, Random Forest and Logistic Regression.
% The Multinomial Naive Bayes method was used solely for benchmarking reasons and for its fast runtime and low number of parameters required to configure.
% All algorithms were run on a 16-core Linux machine with 72GB of RAM.\@ Processing and learning scripts were run on Python.
% When available, we showed how the feature importance methods qualified the attributes of the dataset, selecting the most relevant for the experiment.

% \todo{add python, graphlab, sklearn, pandas citations}.


% For the latter part of the job, we used Pandas and Graphlab whilst for the classification part it was mostly implemented with Scikit-learn~\textcite{scikit-learn} and, in some cases, with Graphlab~\textcite{graphlab}.

% We can not recommend highly enough all the aforementioned python modules.
% All were critical in the development of a working experimentation pipeline.




