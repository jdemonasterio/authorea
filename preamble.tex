% !BIB TS-program = biber
% !BIB program = biber
%===============================================================================
%          File: preamble.tex
%        Author: Pedro Ferrari
%       Created: 08 Feb 2017
% Last Modified: 08 Feb 2017
%   Description: Preamble for Thesis File
%===============================================================================
%---------------------------------------+
% Source code, programming and patching |
%---------------------------------------+
\usepackage{etoolbox}    % Toolbox of programming tools
\usepackage{xpatch}      % Extension of etoolbox patching commands

%--------------------------------------+
% Language, hyphenation, encoding, etc |
%--------------------------------------+
\usepackage[english]{babel}
\usepackage{lmodern}             % Use Latin Modern fonts
\usepackage[T1]{fontenc}         % Better output when a diacritic/accent is used
\usepackage[utf8]{inputenc}      % Allows to input accented characters
\usepackage{textcomp}            % Avoid conflicts with siunitx and microtype
\usepackage{microtype}           % Improves justification and typography
\usepackage[svgnames]{xcolor}    % Svgnames option loads navy (blue) colour

%---------------------------------------------+
% Page style: titles, margins, footnotes, etc |
%---------------------------------------------+
% A4 page layout:
\usepackage[width=14cm,left=3.5cm,marginparwidth=3cm,marginparsep=0.35cm,
height=21cm,top=3.7cm,headsep=1cm,footskip=1.1cm]{geometry}

\usepackage[pagestyles,outermarks]{titlesec}  % Customize titles and headers
\newpagestyle{main}[\scshape]{%
  \headrule
  \sethead
  [\thepage][][\chaptertitlename\space\thechapter. \chaptertitle]
  {\ifthesection{\thesection\space\,\sectiontitle}
  {\chaptertitlename\space\thechapter. \chaptertitle}}{}{\thepage}
}
\newpagestyle{special}[\scshape]{%
  \headrule
  \sethead
  [\thepage][][\chaptertitle]
  {\ifthesection{\sectiontitle}{\chaptertitle}}{}{\thepage}
}
% The following pagestyle is needed because titlesec isn't compatible with
% refsegment=chapter
\newpagestyle{bibatend}[\scshape]{
  \headrule
  \sethead
  [\thepage][][\chaptertitle]
  {\sectiontitle}{}{\thepage}
}
\pagestyle{special}
\appto{\mainmatter}{\pagestyle{main}}
\appto{\backmatter}{\pagestyle{bibatend}}
\appto{\printindex}{\pagestyle{special}}

% Use empty page style instead of plain in parts and chapters title pages
\patchcmd{\part}{plain}{empty}{}{}
\patchcmd{\chapter}{plain}{empty}{}{}

\usepackage{emptypage}  % Empty blank pages created by \cleardoublepage

% Change chapter heading style to match titlepage
\titleformat{\chapter}[display]
{\bfseries\filcenter}
{\titlerule[1.5pt]\vspace{4ex}%
\LARGE{\chaptertitlename\space\thechapter}}{0.5cm}{\huge}
[\vspace{2ex}{\titlerule[1.5pt]}\vspace{0.3cm}]
% Do the same for unnumbered chapters (TOC, preface, etc)
\titleformat{name=\chapter,numberless}[display]
{\bfseries\filcenter}
{\titlerule[1.5pt]\vspace{4ex}}{0.5cm}{\huge}
[\vspace{2ex}{\titlerule[1.5pt]}\vspace{0.3cm}]

\usepackage[stable,multiple]{footmisc}  % Customizations of footnotes
\renewcommand*{\footnoterule}{\vspace*{0.3cm}\hrule width 2.5cm\vspace*{0.3cm}}
\makeatletter
  \renewcommand\@makefntext[1]{
  \setlength{\parindent}{15pt}\mbox{\@thefnmark.\space}{#1}}
\makeatother

%-------------------------------+
% Math symbols and environments |
%-------------------------------+
\usepackage{amsmath}               % Load new math environments
\numberwithin{equation}{section}
\usepackage{amssymb}               % Defines most math symbols (such as \mathbb)
\usepackage{mathtools}             % Extension and bug fixes for amsmath package
\usepackage{mathrsfs}              % Math script like font
\usepackage{breqn}                 % Automatic line breaking of math expressions
\renewcommand*{\intlimits}{\displaylimits}  % Fix breqn clash with intlimits


%---------------------+
% Floats and captions |
%---------------------+
\usepackage{graphicx}           % To include graphics files
%\graphicspath{{/Users/Pedro/OneDrive/programming/Latex/logos/}{figures/}{tables/}}
\usepackage{pdflscape}
\usepackage[font=small,labelfont=bf]{caption}
\captionsetup*[figure]{format=plain,justification=centerlast,labelsep=quad}
\captionsetup*[table]{justification=centering,labelsep=newline}
\numberwithin{figure}{section}
\numberwithin{table}{section}

% Use subcaption for subfigures (to work properly with hyperref)
\usepackage{subcaption}
\captionsetup*[subfigure]{subrefformat=simple,labelformat=simple}
\renewcommand*{\thesubfigure}{(\alph{subfigure})}

% Further modifications of float layout
\usepackage[captionskip=5pt]{floatrow}  % We set caption skip here
\floatsetup[table]{style=Plaintop,font=small,footnoterule=none,footskip=2.5pt}

%----------------+
% Table packages |
%----------------+
\usepackage{array}          % Flexible column formatting
% \usepackage{spreadtab}  % Spreadsheet features
\usepackage{multirow}       % Allows table cells that span more than one row
\usepackage{booktabs}       % Enhance quality of tables
\setlength{\heavyrulewidth}{1pt}

% \usepackage{longtable}        % Allows to break tables through pages
% \floatsetup[longtable]{margins=centering,LTcapwidth=table}


%---------------------------------------------------------+
% Miscellaneous packages: lists, setspace, todonotes, etc |
%---------------------------------------------------------+
\usepackage[shortlabels,inline]{enumitem}   % Customize lists
\setlist[itemize,1]{label=$\bullet$}
\setlist[itemize,2]{label=\footnotesize{$\blacktriangleright$}}
\setlist[itemize,3]{label=\tiny{$\blacksquare$}}
\setlist[itemize,4]{label=\bfseries{\large{--}}}
% \setlist[enumerate,2]{label=\emph{\alph*})}
\newlist{steps}{enumerate}{1}               % List of steps to be used in proofs
\setlist[steps,1]{leftmargin=*,label=\textit{Step \arabic*.},ref=\arabic*}

\usepackage{setspace}  % Commands for double and one-and-a-half spacing
\setstretch{1.2}       % 1.2 spacing

% \usepackage{listings}  % Useful for inserting code (no unicode support)
% \lstset{basicstyle=\small\ttfamily}

% \usepackage[colorinlistoftodos,textsize=small,figheight=5cm,
% figwidth=10cm,color=red!85]{todonotes}

% \usepackage{lipsum}    % Dummy text generator

\usepackage{algorithm2e} % to write pseudo-code algorithms
\usepackage{qtree} % To draw trees
\usepackage{todonotes} % To draw trees

%----------------------------------+
% Appendix, bibliography and index |
%----------------------------------+
% Solve bad interaction between titlesec and \appendix
\preto{\appendix}{\cleardoublepage}

% this might insert blank pages here and there but now compilation works correctly with biber
\usepackage[style=american]{csquotes}  % Language sensitive quotation facilities
\usepackage[style=authoryear-comp,backref=true,refsection=chapter,backend=biber]{biblatex}

% \usepackage[
% 	backend=biber,
% 	sortlocale=us_EN,
% 	natbib=true,
% 	style=authoryear-comp,backref=true,refsection=chapter,
% 	url=false,
% 	doi=true,
% 	eprint=false
% 	]{biblatex}

\addbibresource{biblio.bib}

% Bibliography format
\usepackage{mybibformat} % Modifications to authoryear-comp style and hyperlinks
\setlength{\bibitemsep}{0.1cm}

\usepackage{imakeidx}  % Creation and formatting of indexes
\indexsetup{level=\chapter,firstpagestyle=empty,othercode=\small}
\makeindex[title=Alphabetical Index]

%------------------------------------------------------+
% Hyperlinks, bookmarks, theorems and cross-references |
%------------------------------------------------------+
\usepackage[hyperfootnotes=false]{hyperref}
\hypersetup{colorlinks=true, allcolors=Navy, linktoc=page,
pdfstartview={XYZ null null 1}, pdfcreator={Vim LaTeX},
pdfsubject={Machine Learning},
pdftitle={Math thesis},
pdfauthor={Juan Mateo De Monasterio},
pdfkeywords={machine learning}
}
\usepackage[numbered,open,openlevel=1]{bookmark}

\usepackage{amsthm,amsmath}        % We load theorem environments here to avoid warnings

\usepackage[noabbrev,capitalise]{cleveref}

%-----------------------------------------------------------+
% Comment out the math equations, figurest,etc                |
% This will help us spell checking the doc with online tools |
%-----------------------------------------------------------+

% These allow one to compile the pdf without any figures, equations
% etc. which will let us copy paste the output for spell checking.

%\usepackage{comment}
%\excludecomment{figure}
%\let\endfigure\relax
%\excludecomment{equation}
%\let\endequation\relax

%\excludecomment{lstlisting}
%\let\endlstlisting\relax



%------------------------------------+
% Definition of theorem environments |
%------------------------------------+
% Declare theorem styles that remove final dot and use bold font for notes
\newtheoremstyle{plaindotless}{\topsep}{\topsep}{\itshape}{0pt}{\bfseries}{}%
{5pt plus 1pt minus 1pt}{\thmname{#1}\thmnumber{ #2}\bfseries{\thmnote{ (#3)}}}
\newtheoremstyle{definitiondotless}{\topsep}{\topsep}{\normalfont}{0pt}%
{\bfseries}{}{5pt plus 1pt minus 1pt}%
{\thmname{#1}\thmnumber{ #2}\bfseries{\thmnote{ (#3)}}}
\newtheoremstyle{remarkdotless}{0.5\topsep}{0.5\topsep}{\normalfont}{0pt}%
{\itshape}{}{5pt plus 1pt minus 1pt}%
{\thmname{#1}\normalfont\thmnumber{ #2}\itshape{\thmnote{ (#3)}}}

% Define style dependent environments and number them consecutively per section
\theoremstyle{plaindotless}
\newtheorem{theorem}{Theorem}[section]
\newtheorem*{theorem*}{Theorem.}
\newtheorem{proposition}[theorem]{Proposition}
\newtheorem*{proposition*}{Proposition.}
\newtheorem{lemma}[theorem]{Lemma}
\newtheorem*{lemma*}{Lemma.}
\newtheorem{corollary}[theorem]{Corollary}
\newtheorem*{corollary*}{Corollary.}

\theoremstyle{definitiondotless}
\newtheorem{definition}[theorem]{Definition}
\newtheorem*{definition*}{Definition.}
\newtheorem{examplex}[theorem]{Example}
\newtheorem*{examplestarred}{Example.}
\newtheorem*{continuedex}{Example \continuedexref\space Continued.}
\newtheorem{exercise}[theorem]{Exercise}
\newtheorem*{exercise*}{Exercise.}
\newtheorem*{solution*}{Solution.}
\newtheorem{problem}{Problem}

\theoremstyle{remarkdotless}
\newtheorem{remark}[theorem]{Remark}
\newtheorem*{remark*}{Remark.}
\newtheorem*{notation*}{Notation.}

% Define numbered, unnumbered and continued examples with triangle end mark
\newcommand{\myqedsymbol}{\ensuremath{\triangle}}

\newenvironment{example}
  {\pushQED{\qed} \renewcommand{\qedsymbol}{\myqedsymbol}\examplex}
  {\popQED\endexamplex}

\newenvironment{example*}
  {\pushQED{\qed}\renewcommand{\qedsymbol}{\myqedsymbol}\examplestarred}
  {\popQED\endexamplestarred}

\newenvironment{examcont}[1]
  {\pushQED{\qed}\renewcommand{\qedsymbol}{\myqedsymbol}%
    \newcommand{\continuedexref}{\ref*{#1}}\continuedex}
  {\popQED\endcontinuedex}


%-----------------------------------------------+
% Cross-references settings (cleveref settings) |
%-----------------------------------------------+

\crefname{exercise}{Exercise}{Exercises}
\crefname{enumerate}{Enumeration}{Enumerations}
\crefname{stepsi}{Step}{Steps}
\crefname{problem}{Problem}{Problems}

\crefname{equation}{}{}
\crefformat{equation}{#2(#1)#3}
\crefrangeformat{equation}{#3(#1)#4 to #5(#2)#6}
\crefmultiformat{equation}{#2(#1)#3}{ and #2(#1)#3}{, #2(#1)#3}{ and #2(#1)#3}
\crefrangemultiformat{equation}{#3(#1)#4 to #5(#2)#6}{ and #3(#1)#4 to #5(#2)#6}%
{, #3(#1)#4 to #5(#2)#6}{ and #3(#1)#4 to #5(#2)#6}

%----------------------------------------------+
% Half-title, titlepage and copyright settings |
%----------------------------------------------+

\newcommand*{\halftitlepg}{%
  \begingroup
    \begin{center}
      \textbf{\huge{Math Thesis}}
    \end{center}
  \endgroup
  \thispagestyle{empty}\cleardoublepage
}

\newcommand*{\titlepg}{%
  \begingroup
    \vspace{0.01\textheight}
    \begin{center}
      \textbf{\LARGE{Universidad de Buenos Aires}}\\
      \vspace{0.02\textheight}
      \textbf{\Large{Facultad de Ciencias Exactas}}\\
      \vspace{0.3\textheight}
      \rule{\textwidth}{1.5pt}\par
      \vspace{\baselineskip}
 {\bfseries\Huge{Explorando migraciones chagásicas en latinoamérica con aprendizaje automático}\par
 	\bigskip\Large{Exploring Latino American chagasic migrations with machine learning}}\\
      \vspace{\baselineskip}
      \rule{\textwidth}{1.5pt}\par
      \vfill
      \textsc{\huge{Juan Mateo De Monasterio}}
      \vfill
      \textbf{\Large{\today}}
    \end{center}
  \endgroup
  \thispagestyle{empty}\clearpage
}

\newcommand*{\copyrightpg}{%
  \begingroup
    \footnotesize
    \parindent 0pt
    \null
    \vfill
    \textcopyright{} 2017 Juan Mateo De Monasterio. All rights reserved.\par
    \vspace{\baselineskip}
    This document is free; you can redistribute it and/or modify it under the
    terms of the GNU General Public License as published by the Free Software
    Foundation; either version 2 of the License, or (at your choice) any later
    version.\par
    \vspace{\baselineskip}
    This document was typeset in Latin Modern font using \LaTeX.\par
  \endgroup
  \thispagestyle{empty}\clearpage
}

\newcommand*{\dedication}{%
  \begingroup
    \vspace*{0.3\textheight}
    \begin{center}
      \emph{\large{To all.}}
      \end{center}
    \endgroup
  \thispagestyle{empty}\cleardoublepage
}

%-------------------+
% Table of contents |
%-------------------+
% Add bookmark for table of contents and increase spacing of items
\preto{\tableofcontents}{\cleardoublepage\pdfbookmark[0]{\contentsname}{toc}%
  \setstretch{1.1}}
\appto{\tableofcontents}{\singlespacing}

%---------------------------------------------------+
% (Re)Definition of new commands and math operators |
%---------------------------------------------------+
% Numbers
\DeclareMathOperator{\N}{\mathbb{N}}
\DeclareMathOperator{\Z}{\mathbb{Z}}
\DeclareMathOperator{\Q}{\mathbb{Q}}
\DeclareMathOperator{\R}{\mathbb{R}}
% Probability
\DeclareMathOperator{\E}{\mathbb{E}}
\DeclareMathOperator{\Expect}{\mathbb{E}}
\DeclareMathOperator{\Var}{\mathrm{Var}}
\DeclareMathOperator{\Cov}{\mathrm{Cov}}
% Delimiters
\DeclarePairedDelimiter{\abs}{\lvert}{\rvert}
\DeclarePairedDelimiter{\norm}{\lvert\lvert}{\rvert\rvert}
% Miscellaneous
\renewcommand{\d}{\ensuremath{\operatorname{d}\!}}  % Differential
\renewcommand{\L}{\ensuremath{\operatorname{\mathcal{L}}}}  % Lagrangian
\DeclareMathOperator{\calN}{\mathcal{N}}
\DeclareMathOperator{\calG}{\mathcal{G}}
\DeclareMathOperator{\calL}{\mathcal{L}}
\DeclareMathOperator{\calE}{\mathcal{E}}
% argmin and max operators
\DeclareMathOperator*{\argmin}{argmin} % no space, limits underneath in displays
% \DeclareMathOperator{\argmin}{argmin} % no space, limits on side in displays
\DeclareMathOperator*{\argmax}{argmax} % no space, limits underneath in displays
% \DeclareMathOperator{\argmax}{argmax} % no space, limits on side in displays



\newcommand\MyBox[2]{
	\fbox{\lower0.75cm
		\vbox to 1.7cm{\vfil
			\hbox to 1.7cm{\hfil\parbox{1.4cm}{#1\\#2}\hfil}
			\vfil}%
	}%
}
